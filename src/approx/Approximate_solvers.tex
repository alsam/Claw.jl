\documentclass[11pt]{article}

    \usepackage[breakable]{tcolorbox}
    \usepackage{parskip} % Stop auto-indenting (to mimic markdown behaviour)
    
    \usepackage{iftex}
    \ifPDFTeX
    	\usepackage[T1]{fontenc}
    	\usepackage{mathpazo}
    \else
    	\usepackage{fontspec}
    \fi

    % Basic figure setup, for now with no caption control since it's done
    % automatically by Pandoc (which extracts ![](path) syntax from Markdown).
    \usepackage{graphicx}
    % Maintain compatibility with old templates. Remove in nbconvert 6.0
    \let\Oldincludegraphics\includegraphics
    % Ensure that by default, figures have no caption (until we provide a
    % proper Figure object with a Caption API and a way to capture that
    % in the conversion process - todo).
    \usepackage{caption}
    \DeclareCaptionFormat{nocaption}{}
    \captionsetup{format=nocaption,aboveskip=0pt,belowskip=0pt}

    \usepackage[Export]{adjustbox} % Used to constrain images to a maximum size
    \adjustboxset{max size={0.9\linewidth}{0.9\paperheight}}
    \usepackage{float}
    \floatplacement{figure}{H} % forces figures to be placed at the correct location
    \usepackage{xcolor} % Allow colors to be defined
    \usepackage{enumerate} % Needed for markdown enumerations to work
    \usepackage{geometry} % Used to adjust the document margins
    \usepackage{amsmath} % Equations
    \usepackage{amssymb} % Equations
    \usepackage{textcomp} % defines textquotesingle
    % Hack from http://tex.stackexchange.com/a/47451/13684:
    \AtBeginDocument{%
        \def\PYZsq{\textquotesingle}% Upright quotes in Pygmentized code
    }
    \usepackage{upquote} % Upright quotes for verbatim code
    \usepackage{eurosym} % defines \euro
    \usepackage[mathletters]{ucs} % Extended unicode (utf-8) support
    \usepackage{fancyvrb} % verbatim replacement that allows latex
    \usepackage{grffile} % extends the file name processing of package graphics 
                         % to support a larger range
    \makeatletter % fix for grffile with XeLaTeX
    \def\Gread@@xetex#1{%
      \IfFileExists{"\Gin@base".bb}%
      {\Gread@eps{\Gin@base.bb}}%
      {\Gread@@xetex@aux#1}%
    }
    \makeatother

    % The hyperref package gives us a pdf with properly built
    % internal navigation ('pdf bookmarks' for the table of contents,
    % internal cross-reference links, web links for URLs, etc.)
    \usepackage{hyperref}
    % The default LaTeX title has an obnoxious amount of whitespace. By default,
    % titling removes some of it. It also provides customization options.
    \usepackage{titling}
    \usepackage{longtable} % longtable support required by pandoc >1.10
    \usepackage{booktabs}  % table support for pandoc > 1.12.2
    \usepackage[inline]{enumitem} % IRkernel/repr support (it uses the enumerate* environment)
    \usepackage[normalem]{ulem} % ulem is needed to support strikethroughs (\sout)
                                % normalem makes italics be italics, not underlines
    \usepackage{mathrsfs}
    

    
    % Colors for the hyperref package
    \definecolor{urlcolor}{rgb}{0,.145,.698}
    \definecolor{linkcolor}{rgb}{.71,0.21,0.01}
    \definecolor{citecolor}{rgb}{.12,.54,.11}

    % ANSI colors
    \definecolor{ansi-black}{HTML}{3E424D}
    \definecolor{ansi-black-intense}{HTML}{282C36}
    \definecolor{ansi-red}{HTML}{E75C58}
    \definecolor{ansi-red-intense}{HTML}{B22B31}
    \definecolor{ansi-green}{HTML}{00A250}
    \definecolor{ansi-green-intense}{HTML}{007427}
    \definecolor{ansi-yellow}{HTML}{DDB62B}
    \definecolor{ansi-yellow-intense}{HTML}{B27D12}
    \definecolor{ansi-blue}{HTML}{208FFB}
    \definecolor{ansi-blue-intense}{HTML}{0065CA}
    \definecolor{ansi-magenta}{HTML}{D160C4}
    \definecolor{ansi-magenta-intense}{HTML}{A03196}
    \definecolor{ansi-cyan}{HTML}{60C6C8}
    \definecolor{ansi-cyan-intense}{HTML}{258F8F}
    \definecolor{ansi-white}{HTML}{C5C1B4}
    \definecolor{ansi-white-intense}{HTML}{A1A6B2}
    \definecolor{ansi-default-inverse-fg}{HTML}{FFFFFF}
    \definecolor{ansi-default-inverse-bg}{HTML}{000000}

    % commands and environments needed by pandoc snippets
    % extracted from the output of `pandoc -s`
    \providecommand{\tightlist}{%
      \setlength{\itemsep}{0pt}\setlength{\parskip}{0pt}}
    \DefineVerbatimEnvironment{Highlighting}{Verbatim}{commandchars=\\\{\}}
    % Add ',fontsize=\small' for more characters per line
    \newenvironment{Shaded}{}{}
    \newcommand{\KeywordTok}[1]{\textcolor[rgb]{0.00,0.44,0.13}{\textbf{{#1}}}}
    \newcommand{\DataTypeTok}[1]{\textcolor[rgb]{0.56,0.13,0.00}{{#1}}}
    \newcommand{\DecValTok}[1]{\textcolor[rgb]{0.25,0.63,0.44}{{#1}}}
    \newcommand{\BaseNTok}[1]{\textcolor[rgb]{0.25,0.63,0.44}{{#1}}}
    \newcommand{\FloatTok}[1]{\textcolor[rgb]{0.25,0.63,0.44}{{#1}}}
    \newcommand{\CharTok}[1]{\textcolor[rgb]{0.25,0.44,0.63}{{#1}}}
    \newcommand{\StringTok}[1]{\textcolor[rgb]{0.25,0.44,0.63}{{#1}}}
    \newcommand{\CommentTok}[1]{\textcolor[rgb]{0.38,0.63,0.69}{\textit{{#1}}}}
    \newcommand{\OtherTok}[1]{\textcolor[rgb]{0.00,0.44,0.13}{{#1}}}
    \newcommand{\AlertTok}[1]{\textcolor[rgb]{1.00,0.00,0.00}{\textbf{{#1}}}}
    \newcommand{\FunctionTok}[1]{\textcolor[rgb]{0.02,0.16,0.49}{{#1}}}
    \newcommand{\RegionMarkerTok}[1]{{#1}}
    \newcommand{\ErrorTok}[1]{\textcolor[rgb]{1.00,0.00,0.00}{\textbf{{#1}}}}
    \newcommand{\NormalTok}[1]{{#1}}
    
    % Additional commands for more recent versions of Pandoc
    \newcommand{\ConstantTok}[1]{\textcolor[rgb]{0.53,0.00,0.00}{{#1}}}
    \newcommand{\SpecialCharTok}[1]{\textcolor[rgb]{0.25,0.44,0.63}{{#1}}}
    \newcommand{\VerbatimStringTok}[1]{\textcolor[rgb]{0.25,0.44,0.63}{{#1}}}
    \newcommand{\SpecialStringTok}[1]{\textcolor[rgb]{0.73,0.40,0.53}{{#1}}}
    \newcommand{\ImportTok}[1]{{#1}}
    \newcommand{\DocumentationTok}[1]{\textcolor[rgb]{0.73,0.13,0.13}{\textit{{#1}}}}
    \newcommand{\AnnotationTok}[1]{\textcolor[rgb]{0.38,0.63,0.69}{\textbf{\textit{{#1}}}}}
    \newcommand{\CommentVarTok}[1]{\textcolor[rgb]{0.38,0.63,0.69}{\textbf{\textit{{#1}}}}}
    \newcommand{\VariableTok}[1]{\textcolor[rgb]{0.10,0.09,0.49}{{#1}}}
    \newcommand{\ControlFlowTok}[1]{\textcolor[rgb]{0.00,0.44,0.13}{\textbf{{#1}}}}
    \newcommand{\OperatorTok}[1]{\textcolor[rgb]{0.40,0.40,0.40}{{#1}}}
    \newcommand{\BuiltInTok}[1]{{#1}}
    \newcommand{\ExtensionTok}[1]{{#1}}
    \newcommand{\PreprocessorTok}[1]{\textcolor[rgb]{0.74,0.48,0.00}{{#1}}}
    \newcommand{\AttributeTok}[1]{\textcolor[rgb]{0.49,0.56,0.16}{{#1}}}
    \newcommand{\InformationTok}[1]{\textcolor[rgb]{0.38,0.63,0.69}{\textbf{\textit{{#1}}}}}
    \newcommand{\WarningTok}[1]{\textcolor[rgb]{0.38,0.63,0.69}{\textbf{\textit{{#1}}}}}
    
    
    % Define a nice break command that doesn't care if a line doesn't already
    % exist.
    \def\br{\hspace*{\fill} \\* }
    % Math Jax compatibility definitions
    \def\gt{>}
    \def\lt{<}
    \let\Oldtex\TeX
    \let\Oldlatex\LaTeX
    \renewcommand{\TeX}{\textrm{\Oldtex}}
    \renewcommand{\LaTeX}{\textrm{\Oldlatex}}
    % Document parameters
    % Document title
    \title{Approximate\_solvers}
    
    
    
    
    
% Pygments definitions
\makeatletter
\def\PY@reset{\let\PY@it=\relax \let\PY@bf=\relax%
    \let\PY@ul=\relax \let\PY@tc=\relax%
    \let\PY@bc=\relax \let\PY@ff=\relax}
\def\PY@tok#1{\csname PY@tok@#1\endcsname}
\def\PY@toks#1+{\ifx\relax#1\empty\else%
    \PY@tok{#1}\expandafter\PY@toks\fi}
\def\PY@do#1{\PY@bc{\PY@tc{\PY@ul{%
    \PY@it{\PY@bf{\PY@ff{#1}}}}}}}
\def\PY#1#2{\PY@reset\PY@toks#1+\relax+\PY@do{#2}}

\expandafter\def\csname PY@tok@w\endcsname{\def\PY@tc##1{\textcolor[rgb]{0.73,0.73,0.73}{##1}}}
\expandafter\def\csname PY@tok@c\endcsname{\let\PY@it=\textit\def\PY@tc##1{\textcolor[rgb]{0.25,0.50,0.50}{##1}}}
\expandafter\def\csname PY@tok@cp\endcsname{\def\PY@tc##1{\textcolor[rgb]{0.74,0.48,0.00}{##1}}}
\expandafter\def\csname PY@tok@k\endcsname{\let\PY@bf=\textbf\def\PY@tc##1{\textcolor[rgb]{0.00,0.50,0.00}{##1}}}
\expandafter\def\csname PY@tok@kp\endcsname{\def\PY@tc##1{\textcolor[rgb]{0.00,0.50,0.00}{##1}}}
\expandafter\def\csname PY@tok@kt\endcsname{\def\PY@tc##1{\textcolor[rgb]{0.69,0.00,0.25}{##1}}}
\expandafter\def\csname PY@tok@o\endcsname{\def\PY@tc##1{\textcolor[rgb]{0.40,0.40,0.40}{##1}}}
\expandafter\def\csname PY@tok@ow\endcsname{\let\PY@bf=\textbf\def\PY@tc##1{\textcolor[rgb]{0.67,0.13,1.00}{##1}}}
\expandafter\def\csname PY@tok@nb\endcsname{\def\PY@tc##1{\textcolor[rgb]{0.00,0.50,0.00}{##1}}}
\expandafter\def\csname PY@tok@nf\endcsname{\def\PY@tc##1{\textcolor[rgb]{0.00,0.00,1.00}{##1}}}
\expandafter\def\csname PY@tok@nc\endcsname{\let\PY@bf=\textbf\def\PY@tc##1{\textcolor[rgb]{0.00,0.00,1.00}{##1}}}
\expandafter\def\csname PY@tok@nn\endcsname{\let\PY@bf=\textbf\def\PY@tc##1{\textcolor[rgb]{0.00,0.00,1.00}{##1}}}
\expandafter\def\csname PY@tok@ne\endcsname{\let\PY@bf=\textbf\def\PY@tc##1{\textcolor[rgb]{0.82,0.25,0.23}{##1}}}
\expandafter\def\csname PY@tok@nv\endcsname{\def\PY@tc##1{\textcolor[rgb]{0.10,0.09,0.49}{##1}}}
\expandafter\def\csname PY@tok@no\endcsname{\def\PY@tc##1{\textcolor[rgb]{0.53,0.00,0.00}{##1}}}
\expandafter\def\csname PY@tok@nl\endcsname{\def\PY@tc##1{\textcolor[rgb]{0.63,0.63,0.00}{##1}}}
\expandafter\def\csname PY@tok@ni\endcsname{\let\PY@bf=\textbf\def\PY@tc##1{\textcolor[rgb]{0.60,0.60,0.60}{##1}}}
\expandafter\def\csname PY@tok@na\endcsname{\def\PY@tc##1{\textcolor[rgb]{0.49,0.56,0.16}{##1}}}
\expandafter\def\csname PY@tok@nt\endcsname{\let\PY@bf=\textbf\def\PY@tc##1{\textcolor[rgb]{0.00,0.50,0.00}{##1}}}
\expandafter\def\csname PY@tok@nd\endcsname{\def\PY@tc##1{\textcolor[rgb]{0.67,0.13,1.00}{##1}}}
\expandafter\def\csname PY@tok@s\endcsname{\def\PY@tc##1{\textcolor[rgb]{0.73,0.13,0.13}{##1}}}
\expandafter\def\csname PY@tok@sd\endcsname{\let\PY@it=\textit\def\PY@tc##1{\textcolor[rgb]{0.73,0.13,0.13}{##1}}}
\expandafter\def\csname PY@tok@si\endcsname{\let\PY@bf=\textbf\def\PY@tc##1{\textcolor[rgb]{0.73,0.40,0.53}{##1}}}
\expandafter\def\csname PY@tok@se\endcsname{\let\PY@bf=\textbf\def\PY@tc##1{\textcolor[rgb]{0.73,0.40,0.13}{##1}}}
\expandafter\def\csname PY@tok@sr\endcsname{\def\PY@tc##1{\textcolor[rgb]{0.73,0.40,0.53}{##1}}}
\expandafter\def\csname PY@tok@ss\endcsname{\def\PY@tc##1{\textcolor[rgb]{0.10,0.09,0.49}{##1}}}
\expandafter\def\csname PY@tok@sx\endcsname{\def\PY@tc##1{\textcolor[rgb]{0.00,0.50,0.00}{##1}}}
\expandafter\def\csname PY@tok@m\endcsname{\def\PY@tc##1{\textcolor[rgb]{0.40,0.40,0.40}{##1}}}
\expandafter\def\csname PY@tok@gh\endcsname{\let\PY@bf=\textbf\def\PY@tc##1{\textcolor[rgb]{0.00,0.00,0.50}{##1}}}
\expandafter\def\csname PY@tok@gu\endcsname{\let\PY@bf=\textbf\def\PY@tc##1{\textcolor[rgb]{0.50,0.00,0.50}{##1}}}
\expandafter\def\csname PY@tok@gd\endcsname{\def\PY@tc##1{\textcolor[rgb]{0.63,0.00,0.00}{##1}}}
\expandafter\def\csname PY@tok@gi\endcsname{\def\PY@tc##1{\textcolor[rgb]{0.00,0.63,0.00}{##1}}}
\expandafter\def\csname PY@tok@gr\endcsname{\def\PY@tc##1{\textcolor[rgb]{1.00,0.00,0.00}{##1}}}
\expandafter\def\csname PY@tok@ge\endcsname{\let\PY@it=\textit}
\expandafter\def\csname PY@tok@gs\endcsname{\let\PY@bf=\textbf}
\expandafter\def\csname PY@tok@gp\endcsname{\let\PY@bf=\textbf\def\PY@tc##1{\textcolor[rgb]{0.00,0.00,0.50}{##1}}}
\expandafter\def\csname PY@tok@go\endcsname{\def\PY@tc##1{\textcolor[rgb]{0.53,0.53,0.53}{##1}}}
\expandafter\def\csname PY@tok@gt\endcsname{\def\PY@tc##1{\textcolor[rgb]{0.00,0.27,0.87}{##1}}}
\expandafter\def\csname PY@tok@err\endcsname{\def\PY@bc##1{\setlength{\fboxsep}{0pt}\fcolorbox[rgb]{1.00,0.00,0.00}{1,1,1}{\strut ##1}}}
\expandafter\def\csname PY@tok@kc\endcsname{\let\PY@bf=\textbf\def\PY@tc##1{\textcolor[rgb]{0.00,0.50,0.00}{##1}}}
\expandafter\def\csname PY@tok@kd\endcsname{\let\PY@bf=\textbf\def\PY@tc##1{\textcolor[rgb]{0.00,0.50,0.00}{##1}}}
\expandafter\def\csname PY@tok@kn\endcsname{\let\PY@bf=\textbf\def\PY@tc##1{\textcolor[rgb]{0.00,0.50,0.00}{##1}}}
\expandafter\def\csname PY@tok@kr\endcsname{\let\PY@bf=\textbf\def\PY@tc##1{\textcolor[rgb]{0.00,0.50,0.00}{##1}}}
\expandafter\def\csname PY@tok@bp\endcsname{\def\PY@tc##1{\textcolor[rgb]{0.00,0.50,0.00}{##1}}}
\expandafter\def\csname PY@tok@fm\endcsname{\def\PY@tc##1{\textcolor[rgb]{0.00,0.00,1.00}{##1}}}
\expandafter\def\csname PY@tok@vc\endcsname{\def\PY@tc##1{\textcolor[rgb]{0.10,0.09,0.49}{##1}}}
\expandafter\def\csname PY@tok@vg\endcsname{\def\PY@tc##1{\textcolor[rgb]{0.10,0.09,0.49}{##1}}}
\expandafter\def\csname PY@tok@vi\endcsname{\def\PY@tc##1{\textcolor[rgb]{0.10,0.09,0.49}{##1}}}
\expandafter\def\csname PY@tok@vm\endcsname{\def\PY@tc##1{\textcolor[rgb]{0.10,0.09,0.49}{##1}}}
\expandafter\def\csname PY@tok@sa\endcsname{\def\PY@tc##1{\textcolor[rgb]{0.73,0.13,0.13}{##1}}}
\expandafter\def\csname PY@tok@sb\endcsname{\def\PY@tc##1{\textcolor[rgb]{0.73,0.13,0.13}{##1}}}
\expandafter\def\csname PY@tok@sc\endcsname{\def\PY@tc##1{\textcolor[rgb]{0.73,0.13,0.13}{##1}}}
\expandafter\def\csname PY@tok@dl\endcsname{\def\PY@tc##1{\textcolor[rgb]{0.73,0.13,0.13}{##1}}}
\expandafter\def\csname PY@tok@s2\endcsname{\def\PY@tc##1{\textcolor[rgb]{0.73,0.13,0.13}{##1}}}
\expandafter\def\csname PY@tok@sh\endcsname{\def\PY@tc##1{\textcolor[rgb]{0.73,0.13,0.13}{##1}}}
\expandafter\def\csname PY@tok@s1\endcsname{\def\PY@tc##1{\textcolor[rgb]{0.73,0.13,0.13}{##1}}}
\expandafter\def\csname PY@tok@mb\endcsname{\def\PY@tc##1{\textcolor[rgb]{0.40,0.40,0.40}{##1}}}
\expandafter\def\csname PY@tok@mf\endcsname{\def\PY@tc##1{\textcolor[rgb]{0.40,0.40,0.40}{##1}}}
\expandafter\def\csname PY@tok@mh\endcsname{\def\PY@tc##1{\textcolor[rgb]{0.40,0.40,0.40}{##1}}}
\expandafter\def\csname PY@tok@mi\endcsname{\def\PY@tc##1{\textcolor[rgb]{0.40,0.40,0.40}{##1}}}
\expandafter\def\csname PY@tok@il\endcsname{\def\PY@tc##1{\textcolor[rgb]{0.40,0.40,0.40}{##1}}}
\expandafter\def\csname PY@tok@mo\endcsname{\def\PY@tc##1{\textcolor[rgb]{0.40,0.40,0.40}{##1}}}
\expandafter\def\csname PY@tok@ch\endcsname{\let\PY@it=\textit\def\PY@tc##1{\textcolor[rgb]{0.25,0.50,0.50}{##1}}}
\expandafter\def\csname PY@tok@cm\endcsname{\let\PY@it=\textit\def\PY@tc##1{\textcolor[rgb]{0.25,0.50,0.50}{##1}}}
\expandafter\def\csname PY@tok@cpf\endcsname{\let\PY@it=\textit\def\PY@tc##1{\textcolor[rgb]{0.25,0.50,0.50}{##1}}}
\expandafter\def\csname PY@tok@c1\endcsname{\let\PY@it=\textit\def\PY@tc##1{\textcolor[rgb]{0.25,0.50,0.50}{##1}}}
\expandafter\def\csname PY@tok@cs\endcsname{\let\PY@it=\textit\def\PY@tc##1{\textcolor[rgb]{0.25,0.50,0.50}{##1}}}

\def\PYZbs{\char`\\}
\def\PYZus{\char`\_}
\def\PYZob{\char`\{}
\def\PYZcb{\char`\}}
\def\PYZca{\char`\^}
\def\PYZam{\char`\&}
\def\PYZlt{\char`\<}
\def\PYZgt{\char`\>}
\def\PYZsh{\char`\#}
\def\PYZpc{\char`\%}
\def\PYZdl{\char`\$}
\def\PYZhy{\char`\-}
\def\PYZsq{\char`\'}
\def\PYZdq{\char`\"}
\def\PYZti{\char`\~}
% for compatibility with earlier versions
\def\PYZat{@}
\def\PYZlb{[}
\def\PYZrb{]}
\makeatother


    % For linebreaks inside Verbatim environment from package fancyvrb. 
    \makeatletter
        \newbox\Wrappedcontinuationbox 
        \newbox\Wrappedvisiblespacebox 
        \newcommand*\Wrappedvisiblespace {\textcolor{red}{\textvisiblespace}} 
        \newcommand*\Wrappedcontinuationsymbol {\textcolor{red}{\llap{\tiny$\m@th\hookrightarrow$}}} 
        \newcommand*\Wrappedcontinuationindent {3ex } 
        \newcommand*\Wrappedafterbreak {\kern\Wrappedcontinuationindent\copy\Wrappedcontinuationbox} 
        % Take advantage of the already applied Pygments mark-up to insert 
        % potential linebreaks for TeX processing. 
        %        {, <, #, %, $, ' and ": go to next line. 
        %        _, }, ^, &, >, - and ~: stay at end of broken line. 
        % Use of \textquotesingle for straight quote. 
        \newcommand*\Wrappedbreaksatspecials {% 
            \def\PYGZus{\discretionary{\char`\_}{\Wrappedafterbreak}{\char`\_}}% 
            \def\PYGZob{\discretionary{}{\Wrappedafterbreak\char`\{}{\char`\{}}% 
            \def\PYGZcb{\discretionary{\char`\}}{\Wrappedafterbreak}{\char`\}}}% 
            \def\PYGZca{\discretionary{\char`\^}{\Wrappedafterbreak}{\char`\^}}% 
            \def\PYGZam{\discretionary{\char`\&}{\Wrappedafterbreak}{\char`\&}}% 
            \def\PYGZlt{\discretionary{}{\Wrappedafterbreak\char`\<}{\char`\<}}% 
            \def\PYGZgt{\discretionary{\char`\>}{\Wrappedafterbreak}{\char`\>}}% 
            \def\PYGZsh{\discretionary{}{\Wrappedafterbreak\char`\#}{\char`\#}}% 
            \def\PYGZpc{\discretionary{}{\Wrappedafterbreak\char`\%}{\char`\%}}% 
            \def\PYGZdl{\discretionary{}{\Wrappedafterbreak\char`\$}{\char`\$}}% 
            \def\PYGZhy{\discretionary{\char`\-}{\Wrappedafterbreak}{\char`\-}}% 
            \def\PYGZsq{\discretionary{}{\Wrappedafterbreak\textquotesingle}{\textquotesingle}}% 
            \def\PYGZdq{\discretionary{}{\Wrappedafterbreak\char`\"}{\char`\"}}% 
            \def\PYGZti{\discretionary{\char`\~}{\Wrappedafterbreak}{\char`\~}}% 
        } 
        % Some characters . , ; ? ! / are not pygmentized. 
        % This macro makes them "active" and they will insert potential linebreaks 
        \newcommand*\Wrappedbreaksatpunct {% 
            \lccode`\~`\.\lowercase{\def~}{\discretionary{\hbox{\char`\.}}{\Wrappedafterbreak}{\hbox{\char`\.}}}% 
            \lccode`\~`\,\lowercase{\def~}{\discretionary{\hbox{\char`\,}}{\Wrappedafterbreak}{\hbox{\char`\,}}}% 
            \lccode`\~`\;\lowercase{\def~}{\discretionary{\hbox{\char`\;}}{\Wrappedafterbreak}{\hbox{\char`\;}}}% 
            \lccode`\~`\:\lowercase{\def~}{\discretionary{\hbox{\char`\:}}{\Wrappedafterbreak}{\hbox{\char`\:}}}% 
            \lccode`\~`\?\lowercase{\def~}{\discretionary{\hbox{\char`\?}}{\Wrappedafterbreak}{\hbox{\char`\?}}}% 
            \lccode`\~`\!\lowercase{\def~}{\discretionary{\hbox{\char`\!}}{\Wrappedafterbreak}{\hbox{\char`\!}}}% 
            \lccode`\~`\/\lowercase{\def~}{\discretionary{\hbox{\char`\/}}{\Wrappedafterbreak}{\hbox{\char`\/}}}% 
            \catcode`\.\active
            \catcode`\,\active 
            \catcode`\;\active
            \catcode`\:\active
            \catcode`\?\active
            \catcode`\!\active
            \catcode`\/\active 
            \lccode`\~`\~ 	
        }
    \makeatother

    \let\OriginalVerbatim=\Verbatim
    \makeatletter
    \renewcommand{\Verbatim}[1][1]{%
        %\parskip\z@skip
        \sbox\Wrappedcontinuationbox {\Wrappedcontinuationsymbol}%
        \sbox\Wrappedvisiblespacebox {\FV@SetupFont\Wrappedvisiblespace}%
        \def\FancyVerbFormatLine ##1{\hsize\linewidth
            \vtop{\raggedright\hyphenpenalty\z@\exhyphenpenalty\z@
                \doublehyphendemerits\z@\finalhyphendemerits\z@
                \strut ##1\strut}%
        }%
        % If the linebreak is at a space, the latter will be displayed as visible
        % space at end of first line, and a continuation symbol starts next line.
        % Stretch/shrink are however usually zero for typewriter font.
        \def\FV@Space {%
            \nobreak\hskip\z@ plus\fontdimen3\font minus\fontdimen4\font
            \discretionary{\copy\Wrappedvisiblespacebox}{\Wrappedafterbreak}
            {\kern\fontdimen2\font}%
        }%
        
        % Allow breaks at special characters using \PYG... macros.
        \Wrappedbreaksatspecials
        % Breaks at punctuation characters . , ; ? ! and / need catcode=\active 	
        \OriginalVerbatim[#1,codes*=\Wrappedbreaksatpunct]%
    }
    \makeatother

    % Exact colors from NB
    \definecolor{incolor}{HTML}{303F9F}
    \definecolor{outcolor}{HTML}{D84315}
    \definecolor{cellborder}{HTML}{CFCFCF}
    \definecolor{cellbackground}{HTML}{F7F7F7}
    
    % prompt
    \makeatletter
    \newcommand{\boxspacing}{\kern\kvtcb@left@rule\kern\kvtcb@boxsep}
    \makeatother
    \newcommand{\prompt}[4]{
        \ttfamily\llap{{\color{#2}[#3]:\hspace{3pt}#4}}\vspace{-\baselineskip}
    }
    

    
    % Prevent overflowing lines due to hard-to-break entities
    \sloppy 
    % Setup hyperref package
    \hypersetup{
      breaklinks=true,  % so long urls are correctly broken across lines
      colorlinks=true,
      urlcolor=urlcolor,
      linkcolor=linkcolor,
      citecolor=citecolor,
      }
    % Slightly bigger margins than the latex defaults
    
    \geometry{verbose,tmargin=1in,bmargin=1in,lmargin=1in,rmargin=1in}
    
    

\begin{document}
    
    \maketitle
    
    

    
    \hypertarget{part-ii.-approximate-riemann-solvers}{%
\section{Part II. Approximate Riemann
Solvers}\label{part-ii.-approximate-riemann-solvers}}

    In Part II of this book we present a number of \emph{approximate Riemann
solvers}. We have already seen that for many important hyperbolic
systems it is possible to work out the exact Riemann solution for
arbitrary left and right states. However, for complicated nonlinear
systems, such as the Euler equations (see \href{Euler.ipynb}{Euler}),
this exact solution can only be determined by solving a nonlinear system
of algebraic equations for the intermediate states and the waves that
connect them. This can be done to arbitrary precision, but only at some
computational expense. The cost of exactly solving a single Riemann
problem may seem insignificant, but it can become prohibitively
expensive when the Riemann solver is used as a building block in a
finite volume method. In this case a Riemann problem must be solved at
every cell edge at every time step.

For example, if we consider a very coarse grid in one space dimension
with only 100 cells and take 100 time steps, then 10,000 Riemann
problems must be solved. In solving practical problems in two or three
space dimensions it is not unusual to require the solution of billions
or trillions of Riemann problems. In this context it can be very
important to develop efficient approximate Riemann solvers that quickly
produce a sufficiently good approximation to the true Riemann solution.

    The following points have helped to guide the development of approximate
Riemann solvers:

\begin{itemize}
\item
  If the solution is smooth over much of the domain, then the jump in
  states between neighboring cells will be very small (on the order of
  \(\Delta x\), the cell size) for most of the Riemann problems
  encountered in the numerical solution. Even if the hyperbolic system
  being studied is nonlinear, for such data the equations can be
  approximated by a linearization and we have seen that linear Riemann
  problems can be solved more easily than nonlinear ones. Rather than
  solving a nonlinear system of equations by some iterative method, one
  need only solve a linear system (provided the eigenvalues and
  eigenvectors of the Jacobian matrix are known analytically, as they
  often are for practical problems). In many cases the solution of this
  linear system can also be worked out analytically and is easy to
  implement, so numerical linear algebra is not required.
\item
  In spite of smoothness over much of the domain, in interesting
  problems there are often isolated discontinuities such as shock waves
  that are important to model accurately. So some Riemann problems
  arising in a finite volume method may have large jumps between the
  left and right states. Hence a robust approximate Riemann solver must
  also handle these cases without introducing too much error.
\item
  But even in the case of large jumps in the data, it may not be
  necessary or worthwhile to solve the Riemann problem exactly. The
  information produced by the Riemann solver goes into a numerical
  method that updates the approximate solution in each grid cell and the
  exact structure of the Riemann solution is lost in the process.
\end{itemize}

    Each chapter in this part of the book illustrates some common
approximate Riemann solvers in the context of one of the nonlinear
systems studied in part 1. We focus on two popular approaches to
devising approximate Riemann solvers, though these are certainly not the
only approaches: linearized solvers and two-wave solvers.

    \hypertarget{finite-volume-methods}{%
\subsection{Finite volume methods}\label{finite-volume-methods}}

We give a short review of Riemann-based finite volume methods to
illustrate what is typically needed from a Riemann solver in order to
implement such methods. In one space dimension, a finite volume
approximation to the solution \(q(x,t_n)\) at the \(n\)th time step
consists of discrete values \(Q_j^n\), each of which can be viewed as
approximating the cell average of the solution over a grid cell
\(x_{j-1/2} < x < x_{j+1/2}\) for some discrete grid. The cell length
\(\Delta x_j = x_{j+1/2} - x_{j-1/2}\) is often uniform, but this is not
required. Many methods for hyperbolic conservation laws are written in
\emph{conservation form}, in which the numerical solution is advanced
from time \(t_n\) to \(t_{n+1} = t_n + \Delta t_n\) by the explicit
formula

\begin{align}\label{FVupdate}
Q_j^{n+1} = Q_j^n - \frac{\Delta t_n}{\Delta x_j} (F_{j+1/2}^n - F_{j-1/2}^n),
\end{align}

for some definition of the \emph{numerical flux} \(F_{j-1/2}^n\),
typically based on \(Q_{j-1}^n\) and \(Q_j^n\) (and possibly other
nearby cell values at time \(t_n\)).

Dividing (\ref{FVupdate}) by \(\Delta t_n\) and rearranging, this form
can be viewed as a discretization of \(q_t + f(q)_x = 0\), provided the
numerical flux is \emph{consistent} with the true flux \(f(q)\) in a
suitable manner. In particular if the \(Q_i\) used in defining
\(F_{j-1/2}^n\) are all equal to the same value \(\bar q\), then
\(F_{j-1/2}^n\) should reduce to \(f(\bar q)\).

A big advantage of using conservation form is that the numerical method
is conservative. The sum \(\sum \Delta x_j Q_j^n\) approximates the
integral \(\int q(x,t_n)\,dx\). Multiplying (\ref{FVupdate}) by
\(\Delta x_j\) and summing shows that at time \(t_{n+1}\) this sum only
changes due to fluxes at the boundaries of the region in question (due
to cancellation of the flux differences when summing), a property shared
with the true solution. For problems with shock waves, using methods in
conservation form is particularly important since nonconservative
formulations can lead to methods that converge to discontinuous
solutions that look fine but are not correct, e.g.~the shock wave might
propagate at entirely the wrong speed.

    \hypertarget{godunovs-method}{%
\subsubsection{Godunov's method}\label{godunovs-method}}

Trying to compute numerical approximations to hyperbolic problems with
strong shock waves is challenging because of the discontinuities in the
solution --- classical interpretations of
\((F_{j+1/2}^n - F_{j-1/2}^n)/\Delta x_j \approx \partial f/ \partial x\)
break down, oscillations near discontinuities often appear, and methods
can easily go catastrophically unstable.

The landmark paper of Godunov \cite{godunov} was the first to suggest
using the solution to Riemann problems in defining the numerical flux:
\(F_{j-1/2}^n\) is obtained by evaluating \(f(Q_{j-1/2}^*)\), where
\(Q_{j-1/2}^*\) is the Riemann solution evaluated along the ray
\(x/t = 0\) after the standard Riemann problem is solved between states
\(q_\ell=Q_{j-1}^n\) and \(q_r=Q_j^n\) (with the discontinuity placed at
\(x=t=0\) as usual in the definition of the Riemann problem). If the
numerical solution is now defined as a piecewise constant function with
value \(Q_j^n\) in the \(j\)th cell at time \(t_n\), then the exact
solution takes this value along the cell interface \(x_{j-1/2}\) for
sufficiently small later times \(t > t_n\) (until waves from other cell
interfaces begin to interact).

The classic Godunov method was developed for gas dynamics and the exact
Riemann solution was used, but since only one value is used from this
solution and the rest of the structure is thrown away, it is natural to
use some \emph{approximate Riemann solver} that more cheaply estimates
\(Q_{j-1/2}^*\). The approximations discussed in the next few chapters
are often suitable.

Godunov's method turns out to be very robust -- because the shock
structure of the solution is used in defining the interface flux, the
method generally remains stable provided that the
\emph{Courant-Friedrichs-Lewy (CFL) Condition} is satisfied, which
restricts the allowable time step relative to the cell sizes and wave
speeds by requiring that no wave can pass through more than one grid
cell in a single time step. This is clearly a necessary condition for
convergence of the method, based on domain of dependence arguments, and
for Godunov's method (with the exact solver) this is generally
sufficient as well, as verified in countless simulations (though
seemingly impossible to prove in complete generality for nonlinear
systems of equations). When the exact solution is replaced by an
approximate solution, the method may not work as well, and so some care
has to be used in defining a suitable approximation.

    \hypertarget{high-resolution-methods}{%
\subsubsection{High-resolution methods}\label{high-resolution-methods}}

In spite of its robustness, Godunov's method is seldom used as just
described because the built-in \emph{numerical viscosity} that gives it
robustness also leads to very smeared out solutions, particularly around
discontinuities, unless a very fine computational grid is used. For the
advection equation, Godunov's method reduces to the standard
``first-order upwind'' method and in general it is only first order
accurate even on smooth solutions.

A wide variety of higher order Godunov-type (i.e., Riemann solver based)
methods have been developed. One approach first reconstructs better
approximations to the solution at each time from the states \(Q_j^n\),
e.g.~a piecewise polynomial that is linear or quadratic in each grid
cell rather than constant, and then uses the states from these
polynomials evaluated at the cell interfaces to define the Riemann
problems. Another approach, used in the ``wave propagation methods''
developed in \cite{fvmhp}, for example, is to take the waves that come
out of the Riemann solution based on the original data to also define
second order correction terms. In either case some \emph{limiters} must
generally be applied in order to avoid nonphysical oscillations in
solutions, particularly when the true solution has discontinuities.
There is a vast literature on such methods; see for example many of the
books cited in the \href{Preface.ipynb}{Preface}. For our present
purposes the main point is that an approximate Riemann solver is a
necessary ingredient in many methods that are commonly used to obtain
high-resolution approximations to hyperbolic PDEs.

    \hypertarget{notation-and-structure-of-approximate-solutions}{%
\subsection{Notation and structure of approximate
solutions}\label{notation-and-structure-of-approximate-solutions}}

    We consider a single Riemann problem with left state \(q_\ell\) and
right state \(q_r\). These states might be \(Q_{j-1}^n\) and \(Q_j^n\)
for a typical interace when using Godunov's method, or other states
defined from them, e.g.~after doing a polynomial reconstruction. At any
rate, from now on we will not discuss the numerical methods or the grid
in its totality, but simply focus on how to define an approximate
Riemann solution based an an arbitrary pair of states. The resulting
``interface solution'' and ``interface flux'' will be denoted simply by
\(Q^*\) and \(F^*\), respectively, as approximations to the Riemann
solution and flux along \(x/t =0\) in the similarity solution.

The Riemann solution gives a resolution of the jump
\(\Delta q = (q_r - q_\ell)\) into a set of propagating waves. In both
of the approaches described below, the approximate Riemann solution
consists entirely of traveling discontinuities, i.e., there are no
rarefaction waves in the approximate solution, although there may be a
discontinuity that approximates such a wave. One should rightly worry
about whether the approximate solution generated with such a method will
satisfy the required entropy condition and end up with rarefaction waves
where needed rather than entropy-violating shocks, and we address this
to some extent in the examples in the following chapters. It is
important to remember that we are discussing the approximate solver that
will be used \emph{at every grid interface} in every time step, and the
numerical viscosity inherent in the numerical method can lead to
rarefactions in the overall numerical approximation even if each
approximate Riemann solution lacks rarefactions. Nonetheless some care
is needed, particularly in the case of \emph{transonic} rarefactions, as
we will see.

Following \cite{fvmhp}, we refer to these traveling discontinuities as
\emph{waves} and denote them by \({\cal W}_p \in {\mathbb R}^m\), where
the index \(p\) denotes the characteristic family and typically ranges
from \(1\) to \(m\) for a system of \(m\) equations, although in an
approximate solver the number of waves may be smaller (or possibly
larger). At any rate, they always have the property that
\begin{align}\label{Wsum}
q_r - q_\ell = \sum_{p} {\cal W}_p.
\end{align} For each wave, the approximate solver must also give a wave
speed \(s_p \in{\mathbb R}\). For a linear system such as acoustics, the
true solution has this form with the \(s_p\) being eigenvalues of the
coefficient matrix and each wave is a corresponding eigenvector, as
described in \href{Acoustics.ipynb}{Acoustics}. One class of approximate
Riemann solvers descussed below is based on approximating a nonlinear
problem by a linearization locally at each interface.

Once a set of waves and speeds have been defined, we can define an
interface flux as follows: the waves for which \(s_p < 0\) are traveling
to the left while those with \(s_p>0\) are traveling to the right, and
so as an interface flux we could use \(f(Q^*)\), where \(Q^*\) is
defined by either \[
Q^* = q_\ell + \sum_{p: s_p < 0} {\cal W}_p.
\] or \[
Q^* = q_r - \sum_{p: s_p > 0} {\cal W}_p.
\] These two expressions give the same value for \(Q^*\) unless there is
a wave with \(s_p=0\), in which case they could be different if the
corresponding \({\cal W}_p\) is nonzero. However, if we are using the
\emph{exact} Riemann solution (e.g.~for a linear system or a nonlinear
problem in which the solution consists only of shock waves), then a
stationary discontinuity with \(s_p=0\) must have no jump in flux across
it (by the Rankine-Hugoniot condition) and so even if the two values of
\(Q^*\) differ, the flux \(f(Q^*)\) is uniquely defined. For an
approximate Riemann solution this might not be true.

Another way to define the interface flux \(F^*\) would be as
\begin{align}\label{Fstar}
F^* = f(q_\ell) + \sum_{p: s_p \leq 0} s_p{\cal W}_p
= f(q_r) - \sum_{p: s_p \geq 0} s_p{\cal W}_p
\end{align} suggested by the fact that this is the correct flux along
\(x/t = 0\) for a linear system \(f(q)=Aq\) or for a nonlinear system
with only shocks in the solution; in these cases the Rankine-Hugoniot
condition implies that \(s_p{\cal W}_p\) is equal to the jump in flux
across each wave. Note that in this expression the terms in the sum for
\(s_p=0\) drop out so the two expressions always agree.

    The wave propagation algorithms described in \cite{fvmhp} and
implemented in Clawpack use a form of Godunov's method based on the sums
appearing in (\ref{Fstar}), called ``fluctuations'', to update the
neighboring cell averages, rather than the flux difference form. In
these methods the waves and speeds which are further used (after
applying a limiter to the waves) to obtain the high-resolution
corrections. An advantage of working with fluctuations, waves, and
speeds rather than interface fluxes is that these quantities often make
sense also for \emph{non-conservative hyperbolic systems,} such as the
variable coefficient linear problem \(q_t + A(x)q_x = 0\), for which
there is no ``flux function''. A Riemann problem is defined by
prescribing matrices \(A_\ell\) and \(A_r\) along with the initial data
\(q_\ell\) and \(q_r\), for example by using the material properties in
the grid cells to the left and right of the interface for acoustics
through a heterogeneous material. The waves are then naturally defined
using the eigenvectors of \(A_\ell\) corresponding to negative
eigenvalues for the left-going waves, and using eigenvectors of \(A_r\)
corresponding to positive eigenvalues for the right-going waves. See
\cite{fvmhp} for more details.

Updating cell averages by fluctuations rather than flux differencing
will give a conservative method (when applied to a hyperbolic problem in
conservation form) only if the waves and speeds in the approximate
solver satisfy \begin{align}
\label{adqdf}
\sum_{p=1}^m s_p {\cal W}_p = f(q_r) - f(q_\ell).
\end{align} This is a natural condition to require of our approximate
Riemann solvers in general, even though the flux-differencing form
(\ref{FVupdate}) always leads to a conservative method, since this is
satisfied by the exact solution in cases where it consists only of
discontinuities and each wave satisfies the Rankine-Hugoniot condition.
When (\ref{adqdf}) is satisfied we say the approximate solver is
conservative.

    \hypertarget{linearized-riemann-solvers}{%
\subsection{Linearized Riemann
solvers}\label{linearized-riemann-solvers}}

Consider a nonlinear system \(q_t + f(q)_x = 0\). If \(q_\ell\) and
\(q_r\) are close to each other, as is often the case over smooth
regions of a more general solution, then the nonlinear system can be
approximated by a linear problem of the form \(q_t + \hat A q_x = 0\).
The coefficient matrix \(\hat A\) should be some approximation to
\(f'(q_\ell) \approx f'(q_r)\) in the case where \(\|q_\ell-q_r\|\) is
small. The idea of a general linearized Riemann solver is to define a
matrix \(\hat A(q_\ell, q_r)\) that has this property but also makes
sense as an approximation in the case when \(\|q_\ell-q_r\|\) is not
small. For many nonlinear systems there is a \emph{Roe linearization}, a
particular function that works works very well based on ideas introduced
originally by Roe \cite{Roe1981}. For systems such as the shallow water
equations or the Euler equations, there are closed-form expressions for
the eigenvalues and eigenvectors of \(\hat A\) and the solution of the
linearized Riemann problem, leading to efficient solvers. These will be
presented in the next few chapters.

    \hypertarget{two-wave-solvers}{%
\subsection{Two-wave solvers}\label{two-wave-solvers}}

Since the Riemann solution impacts the overall numerical solution only
based on how it modifies the two neighboring solution values, it seems
reasonable to consider approximations in which only a single wave
propagates in each direction. The solution will have a single
intermediate state \(q_m\) such that \({\cal W}_1 = q_m - q_\ell\) and
\({\cal W}_2 = q_r-q_m\). There are apparently \(m+2\) values to be
determined: the middle state \(q_m \in {\mathbb R}^m\) and the speeds
\(s_1, s_2\). In order for the approximate solver to be conservative, it
must satisfy (\ref{adqdf}), and hence the \(m\) conditions\\
\begin{align}
f(q_r) - f(q_\ell) = s_1 {\cal W}_1 + s_2 {\cal W}_2.
\end{align}\\
This can be solved for the middle state to find\\
\begin{align}  \label{AS:middle_state}
q_m = \frac{f(q_r) - f(q_\ell) - s_2 q_r + s_1 q_\ell}{s_1 - s_2}.
\end{align}\\
It remains only to specify the wave speeds, and it is in this
specification that the various two-wave solvers differ. In the following
sections we briefly discuss the choice of wave speed for a scalar
problem; the choice for systems will be elaborated in subsequent
chapters.

Typically \(s_1 < 0 < s_2\) and so the intermediate state we need is
\(Q^* = q_m\) and \(F^* = f(Q^*)\). However, in some cases both \(s_1\)
and \(s_2\) could have the same sign, in which case \(F^*\) is either
\(f(q_\ell)\) or \(f(q_r)\).

In addition to the references provided below, this class of solvers is
also an ingredient in the so-called \emph{central schemes}. Due to the
extreme simplicity of two-wave solvers, the resulting central schemes
are often even referred to as being ``Riemann-solver-free''.

    \hypertarget{lax-friedrichs-lf-and-local-lax-friedrichs-llf}{%
\subsubsection{Lax-Friedrichs (LF) and local-Lax-Friedrichs
(LLF)}\label{lax-friedrichs-lf-and-local-lax-friedrichs-llf}}

    The simplest such solver is the \emph{Lax-Friedrichs method}, in which
it is assumed that both waves have the same speed, in opposite
directions: \[-s_1 = s_2 = a,\] where \(a\ge 0\). Then
(\ref{AS:middle_state}) becomes
\[q_m = -\frac{f(q_r) - f(q_\ell)}{2a} + \frac{q_r + q_\ell}{2}.\] In
the original Lax-Friedrichs method, the wave speed \(a\) is taken to be
the same in every Riemann problem over the entire grid; in the
\emph{local Lax Friedrichs (LLF) method}, a different speed \(a\) may be
chosen for each Riemann problem.

For stability reasons, the wave speed should be chosen at least as large
as the fastest wave speed appearing in the true Riemann solution.
However, choosing a wave speed that is too large leads to excess
diffusion. For the LLF method (originally due to Rusanov), the wave
speed is chosen as \[a(q_r, q_\ell) = \max(|f'(q)|)\]\\
where the maximum is taken over all values of \(q\) between \(q_r\) and
\(q_\ell\). This ensures stability, but may still introduce substantial
damping of slower waves.

    \hypertarget{harten-lax-van-leer-hll}{%
\subsubsection{Harten-Lax-van Leer
(HLL)}\label{harten-lax-van-leer-hll}}

    A less dissipative solver can be obtained by allowing the left- and
right-going waves to have different speeds. This approach was developed
in \cite{HLL}. The solution is then determined by
(\ref{AS:middle_state}). In the original HLL solver, it was suggested to
again to use speeds that bound the possible speeds occurring in the true
solution. For a scalar problem, this translates to\\
\begin{align*}
s_1 & = \min(f'(q)) \\
s_2 & = \max(f'(q)),
\end{align*}\\
where again the minimum and maximum are taken over all values between
\(q_r\) and \(q_\ell\). Many refinements of this choice have been
proposed in the context of systems of equations, some of which will be
discussed in later chapters.

    \begin{tcolorbox}[breakable, size=fbox, boxrule=1pt, pad at break*=1mm,colback=cellbackground, colframe=cellborder]
\prompt{In}{incolor}{ }{\boxspacing}
\begin{Verbatim}[commandchars=\\\{\}]

\end{Verbatim}
\end{tcolorbox}


    % Add a bibliography block to the postdoc
    
    
    
\end{document}
