\documentclass[11pt]{article}

    \usepackage[breakable]{tcolorbox}
    \usepackage{parskip} % Stop auto-indenting (to mimic markdown behaviour)
    
    \usepackage{iftex}
    \ifPDFTeX
    	\usepackage[T1]{fontenc}
    	\usepackage{mathpazo}
    \else
    	\usepackage{fontspec}
    \fi

    % Basic figure setup, for now with no caption control since it's done
    % automatically by Pandoc (which extracts ![](path) syntax from Markdown).
    \usepackage{graphicx}
    % Maintain compatibility with old templates. Remove in nbconvert 6.0
    \let\Oldincludegraphics\includegraphics
    % Ensure that by default, figures have no caption (until we provide a
    % proper Figure object with a Caption API and a way to capture that
    % in the conversion process - todo).
    \usepackage{caption}
    \DeclareCaptionFormat{nocaption}{}
    \captionsetup{format=nocaption,aboveskip=0pt,belowskip=0pt}

    \usepackage{float}
    \floatplacement{figure}{H} % forces figures to be placed at the correct location
    \usepackage{xcolor} % Allow colors to be defined
    \usepackage{enumerate} % Needed for markdown enumerations to work
    \usepackage{geometry} % Used to adjust the document margins
    \usepackage{amsmath} % Equations
    \usepackage{amssymb} % Equations
    \usepackage{textcomp} % defines textquotesingle
    % Hack from http://tex.stackexchange.com/a/47451/13684:
    \AtBeginDocument{%
        \def\PYZsq{\textquotesingle}% Upright quotes in Pygmentized code
    }
    \usepackage{upquote} % Upright quotes for verbatim code
    \usepackage{eurosym} % defines \euro
    \usepackage[mathletters]{ucs} % Extended unicode (utf-8) support
    \usepackage{fancyvrb} % verbatim replacement that allows latex
    \usepackage{grffile} % extends the file name processing of package graphics 
                         % to support a larger range
    \makeatletter % fix for old versions of grffile with XeLaTeX
    \@ifpackagelater{grffile}{2019/11/01}
    {
      % Do nothing on new versions
    }
    {
      \def\Gread@@xetex#1{%
        \IfFileExists{"\Gin@base".bb}%
        {\Gread@eps{\Gin@base.bb}}%
        {\Gread@@xetex@aux#1}%
      }
    }
    \makeatother
    \usepackage[Export]{adjustbox} % Used to constrain images to a maximum size
    \adjustboxset{max size={0.9\linewidth}{0.9\paperheight}}

    % The hyperref package gives us a pdf with properly built
    % internal navigation ('pdf bookmarks' for the table of contents,
    % internal cross-reference links, web links for URLs, etc.)
    \usepackage{hyperref}
    % The default LaTeX title has an obnoxious amount of whitespace. By default,
    % titling removes some of it. It also provides customization options.
    \usepackage{titling}
    \usepackage{longtable} % longtable support required by pandoc >1.10
    \usepackage{booktabs}  % table support for pandoc > 1.12.2
    \usepackage[inline]{enumitem} % IRkernel/repr support (it uses the enumerate* environment)
    \usepackage[normalem]{ulem} % ulem is needed to support strikethroughs (\sout)
                                % normalem makes italics be italics, not underlines
    \usepackage{mathrsfs}
    

    
    % Colors for the hyperref package
    \definecolor{urlcolor}{rgb}{0,.145,.698}
    \definecolor{linkcolor}{rgb}{.71,0.21,0.01}
    \definecolor{citecolor}{rgb}{.12,.54,.11}

    % ANSI colors
    \definecolor{ansi-black}{HTML}{3E424D}
    \definecolor{ansi-black-intense}{HTML}{282C36}
    \definecolor{ansi-red}{HTML}{E75C58}
    \definecolor{ansi-red-intense}{HTML}{B22B31}
    \definecolor{ansi-green}{HTML}{00A250}
    \definecolor{ansi-green-intense}{HTML}{007427}
    \definecolor{ansi-yellow}{HTML}{DDB62B}
    \definecolor{ansi-yellow-intense}{HTML}{B27D12}
    \definecolor{ansi-blue}{HTML}{208FFB}
    \definecolor{ansi-blue-intense}{HTML}{0065CA}
    \definecolor{ansi-magenta}{HTML}{D160C4}
    \definecolor{ansi-magenta-intense}{HTML}{A03196}
    \definecolor{ansi-cyan}{HTML}{60C6C8}
    \definecolor{ansi-cyan-intense}{HTML}{258F8F}
    \definecolor{ansi-white}{HTML}{C5C1B4}
    \definecolor{ansi-white-intense}{HTML}{A1A6B2}
    \definecolor{ansi-default-inverse-fg}{HTML}{FFFFFF}
    \definecolor{ansi-default-inverse-bg}{HTML}{000000}

    % common color for the border for error outputs.
    \definecolor{outerrorbackground}{HTML}{FFDFDF}

    % commands and environments needed by pandoc snippets
    % extracted from the output of `pandoc -s`
    \providecommand{\tightlist}{%
      \setlength{\itemsep}{0pt}\setlength{\parskip}{0pt}}
    \DefineVerbatimEnvironment{Highlighting}{Verbatim}{commandchars=\\\{\}}
    % Add ',fontsize=\small' for more characters per line
    \newenvironment{Shaded}{}{}
    \newcommand{\KeywordTok}[1]{\textcolor[rgb]{0.00,0.44,0.13}{\textbf{{#1}}}}
    \newcommand{\DataTypeTok}[1]{\textcolor[rgb]{0.56,0.13,0.00}{{#1}}}
    \newcommand{\DecValTok}[1]{\textcolor[rgb]{0.25,0.63,0.44}{{#1}}}
    \newcommand{\BaseNTok}[1]{\textcolor[rgb]{0.25,0.63,0.44}{{#1}}}
    \newcommand{\FloatTok}[1]{\textcolor[rgb]{0.25,0.63,0.44}{{#1}}}
    \newcommand{\CharTok}[1]{\textcolor[rgb]{0.25,0.44,0.63}{{#1}}}
    \newcommand{\StringTok}[1]{\textcolor[rgb]{0.25,0.44,0.63}{{#1}}}
    \newcommand{\CommentTok}[1]{\textcolor[rgb]{0.38,0.63,0.69}{\textit{{#1}}}}
    \newcommand{\OtherTok}[1]{\textcolor[rgb]{0.00,0.44,0.13}{{#1}}}
    \newcommand{\AlertTok}[1]{\textcolor[rgb]{1.00,0.00,0.00}{\textbf{{#1}}}}
    \newcommand{\FunctionTok}[1]{\textcolor[rgb]{0.02,0.16,0.49}{{#1}}}
    \newcommand{\RegionMarkerTok}[1]{{#1}}
    \newcommand{\ErrorTok}[1]{\textcolor[rgb]{1.00,0.00,0.00}{\textbf{{#1}}}}
    \newcommand{\NormalTok}[1]{{#1}}
    
    % Additional commands for more recent versions of Pandoc
    \newcommand{\ConstantTok}[1]{\textcolor[rgb]{0.53,0.00,0.00}{{#1}}}
    \newcommand{\SpecialCharTok}[1]{\textcolor[rgb]{0.25,0.44,0.63}{{#1}}}
    \newcommand{\VerbatimStringTok}[1]{\textcolor[rgb]{0.25,0.44,0.63}{{#1}}}
    \newcommand{\SpecialStringTok}[1]{\textcolor[rgb]{0.73,0.40,0.53}{{#1}}}
    \newcommand{\ImportTok}[1]{{#1}}
    \newcommand{\DocumentationTok}[1]{\textcolor[rgb]{0.73,0.13,0.13}{\textit{{#1}}}}
    \newcommand{\AnnotationTok}[1]{\textcolor[rgb]{0.38,0.63,0.69}{\textbf{\textit{{#1}}}}}
    \newcommand{\CommentVarTok}[1]{\textcolor[rgb]{0.38,0.63,0.69}{\textbf{\textit{{#1}}}}}
    \newcommand{\VariableTok}[1]{\textcolor[rgb]{0.10,0.09,0.49}{{#1}}}
    \newcommand{\ControlFlowTok}[1]{\textcolor[rgb]{0.00,0.44,0.13}{\textbf{{#1}}}}
    \newcommand{\OperatorTok}[1]{\textcolor[rgb]{0.40,0.40,0.40}{{#1}}}
    \newcommand{\BuiltInTok}[1]{{#1}}
    \newcommand{\ExtensionTok}[1]{{#1}}
    \newcommand{\PreprocessorTok}[1]{\textcolor[rgb]{0.74,0.48,0.00}{{#1}}}
    \newcommand{\AttributeTok}[1]{\textcolor[rgb]{0.49,0.56,0.16}{{#1}}}
    \newcommand{\InformationTok}[1]{\textcolor[rgb]{0.38,0.63,0.69}{\textbf{\textit{{#1}}}}}
    \newcommand{\WarningTok}[1]{\textcolor[rgb]{0.38,0.63,0.69}{\textbf{\textit{{#1}}}}}
    
    
    % Define a nice break command that doesn't care if a line doesn't already
    % exist.
    \def\br{\hspace*{\fill} \\* }
    % Math Jax compatibility definitions
    \def\gt{>}
    \def\lt{<}
    \let\Oldtex\TeX
    \let\Oldlatex\LaTeX
    \renewcommand{\TeX}{\textrm{\Oldtex}}
    \renewcommand{\LaTeX}{\textrm{\Oldlatex}}
    % Document parameters
    % Document title
    \title{Euler}
    
    
    
    
    
% Pygments definitions
\makeatletter
\def\PY@reset{\let\PY@it=\relax \let\PY@bf=\relax%
    \let\PY@ul=\relax \let\PY@tc=\relax%
    \let\PY@bc=\relax \let\PY@ff=\relax}
\def\PY@tok#1{\csname PY@tok@#1\endcsname}
\def\PY@toks#1+{\ifx\relax#1\empty\else%
    \PY@tok{#1}\expandafter\PY@toks\fi}
\def\PY@do#1{\PY@bc{\PY@tc{\PY@ul{%
    \PY@it{\PY@bf{\PY@ff{#1}}}}}}}
\def\PY#1#2{\PY@reset\PY@toks#1+\relax+\PY@do{#2}}

\expandafter\def\csname PY@tok@w\endcsname{\def\PY@tc##1{\textcolor[rgb]{0.73,0.73,0.73}{##1}}}
\expandafter\def\csname PY@tok@c\endcsname{\let\PY@it=\textit\def\PY@tc##1{\textcolor[rgb]{0.25,0.50,0.50}{##1}}}
\expandafter\def\csname PY@tok@cp\endcsname{\def\PY@tc##1{\textcolor[rgb]{0.74,0.48,0.00}{##1}}}
\expandafter\def\csname PY@tok@k\endcsname{\let\PY@bf=\textbf\def\PY@tc##1{\textcolor[rgb]{0.00,0.50,0.00}{##1}}}
\expandafter\def\csname PY@tok@kp\endcsname{\def\PY@tc##1{\textcolor[rgb]{0.00,0.50,0.00}{##1}}}
\expandafter\def\csname PY@tok@kt\endcsname{\def\PY@tc##1{\textcolor[rgb]{0.69,0.00,0.25}{##1}}}
\expandafter\def\csname PY@tok@o\endcsname{\def\PY@tc##1{\textcolor[rgb]{0.40,0.40,0.40}{##1}}}
\expandafter\def\csname PY@tok@ow\endcsname{\let\PY@bf=\textbf\def\PY@tc##1{\textcolor[rgb]{0.67,0.13,1.00}{##1}}}
\expandafter\def\csname PY@tok@nb\endcsname{\def\PY@tc##1{\textcolor[rgb]{0.00,0.50,0.00}{##1}}}
\expandafter\def\csname PY@tok@nf\endcsname{\def\PY@tc##1{\textcolor[rgb]{0.00,0.00,1.00}{##1}}}
\expandafter\def\csname PY@tok@nc\endcsname{\let\PY@bf=\textbf\def\PY@tc##1{\textcolor[rgb]{0.00,0.00,1.00}{##1}}}
\expandafter\def\csname PY@tok@nn\endcsname{\let\PY@bf=\textbf\def\PY@tc##1{\textcolor[rgb]{0.00,0.00,1.00}{##1}}}
\expandafter\def\csname PY@tok@ne\endcsname{\let\PY@bf=\textbf\def\PY@tc##1{\textcolor[rgb]{0.82,0.25,0.23}{##1}}}
\expandafter\def\csname PY@tok@nv\endcsname{\def\PY@tc##1{\textcolor[rgb]{0.10,0.09,0.49}{##1}}}
\expandafter\def\csname PY@tok@no\endcsname{\def\PY@tc##1{\textcolor[rgb]{0.53,0.00,0.00}{##1}}}
\expandafter\def\csname PY@tok@nl\endcsname{\def\PY@tc##1{\textcolor[rgb]{0.63,0.63,0.00}{##1}}}
\expandafter\def\csname PY@tok@ni\endcsname{\let\PY@bf=\textbf\def\PY@tc##1{\textcolor[rgb]{0.60,0.60,0.60}{##1}}}
\expandafter\def\csname PY@tok@na\endcsname{\def\PY@tc##1{\textcolor[rgb]{0.49,0.56,0.16}{##1}}}
\expandafter\def\csname PY@tok@nt\endcsname{\let\PY@bf=\textbf\def\PY@tc##1{\textcolor[rgb]{0.00,0.50,0.00}{##1}}}
\expandafter\def\csname PY@tok@nd\endcsname{\def\PY@tc##1{\textcolor[rgb]{0.67,0.13,1.00}{##1}}}
\expandafter\def\csname PY@tok@s\endcsname{\def\PY@tc##1{\textcolor[rgb]{0.73,0.13,0.13}{##1}}}
\expandafter\def\csname PY@tok@sd\endcsname{\let\PY@it=\textit\def\PY@tc##1{\textcolor[rgb]{0.73,0.13,0.13}{##1}}}
\expandafter\def\csname PY@tok@si\endcsname{\let\PY@bf=\textbf\def\PY@tc##1{\textcolor[rgb]{0.73,0.40,0.53}{##1}}}
\expandafter\def\csname PY@tok@se\endcsname{\let\PY@bf=\textbf\def\PY@tc##1{\textcolor[rgb]{0.73,0.40,0.13}{##1}}}
\expandafter\def\csname PY@tok@sr\endcsname{\def\PY@tc##1{\textcolor[rgb]{0.73,0.40,0.53}{##1}}}
\expandafter\def\csname PY@tok@ss\endcsname{\def\PY@tc##1{\textcolor[rgb]{0.10,0.09,0.49}{##1}}}
\expandafter\def\csname PY@tok@sx\endcsname{\def\PY@tc##1{\textcolor[rgb]{0.00,0.50,0.00}{##1}}}
\expandafter\def\csname PY@tok@m\endcsname{\def\PY@tc##1{\textcolor[rgb]{0.40,0.40,0.40}{##1}}}
\expandafter\def\csname PY@tok@gh\endcsname{\let\PY@bf=\textbf\def\PY@tc##1{\textcolor[rgb]{0.00,0.00,0.50}{##1}}}
\expandafter\def\csname PY@tok@gu\endcsname{\let\PY@bf=\textbf\def\PY@tc##1{\textcolor[rgb]{0.50,0.00,0.50}{##1}}}
\expandafter\def\csname PY@tok@gd\endcsname{\def\PY@tc##1{\textcolor[rgb]{0.63,0.00,0.00}{##1}}}
\expandafter\def\csname PY@tok@gi\endcsname{\def\PY@tc##1{\textcolor[rgb]{0.00,0.63,0.00}{##1}}}
\expandafter\def\csname PY@tok@gr\endcsname{\def\PY@tc##1{\textcolor[rgb]{1.00,0.00,0.00}{##1}}}
\expandafter\def\csname PY@tok@ge\endcsname{\let\PY@it=\textit}
\expandafter\def\csname PY@tok@gs\endcsname{\let\PY@bf=\textbf}
\expandafter\def\csname PY@tok@gp\endcsname{\let\PY@bf=\textbf\def\PY@tc##1{\textcolor[rgb]{0.00,0.00,0.50}{##1}}}
\expandafter\def\csname PY@tok@go\endcsname{\def\PY@tc##1{\textcolor[rgb]{0.53,0.53,0.53}{##1}}}
\expandafter\def\csname PY@tok@gt\endcsname{\def\PY@tc##1{\textcolor[rgb]{0.00,0.27,0.87}{##1}}}
\expandafter\def\csname PY@tok@err\endcsname{\def\PY@bc##1{\setlength{\fboxsep}{0pt}\fcolorbox[rgb]{1.00,0.00,0.00}{1,1,1}{\strut ##1}}}
\expandafter\def\csname PY@tok@kc\endcsname{\let\PY@bf=\textbf\def\PY@tc##1{\textcolor[rgb]{0.00,0.50,0.00}{##1}}}
\expandafter\def\csname PY@tok@kd\endcsname{\let\PY@bf=\textbf\def\PY@tc##1{\textcolor[rgb]{0.00,0.50,0.00}{##1}}}
\expandafter\def\csname PY@tok@kn\endcsname{\let\PY@bf=\textbf\def\PY@tc##1{\textcolor[rgb]{0.00,0.50,0.00}{##1}}}
\expandafter\def\csname PY@tok@kr\endcsname{\let\PY@bf=\textbf\def\PY@tc##1{\textcolor[rgb]{0.00,0.50,0.00}{##1}}}
\expandafter\def\csname PY@tok@bp\endcsname{\def\PY@tc##1{\textcolor[rgb]{0.00,0.50,0.00}{##1}}}
\expandafter\def\csname PY@tok@fm\endcsname{\def\PY@tc##1{\textcolor[rgb]{0.00,0.00,1.00}{##1}}}
\expandafter\def\csname PY@tok@vc\endcsname{\def\PY@tc##1{\textcolor[rgb]{0.10,0.09,0.49}{##1}}}
\expandafter\def\csname PY@tok@vg\endcsname{\def\PY@tc##1{\textcolor[rgb]{0.10,0.09,0.49}{##1}}}
\expandafter\def\csname PY@tok@vi\endcsname{\def\PY@tc##1{\textcolor[rgb]{0.10,0.09,0.49}{##1}}}
\expandafter\def\csname PY@tok@vm\endcsname{\def\PY@tc##1{\textcolor[rgb]{0.10,0.09,0.49}{##1}}}
\expandafter\def\csname PY@tok@sa\endcsname{\def\PY@tc##1{\textcolor[rgb]{0.73,0.13,0.13}{##1}}}
\expandafter\def\csname PY@tok@sb\endcsname{\def\PY@tc##1{\textcolor[rgb]{0.73,0.13,0.13}{##1}}}
\expandafter\def\csname PY@tok@sc\endcsname{\def\PY@tc##1{\textcolor[rgb]{0.73,0.13,0.13}{##1}}}
\expandafter\def\csname PY@tok@dl\endcsname{\def\PY@tc##1{\textcolor[rgb]{0.73,0.13,0.13}{##1}}}
\expandafter\def\csname PY@tok@s2\endcsname{\def\PY@tc##1{\textcolor[rgb]{0.73,0.13,0.13}{##1}}}
\expandafter\def\csname PY@tok@sh\endcsname{\def\PY@tc##1{\textcolor[rgb]{0.73,0.13,0.13}{##1}}}
\expandafter\def\csname PY@tok@s1\endcsname{\def\PY@tc##1{\textcolor[rgb]{0.73,0.13,0.13}{##1}}}
\expandafter\def\csname PY@tok@mb\endcsname{\def\PY@tc##1{\textcolor[rgb]{0.40,0.40,0.40}{##1}}}
\expandafter\def\csname PY@tok@mf\endcsname{\def\PY@tc##1{\textcolor[rgb]{0.40,0.40,0.40}{##1}}}
\expandafter\def\csname PY@tok@mh\endcsname{\def\PY@tc##1{\textcolor[rgb]{0.40,0.40,0.40}{##1}}}
\expandafter\def\csname PY@tok@mi\endcsname{\def\PY@tc##1{\textcolor[rgb]{0.40,0.40,0.40}{##1}}}
\expandafter\def\csname PY@tok@il\endcsname{\def\PY@tc##1{\textcolor[rgb]{0.40,0.40,0.40}{##1}}}
\expandafter\def\csname PY@tok@mo\endcsname{\def\PY@tc##1{\textcolor[rgb]{0.40,0.40,0.40}{##1}}}
\expandafter\def\csname PY@tok@ch\endcsname{\let\PY@it=\textit\def\PY@tc##1{\textcolor[rgb]{0.25,0.50,0.50}{##1}}}
\expandafter\def\csname PY@tok@cm\endcsname{\let\PY@it=\textit\def\PY@tc##1{\textcolor[rgb]{0.25,0.50,0.50}{##1}}}
\expandafter\def\csname PY@tok@cpf\endcsname{\let\PY@it=\textit\def\PY@tc##1{\textcolor[rgb]{0.25,0.50,0.50}{##1}}}
\expandafter\def\csname PY@tok@c1\endcsname{\let\PY@it=\textit\def\PY@tc##1{\textcolor[rgb]{0.25,0.50,0.50}{##1}}}
\expandafter\def\csname PY@tok@cs\endcsname{\let\PY@it=\textit\def\PY@tc##1{\textcolor[rgb]{0.25,0.50,0.50}{##1}}}

\def\PYZbs{\char`\\}
\def\PYZus{\char`\_}
\def\PYZob{\char`\{}
\def\PYZcb{\char`\}}
\def\PYZca{\char`\^}
\def\PYZam{\char`\&}
\def\PYZlt{\char`\<}
\def\PYZgt{\char`\>}
\def\PYZsh{\char`\#}
\def\PYZpc{\char`\%}
\def\PYZdl{\char`\$}
\def\PYZhy{\char`\-}
\def\PYZsq{\char`\'}
\def\PYZdq{\char`\"}
\def\PYZti{\char`\~}
% for compatibility with earlier versions
\def\PYZat{@}
\def\PYZlb{[}
\def\PYZrb{]}
\makeatother


    % For linebreaks inside Verbatim environment from package fancyvrb. 
    \makeatletter
        \newbox\Wrappedcontinuationbox 
        \newbox\Wrappedvisiblespacebox 
        \newcommand*\Wrappedvisiblespace {\textcolor{red}{\textvisiblespace}} 
        \newcommand*\Wrappedcontinuationsymbol {\textcolor{red}{\llap{\tiny$\m@th\hookrightarrow$}}} 
        \newcommand*\Wrappedcontinuationindent {3ex } 
        \newcommand*\Wrappedafterbreak {\kern\Wrappedcontinuationindent\copy\Wrappedcontinuationbox} 
        % Take advantage of the already applied Pygments mark-up to insert 
        % potential linebreaks for TeX processing. 
        %        {, <, #, %, $, ' and ": go to next line. 
        %        _, }, ^, &, >, - and ~: stay at end of broken line. 
        % Use of \textquotesingle for straight quote. 
        \newcommand*\Wrappedbreaksatspecials {% 
            \def\PYGZus{\discretionary{\char`\_}{\Wrappedafterbreak}{\char`\_}}% 
            \def\PYGZob{\discretionary{}{\Wrappedafterbreak\char`\{}{\char`\{}}% 
            \def\PYGZcb{\discretionary{\char`\}}{\Wrappedafterbreak}{\char`\}}}% 
            \def\PYGZca{\discretionary{\char`\^}{\Wrappedafterbreak}{\char`\^}}% 
            \def\PYGZam{\discretionary{\char`\&}{\Wrappedafterbreak}{\char`\&}}% 
            \def\PYGZlt{\discretionary{}{\Wrappedafterbreak\char`\<}{\char`\<}}% 
            \def\PYGZgt{\discretionary{\char`\>}{\Wrappedafterbreak}{\char`\>}}% 
            \def\PYGZsh{\discretionary{}{\Wrappedafterbreak\char`\#}{\char`\#}}% 
            \def\PYGZpc{\discretionary{}{\Wrappedafterbreak\char`\%}{\char`\%}}% 
            \def\PYGZdl{\discretionary{}{\Wrappedafterbreak\char`\$}{\char`\$}}% 
            \def\PYGZhy{\discretionary{\char`\-}{\Wrappedafterbreak}{\char`\-}}% 
            \def\PYGZsq{\discretionary{}{\Wrappedafterbreak\textquotesingle}{\textquotesingle}}% 
            \def\PYGZdq{\discretionary{}{\Wrappedafterbreak\char`\"}{\char`\"}}% 
            \def\PYGZti{\discretionary{\char`\~}{\Wrappedafterbreak}{\char`\~}}% 
        } 
        % Some characters . , ; ? ! / are not pygmentized. 
        % This macro makes them "active" and they will insert potential linebreaks 
        \newcommand*\Wrappedbreaksatpunct {% 
            \lccode`\~`\.\lowercase{\def~}{\discretionary{\hbox{\char`\.}}{\Wrappedafterbreak}{\hbox{\char`\.}}}% 
            \lccode`\~`\,\lowercase{\def~}{\discretionary{\hbox{\char`\,}}{\Wrappedafterbreak}{\hbox{\char`\,}}}% 
            \lccode`\~`\;\lowercase{\def~}{\discretionary{\hbox{\char`\;}}{\Wrappedafterbreak}{\hbox{\char`\;}}}% 
            \lccode`\~`\:\lowercase{\def~}{\discretionary{\hbox{\char`\:}}{\Wrappedafterbreak}{\hbox{\char`\:}}}% 
            \lccode`\~`\?\lowercase{\def~}{\discretionary{\hbox{\char`\?}}{\Wrappedafterbreak}{\hbox{\char`\?}}}% 
            \lccode`\~`\!\lowercase{\def~}{\discretionary{\hbox{\char`\!}}{\Wrappedafterbreak}{\hbox{\char`\!}}}% 
            \lccode`\~`\/\lowercase{\def~}{\discretionary{\hbox{\char`\/}}{\Wrappedafterbreak}{\hbox{\char`\/}}}% 
            \catcode`\.\active
            \catcode`\,\active 
            \catcode`\;\active
            \catcode`\:\active
            \catcode`\?\active
            \catcode`\!\active
            \catcode`\/\active 
            \lccode`\~`\~ 	
        }
    \makeatother

    \let\OriginalVerbatim=\Verbatim
    \makeatletter
    \renewcommand{\Verbatim}[1][1]{%
        %\parskip\z@skip
        \sbox\Wrappedcontinuationbox {\Wrappedcontinuationsymbol}%
        \sbox\Wrappedvisiblespacebox {\FV@SetupFont\Wrappedvisiblespace}%
        \def\FancyVerbFormatLine ##1{\hsize\linewidth
            \vtop{\raggedright\hyphenpenalty\z@\exhyphenpenalty\z@
                \doublehyphendemerits\z@\finalhyphendemerits\z@
                \strut ##1\strut}%
        }%
        % If the linebreak is at a space, the latter will be displayed as visible
        % space at end of first line, and a continuation symbol starts next line.
        % Stretch/shrink are however usually zero for typewriter font.
        \def\FV@Space {%
            \nobreak\hskip\z@ plus\fontdimen3\font minus\fontdimen4\font
            \discretionary{\copy\Wrappedvisiblespacebox}{\Wrappedafterbreak}
            {\kern\fontdimen2\font}%
        }%
        
        % Allow breaks at special characters using \PYG... macros.
        \Wrappedbreaksatspecials
        % Breaks at punctuation characters . , ; ? ! and / need catcode=\active 	
        \OriginalVerbatim[#1,codes*=\Wrappedbreaksatpunct]%
    }
    \makeatother

    % Exact colors from NB
    \definecolor{incolor}{HTML}{303F9F}
    \definecolor{outcolor}{HTML}{D84315}
    \definecolor{cellborder}{HTML}{CFCFCF}
    \definecolor{cellbackground}{HTML}{F7F7F7}
    
    % prompt
    \makeatletter
    \newcommand{\boxspacing}{\kern\kvtcb@left@rule\kern\kvtcb@boxsep}
    \makeatother
    \newcommand{\prompt}[4]{
        {\ttfamily\llap{{\color{#2}[#3]:\hspace{3pt}#4}}\vspace{-\baselineskip}}
    }
    

    
    % Prevent overflowing lines due to hard-to-break entities
    \sloppy 
    % Setup hyperref package
    \hypersetup{
      breaklinks=true,  % so long urls are correctly broken across lines
      colorlinks=true,
      urlcolor=urlcolor,
      linkcolor=linkcolor,
      citecolor=citecolor,
      }
    % Slightly bigger margins than the latex defaults
    
    \geometry{verbose,tmargin=1in,bmargin=1in,lmargin=1in,rmargin=1in}
    
    

\begin{document}
    
    \maketitle
    
    

    
    \hypertarget{the-euler-equations-of-gas-dynamics}{%
\section{The Euler equations of gas
dynamics}\label{the-euler-equations-of-gas-dynamics}}

    In this notebook, we discuss the equations and the structure of the
exact solution to the Riemann problem. In
\href{Euler_approximate.ipynb}{Euler\_approximate} and
\href{FV_compare.ipynb}{FV\_compare}, we will investigate approximate
Riemann solvers.

    \hypertarget{fluid-dynamics}{%
\subsection{Fluid dynamics}\label{fluid-dynamics}}

    In this chapter we study the system of hyperbolic PDEs that governs the
motions of a compressible gas in the absence of viscosity. These consist
of conservation laws for \emph{mass, momentum}, and \emph{energy}.
Together, they are referred to as the \emph{compressible Euler
equations}, or simply the Euler equations. Our discussion here is fairly
brief; for much more detail see \cite{fvmhp} or \cite{toro2013riemann}.

    \hypertarget{mass-conservation}{%
\subsubsection{Mass conservation}\label{mass-conservation}}

    We will use \(\rho(x,t)\) to denote the fluid density and \(u(x,t)\) for
its velocity. Then the equation for conservation of mass is just the
familiar \emph{continuity equation}:

\[\rho_t + (\rho u)_x = 0.\]

    \hypertarget{momentum-conservation}{%
\subsubsection{Momentum conservation}\label{momentum-conservation}}

    We discussed the conservation of momentum in a fluid already in
\href{Acoustics.ipynb}{Acoustics}. For convenience, we review the ideas
here. The momentum density is given by the product of mass density and
velocity, \(\rho u\). The momentum flux has two components. First, the
momentum is transported in the same way that the density is; this flux
is given by the momentum density times the velocity: \(\rho u^2\).

To understand the second term in the momentum flux, we must realize that
a fluid is made up of many tiny molecules. The density and velocity we
are modeling are average values over some small region of space. The
individual molecules in that region are not all moving with exactly
velocity \(u\); that's just their average. Each molecule also has some
additional random velocity component. These random velocities are what
accounts for the \emph{pressure} of the fluid, which we'll denote by
\(p\). These velocity components also lead to a net flux of momentum.
Thus the momentum conservation equation is

\[(\rho u)_t + (\rho u^2 + p)_x = 0.\]

This is very similar to the conservation of momentum equation in the
shallow water equations, as discussed in
\href{Shallow_water.ipynb}{Shallow\_water}, in which case \(hu\) is the
momentum density and \(\frac 1 2 gh^2\) is the hydrostatic pressure. For
gas dynamics, a different expression must be used to compute the
pressure \(p\) from the conserved quantities. This relation is called
the \emph{equation of state} of the gas, as discussed further below.

    \hypertarget{energy-conservation}{%
\subsubsection{Energy conservation}\label{energy-conservation}}

    The energy has two components: internal energy density \(\rho e\) and
kinetic energy density \(\rho u^2/2\):

\[E = \rho e + \frac{1}{2}\rho u^2.\]

Like the momentum flux, the energy flux involves both bulk transport
(\(Eu\)) and transport due to pressure (\(pu\)):

\[E_t + (u(E+p)) = 0.\]

    \hypertarget{equation-of-state}{%
\subsubsection{Equation of state}\label{equation-of-state}}

    You may have noticed that we have 4 unknowns (density, momentum, energy,
and pressure) but only 3 conservation laws. We need one more relation to
close the system. That relation, known as the equation of state,
expresses how the pressure is related to the other quantities. We'll
focus on the case of a polytropic ideal gas, for which

\[p = \rho e (\gamma-1).\]

Here \(\gamma\) is the ratio of specific heats, which for air is
approximately 1.4.

    \hypertarget{hyperbolic-structure-of-the-1d-euler-equations}{%
\subsection{Hyperbolic structure of the 1D Euler
equations}\label{hyperbolic-structure-of-the-1d-euler-equations}}

    We can write the three conservation laws as a single system
\(q_t + f(q)_x = 0\) by defining\\
\begin{align}
q & = \begin{pmatrix} \rho \\ \rho u \\ E\end{pmatrix}, & 
f(q) & = \begin{pmatrix} \rho u \\ \rho u^2 + p \\ u(E+p)\end{pmatrix}.
\label{euler_conserved}
\end{align}\\
These are the one-dimensional Euler system. As usual, one can define the
\(3 \times 3\) Jacobian matrix by differentiating this flux function
with respect to the three components of \(q\).

    In our discussion of the structure of these equations, it is convenient
to work with the primitive variables \((\rho, u, p)\) rather than the
conserved variables. The quasilinear form is particularly simple in the
primitive variables:

\begin{align} \label{euler_primitive}
\begin{bmatrix} \rho \\ u \\ p \end{bmatrix}_t + 
\begin{bmatrix} u & \rho & 0 \\ 0 & u & 1/\rho \\ 0 & \gamma \rho & u \end{bmatrix} \begin{bmatrix} \rho \\ u \\ p \end{bmatrix}_x & = 0.
\end{align}

    \hypertarget{characteristic-velocities}{%
\subsubsection{Characteristic
velocities}\label{characteristic-velocities}}

The eigenvalues of the flux Jacobian \(f'(q)\) for the 1D Euler
equations are:

\begin{align}
\lambda_1 & = u-c & \lambda_2 & = u & \lambda_3 & = u+c
\end{align}

Here \(c\) is the sound speed:

\[ c = \sqrt{\frac{\gamma p}{\rho}}.\]

These are also the eigenvalues of the coefficient matrix appearing in
(\ref{euler_primitive}), and show that acoustic waves propagate at
speeds \(\pm c\) relative to the fluid velocity \(u\). There is also a
characteristic speed \(\lambda_2 =u\) corresponding to the transport of
entropy at the fluid velocity, as discussed further below.

The eigenvectors of the coefficient matrix appearing in
(\ref{euler_primitive}) are:

\begin{align}\label{euler_evecs}
r_1 & = \begin{bmatrix} -\rho/c \\ 1 \\ - \rho c \end{bmatrix} &
r_2 & = \begin{bmatrix} 1 \\ 0 \\ 0 \end{bmatrix} &
r_3 & = \begin{bmatrix}  \rho/c \\ 1 \\ \rho c \end{bmatrix}.
\end{align}

These vectors show the relation between jumps in the primitive variables
across waves in each family. The eigenvectors of the flux Jacobian
\(f'(q)\) arising from the conservative form (\ref{euler_conserved})
would be different, and would give the relation between jumps in the
conserved variables across each wave.

Notice that the second characteristic speed, \(\lambda_2\), depends only
on \(u\) and that \(u\) does not change as we move in the direction of
\(r_2\). In other words, the 2-characteristic velocity is constant on
2-integral curves. This is similar to the wave that carries changes in
the tracer that we considered in
\href{Shallow_tracer.ipynb}{Shallow\_tracer}. We say this characteristic
field is \emph{linearly degenerate}; it admits neither shocks nor
rarefactions. In a simple 2-wave, all characteristics are parallel. A
jump in this family carries a change only in the density, and is
referred to as a \emph{contact discontinuity}.

Mathematically, the \(p\)th field is linearly degenerate if
\begin{align}\label{lindegen}
\nabla \lambda_p(q) \cdot r_p(q) = 0,
\end{align} since in this case the eigenvalue \(\lambda_p(q)\) does not
vary in the direction of the eigenvector \(r_p(q)\), and hence is
constant along integral curves of this family. (Recall that \(r_p(q)\)
is the tangent vector at each point on the integral curve.)

The other two fields have characteristic velocities that \emph{do} vary
along the corresponding integral curves. Moreover they vary in a
monotonic manner as we move along an integral curve, always increasing
as we move in one direction, decreasing in the other. Mathematically,
this means that \begin{align}\label{gennonlin}
\nabla \lambda_p(q) \cdot r_p(q) \ne 0
\end{align} as \(q\) varies, so that the directional derivative of
\(\lambda_p(q)\) cannot change sign as we move along the curve. This is
analogous to the flux for a scalar problem being convex, and means that
the 1-wave and the 3-wave in any Riemann solution to the Euler equations
will be a single shock or rarefaction wave, not the sort of compound
waves we observed in \href{Nonconvex_scalar.ipynb}{Nonconvex\_scalar} in
the nonconvex scalar case. Any characteristic field satisfying
(\ref{gennonlin}) is said to be \emph{genuinely nonlinear}.

    \hypertarget{entropy}{%
\subsubsection{Entropy}\label{entropy}}

Another important quantity in gas dynamics is the \emph{specific
entropy}:

\[ s = c_v \log(p/\rho^\gamma) + C,\]

where \(c_v\) and \(C\) are constants. From the expression
(\ref{euler_evecs}) for the eigenvector \(r_2\), we see that the
pressure and velocity are constant across a 2-wave. A simple 2-wave is
also called an \emph{entropy wave} because a variation in density while
the pressure remains constant requires a variation in the entropy of the
gas as well. On the other hand a simple acoustic wave (a continuously
varying pure 1-wave or 3-wave) has constant entropy throughout the wave;
the specific entropy is a Riemann invariant for these families.

A shock wave (either a 1-wave or 3-wave) satisfies the Rankine-Hugoniot
conditions and exhibits a jump in entropy. To be physically correct, the
entropy of the gas must \emph{increase} as gas molecules pass through
the shock, leading to the \emph{entropy condition} for selecting shock
waves. We have already seen this term used in the context of scalar
nonlinear equations shallow water flow, even though the entropy
condition in those cases did not involve the physical entropy.

    \hypertarget{riemann-invariants}{%
\subsubsection{Riemann invariants}\label{riemann-invariants}}

Since the Euler equations have three components, we expect each integral
curve (a 1D set in 3D space) to be defined by two Riemann invariants.
These are:

\begin{align}
1 & : s, u+\frac{2c}{\gamma-1} \\
2 & : u, p \\
3 & : s, u-\frac{2c}{\gamma-1}.
\end{align}

    \hypertarget{integral-curves}{%
\subsubsection{Integral curves}\label{integral-curves}}

The level sets of these Riemann invariants are two-dimensional surfaces;
the intersection of two appropriate level sets defines an integral
curve.

The 2-integral curves, of course, are simply lines of constant pressure
and velocity (with varying density). Since the field is linearly
degenerate, these coincide with the Hugoniot loci. We can determine the
form of the 1- and 3-integral curves using the Riemann invariants above.
For a curve passing through \((\rho_0,u_0,p_0)\), we find

\begin{align}
    \rho(p) &= (p/p_0)^{1/\gamma} \rho_0,\\
    u(p) & = u_0 \pm \frac{2c_0}{\gamma-1}\left(1-(p/p_0)^{(\gamma-1)/(2\gamma)}\right).
\end{align} Here the plus sign is for 1-waves and the minus sign is for
3-waves.

Below we plot the projection of some integral curves on the \(p-u\)
plane.

    \begin{tcolorbox}[breakable, size=fbox, boxrule=1pt, pad at break*=1mm,colback=cellbackground, colframe=cellborder]
\prompt{In}{incolor}{1}{\boxspacing}
\begin{Verbatim}[commandchars=\\\{\}]
\PY{o}{\PYZpc{}}\PY{k}{matplotlib} inline
\end{Verbatim}
\end{tcolorbox}

    \begin{tcolorbox}[breakable, size=fbox, boxrule=1pt, pad at break*=1mm,colback=cellbackground, colframe=cellborder]
\prompt{In}{incolor}{2}{\boxspacing}
\begin{Verbatim}[commandchars=\\\{\}]
\PY{o}{\PYZpc{}}\PY{k}{config} InlineBackend.figure\PYZus{}format = \PYZsq{}svg\PYZsq{}
\PY{k+kn}{from} \PY{n+nn}{exact\PYZus{}solvers} \PY{k+kn}{import} \PY{n}{euler}
\PY{k+kn}{from} \PY{n+nn}{exact\PYZus{}solvers} \PY{k+kn}{import} \PY{n}{euler\PYZus{}demos}
\PY{k+kn}{from} \PY{n+nn}{ipywidgets} \PY{k+kn}{import} \PY{n}{widgets}
\PY{k+kn}{from} \PY{n+nn}{ipywidgets} \PY{k+kn}{import} \PY{n}{interact}
\PY{n}{State} \PY{o}{=} \PY{n}{euler}\PY{o}{.}\PY{n}{Primitive\PYZus{}State}
\PY{n}{gamma} \PY{o}{=} \PY{l+m+mf}{1.4}
\end{Verbatim}
\end{tcolorbox}

    If you wish to examine the Python code for this chapter, see:

\begin{itemize}
\tightlist
\item
  \url{exact_solvers/euler.py} \ldots{}
  \href{https://github.com/clawpack/riemann_book/blob/FA16/exact_solvers/euler.py}{on
  github,}
\item
  \url{exact_solvers/euler_demos.py} \ldots{}
  \href{https://github.com/clawpack/riemann_book/blob/FA16/exact_solvers/euler_demos.py}{on
  github.}
\end{itemize}

    \begin{tcolorbox}[breakable, size=fbox, boxrule=1pt, pad at break*=1mm,colback=cellbackground, colframe=cellborder]
\prompt{In}{incolor}{3}{\boxspacing}
\begin{Verbatim}[commandchars=\\\{\}]
\PY{n}{interact}\PY{p}{(}\PY{n}{euler}\PY{o}{.}\PY{n}{plot\PYZus{}integral\PYZus{}curves}\PY{p}{,}
         \PY{n}{gamma}\PY{o}{=}\PY{n}{widgets}\PY{o}{.}\PY{n}{FloatSlider}\PY{p}{(}\PY{n+nb}{min}\PY{o}{=}\PY{l+m+mf}{1.1}\PY{p}{,}\PY{n+nb}{max}\PY{o}{=}\PY{l+m+mi}{3}\PY{p}{,}\PY{n}{value}\PY{o}{=}\PY{l+m+mf}{1.4}\PY{p}{)}\PY{p}{,}
         \PY{n}{rho\PYZus{}0}\PY{o}{=}\PY{n}{widgets}\PY{o}{.}\PY{n}{FloatSlider}\PY{p}{(}\PY{n+nb}{min}\PY{o}{=}\PY{l+m+mf}{0.1}\PY{p}{,}\PY{n+nb}{max}\PY{o}{=}\PY{l+m+mf}{3.}\PY{p}{,}\PY{n}{value}\PY{o}{=}\PY{l+m+mf}{1.}\PY{p}{,} 
                                   \PY{n}{description}\PY{o}{=}\PY{l+s+sa}{r}\PY{l+s+s1}{\PYZsq{}}\PY{l+s+s1}{\PYZdl{}}\PY{l+s+s1}{\PYZbs{}}\PY{l+s+s1}{rho\PYZus{}0\PYZdl{}}\PY{l+s+s1}{\PYZsq{}}\PY{p}{)}\PY{p}{)}\PY{p}{;}
\end{Verbatim}
\end{tcolorbox}

    
    \begin{Verbatim}[commandchars=\\\{\}]
interactive(children=(Checkbox(value=True, description='plot\_1'), Checkbox(value=False, description='plot\_3'),…
    \end{Verbatim}

    
    \hypertarget{rankine-hugoniot-jump-conditions}{%
\subsection{Rankine-Hugoniot jump
conditions}\label{rankine-hugoniot-jump-conditions}}

The Hugoniot loci for 1- and 3-shocks are \begin{align}
    \rho(p) &= \left(\frac{1 + \beta p/p_0}{p/p_\ell + \beta} \right),\\
    u(p) & = u_0 \pm \frac{2c_0}{\sqrt{2\gamma(\gamma-1)}} 
        \left(\frac{1-p/p_0}{\sqrt{1+\beta p/p_0}}\right), \\
\end{align} where \(\beta = (\gamma+1)/(\gamma-1)\). Here the plus sign
is for 1-shocks and the minus sign is for 3-shocks.

Below we plot the projection of some Hugoniot loci on the \(p-u\) plane.

    \begin{tcolorbox}[breakable, size=fbox, boxrule=1pt, pad at break*=1mm,colback=cellbackground, colframe=cellborder]
\prompt{In}{incolor}{4}{\boxspacing}
\begin{Verbatim}[commandchars=\\\{\}]
\PY{n}{interact}\PY{p}{(}\PY{n}{euler}\PY{o}{.}\PY{n}{plot\PYZus{}hugoniot\PYZus{}loci}\PY{p}{,}
         \PY{n}{gamma}\PY{o}{=}\PY{n}{widgets}\PY{o}{.}\PY{n}{FloatSlider}\PY{p}{(}\PY{n+nb}{min}\PY{o}{=}\PY{l+m+mf}{1.1}\PY{p}{,}\PY{n+nb}{max}\PY{o}{=}\PY{l+m+mi}{3}\PY{p}{,}\PY{n}{value}\PY{o}{=}\PY{l+m+mf}{1.4}\PY{p}{)}\PY{p}{,}
         \PY{n}{rho\PYZus{}0}\PY{o}{=}\PY{n}{widgets}\PY{o}{.}\PY{n}{FloatSlider}\PY{p}{(}\PY{n+nb}{min}\PY{o}{=}\PY{l+m+mf}{0.1}\PY{p}{,}\PY{n+nb}{max}\PY{o}{=}\PY{l+m+mf}{3.}\PY{p}{,}\PY{n}{value}\PY{o}{=}\PY{l+m+mf}{1.}\PY{p}{,} 
                                   \PY{n}{description}\PY{o}{=}\PY{l+s+sa}{r}\PY{l+s+s1}{\PYZsq{}}\PY{l+s+s1}{\PYZdl{}}\PY{l+s+s1}{\PYZbs{}}\PY{l+s+s1}{rho\PYZus{}0\PYZdl{}}\PY{l+s+s1}{\PYZsq{}}\PY{p}{)}\PY{p}{)}\PY{p}{;}
\end{Verbatim}
\end{tcolorbox}

    
    \begin{Verbatim}[commandchars=\\\{\}]
interactive(children=(Checkbox(value=True, description='plot\_1'), Checkbox(value=False, description='plot\_3'),…
    \end{Verbatim}

    
    \hypertarget{entropy-condition}{%
\subsubsection{Entropy condition}\label{entropy-condition}}

As mentioned above, a shock wave is physically relevant only if the
entropy of the gas increases as the gas particles move through the
shock. A discontinuity satisfying the Rankine-Hugoniot jump conditions
that violates this entropy condition (an ``entropy-violating shock'') is
not physically correct and should be replaced by a rarefaction wave in
the Riemann solution.

This physical entropy condition is equivalent to the mathematical
condition that for a 1-shock to be physically relevant, the
1-characteristics must impinge on the shock (the Lax entropy condition).
If the entropy condition is violated, the 1-characteristics would spread
out, allowing the insertion of an expansion fan (rarefaction wave).

    \hypertarget{exact-solution-of-the-riemann-problem}{%
\subsection{Exact solution of the Riemann
problem}\label{exact-solution-of-the-riemann-problem}}

The general Riemann solution is found following the steps listed below.
This is essentially the same procedure used to determine the correct
solution to the Riemann problem for the shallow water equations in
\href{Shallow_water.ipynb}{Shallow\_water}, where more details are
given.

The Euler equations are a system of three equations and the general
Riemann solution consists of three waves, so we must determine two
intermediate states rather than the one intermediate state in the
shallow water equations. However, it is nearly as simple because of the
fact that we know the pressure and velocity are constant across the
2-wave, and so there is a single intermediate pressure \(p_m\) and
velocity \(u_m\) in both intermediate states, and it is only the density
that takes different values \(\rho_{m1}\) and \(\rho_{m2}\). Moreover
any jump in density is allowed across the 2-wave, and we have
expressions given above for how \(u(p)\) varies along any integral curve
or Hugoniot locus, expressions that do not explicitly involve \(\rho\).
So we can determine the intermediate \(p_m\) by finding the intersection
point of two relevant curves, in step 3 of this general algorithm:

\begin{enumerate}
\def\labelenumi{\arabic{enumi}.}
\tightlist
\item
  Define a piecewise function giving the middle state velocity \(u_m\)
  that can be connected to the left state by an entropy-satisfying shock
  or rarefaction, as a function of the middle-state pressure \(p_m\).
\item
  Define a piecewise function giving the middle state velocity \(u_m\)
  that can be connected to the right state by an entropy-satisfying
  shock or rarefaction, as a function of the middle-state pressure
  \(p_m\).
\item
  Use a nonlinear rootfinder to find the intersection of the two
  functions defined above.
\item
  Use the Riemann invariants to find the intermediate state densities
  and the solution structure inside any rarefaction waves.
\end{enumerate}

    Step 4 above requires finding the structure of rarefaction waves. This
can be done using the the fact that the Riemann invariants are constant
through the rarefaction wave. See Chapter 14 of \cite{fvmhp} for more
details.

    \hypertarget{examples-of-riemann-solutions}{%
\subsection{Examples of Riemann
solutions}\label{examples-of-riemann-solutions}}

Here we present some representative examples of Riemann problems and
solutions. The examples chosen are closely related to the examples used
in \href{Shallow_water.ipynb}{Shallow\_water} and you might want to
refer back to that notebook and compare the results.

    \hypertarget{problem-1-sod-shock-tube}{%
\subsubsection{Problem 1: Sod shock
tube}\label{problem-1-sod-shock-tube}}

First we consider the classic shock tube problem. The initial condition
consists of high density and pressure on the left, low density and
pressure on the right and zero velocity on both sides. The solution is
composed of a shock propagating to the right (3-shock), while a
left-going rarefaction forms (1-rarefaction). In between these two
waves, there is a jump in the density, which is the contact
discontinuity (2-wave) in the linearly degenerate characteristic field.

Note that this set of initial conditions is analogous to the ``dam
break'' problem for shallow water quations, and the resulting structure
of the solution is very similar to that obtained when those equations
are solved with the addition of a scalar tracer. However, in the Euler
equations the entropy jump across a 2-wave does affect the fluid
dynamics on either side, so this is not a passive tracer and solving the
Riemann problem is slightly more complex.

    \begin{tcolorbox}[breakable, size=fbox, boxrule=1pt, pad at break*=1mm,colback=cellbackground, colframe=cellborder]
\prompt{In}{incolor}{5}{\boxspacing}
\begin{Verbatim}[commandchars=\\\{\}]
\PY{n}{left\PYZus{}state}  \PY{o}{=} \PY{n}{State}\PY{p}{(}\PY{n}{Density} \PY{o}{=} \PY{l+m+mf}{3.}\PY{p}{,}
                    \PY{n}{Velocity} \PY{o}{=} \PY{l+m+mf}{0.}\PY{p}{,}
                    \PY{n}{Pressure} \PY{o}{=} \PY{l+m+mf}{3.}\PY{p}{)}
\PY{n}{right\PYZus{}state} \PY{o}{=} \PY{n}{State}\PY{p}{(}\PY{n}{Density} \PY{o}{=} \PY{l+m+mf}{1.}\PY{p}{,}
                    \PY{n}{Velocity} \PY{o}{=} \PY{l+m+mf}{0.}\PY{p}{,}
                    \PY{n}{Pressure} \PY{o}{=} \PY{l+m+mf}{1.}\PY{p}{)}

\PY{n}{euler}\PY{o}{.}\PY{n}{riemann\PYZus{}solution}\PY{p}{(}\PY{n}{left\PYZus{}state}\PY{p}{,}\PY{n}{right\PYZus{}state}\PY{p}{)}
\end{Verbatim}
\end{tcolorbox}

    
    \begin{Verbatim}[commandchars=\\\{\}]
interactive(children=(FloatSlider(value=0.5, description='t', max=0.9), Dropdown(description='Show characteris…
    \end{Verbatim}

    
    Here is a plot of the solution in the phase plane, showing the integral
curve connecting the left and middle states, and the Hugoniot locus
connecting the middle and right states.

    \begin{tcolorbox}[breakable, size=fbox, boxrule=1pt, pad at break*=1mm,colback=cellbackground, colframe=cellborder]
\prompt{In}{incolor}{6}{\boxspacing}
\begin{Verbatim}[commandchars=\\\{\}]
\PY{n}{euler}\PY{o}{.}\PY{n}{phase\PYZus{}plane\PYZus{}plot}\PY{p}{(}\PY{n}{left\PYZus{}state}\PY{p}{,} \PY{n}{right\PYZus{}state}\PY{p}{)}
\end{Verbatim}
\end{tcolorbox}

    \begin{center}
    \adjustimage{max size={0.9\linewidth}{0.9\paperheight}}{Euler_files/Euler_32_0.pdf}
    \end{center}
    { \hspace*{\fill} \\}
    
    \hypertarget{problem-2-symmetric-expansion}{%
\subsubsection{Problem 2: Symmetric
expansion}\label{problem-2-symmetric-expansion}}

Next we consider the case of equal densities and pressures, and equal
and opposite velocities, with the initial states moving away from each
other. The result is two rarefaction waves (the contact has zero
strength).

    \begin{tcolorbox}[breakable, size=fbox, boxrule=1pt, pad at break*=1mm,colback=cellbackground, colframe=cellborder]
\prompt{In}{incolor}{7}{\boxspacing}
\begin{Verbatim}[commandchars=\\\{\}]
\PY{n}{left\PYZus{}state}  \PY{o}{=} \PY{n}{State}\PY{p}{(}\PY{n}{Density} \PY{o}{=} \PY{l+m+mf}{1.}\PY{p}{,}
                    \PY{n}{Velocity} \PY{o}{=} \PY{o}{\PYZhy{}}\PY{l+m+mf}{3.}\PY{p}{,}
                    \PY{n}{Pressure} \PY{o}{=} \PY{l+m+mf}{1.}\PY{p}{)}
\PY{n}{right\PYZus{}state} \PY{o}{=} \PY{n}{State}\PY{p}{(}\PY{n}{Density} \PY{o}{=} \PY{l+m+mf}{1.}\PY{p}{,}
                    \PY{n}{Velocity} \PY{o}{=} \PY{l+m+mf}{3.}\PY{p}{,}
                    \PY{n}{Pressure} \PY{o}{=} \PY{l+m+mf}{1.}\PY{p}{)}

\PY{n}{euler}\PY{o}{.}\PY{n}{riemann\PYZus{}solution}\PY{p}{(}\PY{n}{left\PYZus{}state}\PY{p}{,}\PY{n}{right\PYZus{}state}\PY{p}{)}\PY{p}{;}
\end{Verbatim}
\end{tcolorbox}

    
    \begin{Verbatim}[commandchars=\\\{\}]
interactive(children=(FloatSlider(value=0.5, description='t', max=0.9), Dropdown(description='Show characteris…
    \end{Verbatim}

    
    \begin{tcolorbox}[breakable, size=fbox, boxrule=1pt, pad at break*=1mm,colback=cellbackground, colframe=cellborder]
\prompt{In}{incolor}{8}{\boxspacing}
\begin{Verbatim}[commandchars=\\\{\}]
\PY{n}{euler}\PY{o}{.}\PY{n}{phase\PYZus{}plane\PYZus{}plot}\PY{p}{(}\PY{n}{left\PYZus{}state}\PY{p}{,} \PY{n}{right\PYZus{}state}\PY{p}{)}
\end{Verbatim}
\end{tcolorbox}

    \begin{center}
    \adjustimage{max size={0.9\linewidth}{0.9\paperheight}}{Euler_files/Euler_35_0.pdf}
    \end{center}
    { \hspace*{\fill} \\}
    
    \hypertarget{problem-3-colliding-flows}{%
\subsubsection{Problem 3: Colliding
flows}\label{problem-3-colliding-flows}}

Next, consider the case in which the left and right states are moving
toward each other. This leads to a pair of shocks, with a high-density,
high-pressure state in between.

    \begin{tcolorbox}[breakable, size=fbox, boxrule=1pt, pad at break*=1mm,colback=cellbackground, colframe=cellborder]
\prompt{In}{incolor}{9}{\boxspacing}
\begin{Verbatim}[commandchars=\\\{\}]
\PY{n}{left\PYZus{}state}  \PY{o}{=} \PY{n}{State}\PY{p}{(}\PY{n}{Density} \PY{o}{=} \PY{l+m+mf}{1.}\PY{p}{,}
                    \PY{n}{Velocity} \PY{o}{=} \PY{l+m+mf}{3.}\PY{p}{,}
                    \PY{n}{Pressure} \PY{o}{=} \PY{l+m+mf}{1.}\PY{p}{)}
\PY{n}{right\PYZus{}state} \PY{o}{=} \PY{n}{State}\PY{p}{(}\PY{n}{Density} \PY{o}{=} \PY{l+m+mf}{1.}\PY{p}{,}
                    \PY{n}{Velocity} \PY{o}{=} \PY{o}{\PYZhy{}}\PY{l+m+mf}{3.}\PY{p}{,}
                    \PY{n}{Pressure} \PY{o}{=} \PY{l+m+mf}{1.}\PY{p}{)}

\PY{n}{euler}\PY{o}{.}\PY{n}{riemann\PYZus{}solution}\PY{p}{(}\PY{n}{left\PYZus{}state}\PY{p}{,}\PY{n}{right\PYZus{}state}\PY{p}{)}
\end{Verbatim}
\end{tcolorbox}

    
    \begin{Verbatim}[commandchars=\\\{\}]
interactive(children=(FloatSlider(value=0.5, description='t', max=0.9), Dropdown(description='Show characteris…
    \end{Verbatim}

    
    \begin{tcolorbox}[breakable, size=fbox, boxrule=1pt, pad at break*=1mm,colback=cellbackground, colframe=cellborder]
\prompt{In}{incolor}{10}{\boxspacing}
\begin{Verbatim}[commandchars=\\\{\}]
\PY{n}{euler}\PY{o}{.}\PY{n}{phase\PYZus{}plane\PYZus{}plot}\PY{p}{(}\PY{n}{left\PYZus{}state}\PY{p}{,} \PY{n}{right\PYZus{}state}\PY{p}{)}
\end{Verbatim}
\end{tcolorbox}

    \begin{center}
    \adjustimage{max size={0.9\linewidth}{0.9\paperheight}}{Euler_files/Euler_38_0.pdf}
    \end{center}
    { \hspace*{\fill} \\}
    
    \hypertarget{plot-particle-trajectories}{%
\subsection{Plot particle
trajectories}\label{plot-particle-trajectories}}

In the next plot of the Riemann solution in the \(x\)-\(t\) plane, we
also plot the trajectories of a set of particles initially distributed
along the \(x\) axis at \(t=0\), with the spacing inversely proportional
to the density. The evolution of the distance between particles gives an
indication of how the density changes.

    \begin{tcolorbox}[breakable, size=fbox, boxrule=1pt, pad at break*=1mm,colback=cellbackground, colframe=cellborder]
\prompt{In}{incolor}{11}{\boxspacing}
\begin{Verbatim}[commandchars=\\\{\}]
\PY{n}{left\PYZus{}state}  \PY{o}{=} \PY{n}{State}\PY{p}{(}\PY{n}{Density} \PY{o}{=} \PY{l+m+mf}{3.}\PY{p}{,}
                    \PY{n}{Velocity} \PY{o}{=} \PY{l+m+mf}{0.}\PY{p}{,}
                    \PY{n}{Pressure} \PY{o}{=} \PY{l+m+mf}{3.}\PY{p}{)}
\PY{n}{right\PYZus{}state} \PY{o}{=} \PY{n}{State}\PY{p}{(}\PY{n}{Density} \PY{o}{=} \PY{l+m+mf}{1.}\PY{p}{,}
                    \PY{n}{Velocity} \PY{o}{=} \PY{l+m+mf}{0.}\PY{p}{,}
                    \PY{n}{Pressure} \PY{o}{=} \PY{l+m+mf}{1.}\PY{p}{)}

\PY{n}{euler}\PY{o}{.}\PY{n}{plot\PYZus{}riemann\PYZus{}trajectories}\PY{p}{(}\PY{n}{left\PYZus{}state}\PY{p}{,} \PY{n}{right\PYZus{}state}\PY{p}{)}
\end{Verbatim}
\end{tcolorbox}

    \begin{center}
    \adjustimage{max size={0.9\linewidth}{0.9\paperheight}}{Euler_files/Euler_40_0.pdf}
    \end{center}
    { \hspace*{\fill} \\}
    
    Since the distance between particles in the above plot is inversely
proportional to density, we see that the density around a particle
increases as it goes through the shock wave but decreases through the
rarefaction wave, and that in general there is a jump in density across
the contact discontinuity, which lies along the particle trajectory
emanating from \(x=0\) at \(t=0\).

    \hypertarget{riemann-solution-with-a-colored-tracer}{%
\subsection{Riemann solution with a colored
tracer}\label{riemann-solution-with-a-colored-tracer}}

Next we plot the Riemann solution with the density plot also showing an
advected color to help visualize the flow better. The fluid initially to
the left of \(x=0\) is colored red and that initially to the right of
\(x=0\) is colored blue, with stripes of different shades of these
colors to help visualize the motion of the fluid.

    Let's plot the Sod shock tube data with this colored tracer:

    \begin{tcolorbox}[breakable, size=fbox, boxrule=1pt, pad at break*=1mm,colback=cellbackground, colframe=cellborder]
\prompt{In}{incolor}{12}{\boxspacing}
\begin{Verbatim}[commandchars=\\\{\}]
\PY{k}{def} \PY{n+nf}{plot\PYZus{}with\PYZus{}stripes\PYZus{}t\PYZus{}slider}\PY{p}{(}\PY{n}{t}\PY{p}{)}\PY{p}{:}
    \PY{n}{euler\PYZus{}demos}\PY{o}{.}\PY{n}{plot\PYZus{}with\PYZus{}stripes}\PY{p}{(}\PY{n}{rho\PYZus{}l}\PY{o}{=}\PY{l+m+mf}{3.}\PY{p}{,}\PY{n}{u\PYZus{}l}\PY{o}{=}\PY{l+m+mf}{0.}\PY{p}{,}\PY{n}{p\PYZus{}l}\PY{o}{=}\PY{l+m+mf}{3.}\PY{p}{,}
                                  \PY{n}{rho\PYZus{}r}\PY{o}{=}\PY{l+m+mf}{1.}\PY{p}{,}\PY{n}{u\PYZus{}r}\PY{o}{=}\PY{l+m+mf}{0.}\PY{p}{,}\PY{n}{p\PYZus{}r}\PY{o}{=}\PY{l+m+mf}{1.}\PY{p}{,}
                                  \PY{n}{gamma}\PY{o}{=}\PY{n}{gamma}\PY{p}{,}\PY{n}{t}\PY{o}{=}\PY{n}{t}\PY{p}{)}
    
\PY{n}{interact}\PY{p}{(}\PY{n}{plot\PYZus{}with\PYZus{}stripes\PYZus{}t\PYZus{}slider}\PY{p}{,} 
         \PY{n}{t}\PY{o}{=}\PY{n}{widgets}\PY{o}{.}\PY{n}{FloatSlider}\PY{p}{(}\PY{n+nb}{min}\PY{o}{=}\PY{l+m+mf}{0.}\PY{p}{,}\PY{n+nb}{max}\PY{o}{=}\PY{l+m+mf}{1.}\PY{p}{,}\PY{n}{step}\PY{o}{=}\PY{l+m+mf}{0.1}\PY{p}{,}\PY{n}{value}\PY{o}{=}\PY{l+m+mf}{0.5}\PY{p}{)}\PY{p}{)}\PY{p}{;}
\end{Verbatim}
\end{tcolorbox}

    
    \begin{Verbatim}[commandchars=\\\{\}]
interactive(children=(FloatSlider(value=0.5, description='t', max=1.0), Output()), \_dom\_classes=('widget-inter…
    \end{Verbatim}

    
    Note the following in the figure above:

\begin{itemize}
\tightlist
\item
  The edges of each stripe are being advected with the fluid velocity,
  so you can visualize how the fluid is moving.
\item
  The width of each stripe initially is inversely proportional to the
  density of the fluid, so that the total mass of gas within each stripe
  is the same.
\item
  The total mass within each stripe remains constant as the flow
  evolves, and the width of each stripe remains inversely proportional
  to the local density.
\item
  The interface between the red and blue gas moves with the contact
  discontinuity. The velocity and pressure are constant but the density
  can vary across this wave.
\end{itemize}

    \hypertarget{interactive-riemann-solver}{%
\subsection{Interactive Riemann
solver}\label{interactive-riemann-solver}}

The initial configuration specified below gives a rather different
looking solution than when using initial conditions of Sod, but with the
same mathematical structure. \emph{In the live notebook, you can easily
adjust the initial data and immediately see the resulting solution.}

    \begin{tcolorbox}[breakable, size=fbox, boxrule=1pt, pad at break*=1mm,colback=cellbackground, colframe=cellborder]
\prompt{In}{incolor}{13}{\boxspacing}
\begin{Verbatim}[commandchars=\\\{\}]
\PY{n}{euler\PYZus{}demos}\PY{o}{.}\PY{n}{euler\PYZus{}demo1}\PY{p}{(}\PY{n}{rho\PYZus{}l}\PY{o}{=}\PY{l+m+mf}{2.}\PY{p}{,}\PY{n}{u\PYZus{}l}\PY{o}{=}\PY{l+m+mf}{0.}\PY{p}{,}\PY{n}{p\PYZus{}l}\PY{o}{=}\PY{l+m+mf}{2.5}\PY{p}{,}
                        \PY{n}{rho\PYZus{}r}\PY{o}{=}\PY{l+m+mf}{3.}\PY{p}{,}\PY{n}{u\PYZus{}r}\PY{o}{=}\PY{l+m+mf}{0.}\PY{p}{,}\PY{n}{p\PYZus{}r}\PY{o}{=}\PY{l+m+mf}{5.}\PY{p}{,} \PY{n}{gamma}\PY{o}{=}\PY{n}{gamma}\PY{p}{)}
\end{Verbatim}
\end{tcolorbox}

    
    \begin{Verbatim}[commandchars=\\\{\}]
interactive(children=(FloatSlider(value=2.0, description='\$\textbackslash{}\textbackslash{}rho\_l\$', max=10.0, min=1.0), FloatSlider(value=0.…
    \end{Verbatim}

    
    
    \begin{Verbatim}[commandchars=\\\{\}]
VBox(children=(HBox(children=(FloatSlider(value=2.0, description='\$\textbackslash{}\textbackslash{}rho\_l\$', max=10.0, min=1.0), FloatSlider(…
    \end{Verbatim}

    
    
    \begin{Verbatim}[commandchars=\\\{\}]
Output()
    \end{Verbatim}

    
    \hypertarget{riemann-problems-with-vacuum}{%
\subsection{Riemann problems with
vacuum}\label{riemann-problems-with-vacuum}}

A vacuum state (with zero pressure and density) in the Euler equations
is similar to a dry state (with depth \(h=0\)) in the shallow water
equations. It can arise in the solution of the Riemann problem in two
ways:

\begin{enumerate}
\def\labelenumi{\arabic{enumi}.}
\tightlist
\item
  An initial left or right vacuum state: in this case the Riemann
  solution consists of a single rarefaction, connecting the non-vacuum
  state to vacuum.
\item
  A problem where the left and right states are not vacuum but middle
  states are vacuum. Since this means the middle pressure is smaller
  than that to the left or right, this can occur only if the 1- and
  3-waves are both rarefactions. These rarefactions are precisely those
  required to connect the left and right states to the middle vacuum
  state.
\end{enumerate}

    \hypertarget{initial-vacuum-state}{%
\subsubsection{Initial vacuum state}\label{initial-vacuum-state}}

Next we start with the density and pressure set to 0 in the left state.
The velocity plot looks a bit strange, but note that the velocity is
undefined in vacuum. The solution structure consists of a rarefaction
wave, similar to what is observed in the analogous case of a dam break
problem with dry land on one side (depth \(=0\)), as discussed in
\href{Shallow_water.ipynb}{Shallow\_water}.

    \begin{tcolorbox}[breakable, size=fbox, boxrule=1pt, pad at break*=1mm,colback=cellbackground, colframe=cellborder]
\prompt{In}{incolor}{14}{\boxspacing}
\begin{Verbatim}[commandchars=\\\{\}]
\PY{n}{left\PYZus{}state}  \PY{o}{=} \PY{n}{State}\PY{p}{(}\PY{n}{Density} \PY{o}{=}\PY{l+m+mf}{0.}\PY{p}{,}
                    \PY{n}{Velocity} \PY{o}{=} \PY{l+m+mf}{0.}\PY{p}{,}
                    \PY{n}{Pressure} \PY{o}{=} \PY{l+m+mf}{0.}\PY{p}{)}
\PY{n}{right\PYZus{}state} \PY{o}{=} \PY{n}{State}\PY{p}{(}\PY{n}{Density} \PY{o}{=} \PY{l+m+mf}{1.}\PY{p}{,}
                    \PY{n}{Velocity} \PY{o}{=} \PY{o}{\PYZhy{}}\PY{l+m+mf}{3.}\PY{p}{,}
                    \PY{n}{Pressure} \PY{o}{=} \PY{l+m+mf}{1.}\PY{p}{)}

\PY{n}{euler}\PY{o}{.}\PY{n}{riemann\PYZus{}solution}\PY{p}{(}\PY{n}{left\PYZus{}state}\PY{p}{,}\PY{n}{right\PYZus{}state}\PY{p}{)}
\end{Verbatim}
\end{tcolorbox}

    
    \begin{Verbatim}[commandchars=\\\{\}]
interactive(children=(FloatSlider(value=0.5, description='t', max=0.9), Dropdown(description='Show characteris…
    \end{Verbatim}

    
    \begin{tcolorbox}[breakable, size=fbox, boxrule=1pt, pad at break*=1mm,colback=cellbackground, colframe=cellborder]
\prompt{In}{incolor}{15}{\boxspacing}
\begin{Verbatim}[commandchars=\\\{\}]
\PY{n}{euler}\PY{o}{.}\PY{n}{phase\PYZus{}plane\PYZus{}plot}\PY{p}{(}\PY{n}{left\PYZus{}state}\PY{p}{,} \PY{n}{right\PYZus{}state}\PY{p}{)}
\end{Verbatim}
\end{tcolorbox}

    \begin{center}
    \adjustimage{max size={0.9\linewidth}{0.9\paperheight}}{Euler_files/Euler_51_0.pdf}
    \end{center}
    { \hspace*{\fill} \\}
    
    The phase plane plot may look odd, but recall that in the vacuum state
velocity is undefined, and since \(p_\ell = p_m = 0\), the left and
middle states are actually the same.

    \hypertarget{middle-vacuum-state}{%
\subsubsection{Middle vacuum state}\label{middle-vacuum-state}}

Finally, we consider an example where there is sufficiently strong
outflow (\(u_\ell<0\) and \(u_r>0\)) that a vacuum state forms,
analogous to the dry state that appears in the similar example in
\href{Shallow_water.ipynb}{Shallow\_water}.

    \begin{tcolorbox}[breakable, size=fbox, boxrule=1pt, pad at break*=1mm,colback=cellbackground, colframe=cellborder]
\prompt{In}{incolor}{16}{\boxspacing}
\begin{Verbatim}[commandchars=\\\{\}]
\PY{n}{left\PYZus{}state}  \PY{o}{=} \PY{n}{State}\PY{p}{(}\PY{n}{Density} \PY{o}{=}\PY{l+m+mf}{1.}\PY{p}{,}
                    \PY{n}{Velocity} \PY{o}{=} \PY{o}{\PYZhy{}}\PY{l+m+mf}{10.}\PY{p}{,}
                    \PY{n}{Pressure} \PY{o}{=} \PY{l+m+mf}{1.}\PY{p}{)}
\PY{n}{right\PYZus{}state} \PY{o}{=} \PY{n}{State}\PY{p}{(}\PY{n}{Density} \PY{o}{=} \PY{l+m+mf}{1.}\PY{p}{,}
                    \PY{n}{Velocity} \PY{o}{=} \PY{l+m+mf}{10.}\PY{p}{,}
                    \PY{n}{Pressure} \PY{o}{=} \PY{l+m+mf}{1.}\PY{p}{)}

\PY{n}{euler}\PY{o}{.}\PY{n}{riemann\PYZus{}solution}\PY{p}{(}\PY{n}{left\PYZus{}state}\PY{p}{,}\PY{n}{right\PYZus{}state}\PY{p}{)}
\end{Verbatim}
\end{tcolorbox}

    
    \begin{Verbatim}[commandchars=\\\{\}]
interactive(children=(FloatSlider(value=0.5, description='t', max=0.9), Dropdown(description='Show characteris…
    \end{Verbatim}

    
    \begin{tcolorbox}[breakable, size=fbox, boxrule=1pt, pad at break*=1mm,colback=cellbackground, colframe=cellborder]
\prompt{In}{incolor}{17}{\boxspacing}
\begin{Verbatim}[commandchars=\\\{\}]
\PY{n}{euler}\PY{o}{.}\PY{n}{phase\PYZus{}plane\PYZus{}plot}\PY{p}{(}\PY{n}{left\PYZus{}state}\PY{p}{,} \PY{n}{right\PYZus{}state}\PY{p}{)}
\end{Verbatim}
\end{tcolorbox}

    \begin{center}
    \adjustimage{max size={0.9\linewidth}{0.9\paperheight}}{Euler_files/Euler_55_0.pdf}
    \end{center}
    { \hspace*{\fill} \\}
    
    Again the phase plane plot may look odd, but since the velocity is
undefined in the vacuum state the middle state with \(p_m = 0\) actually
connects to both integral curves, which correspond to the two outgoing
rarefaction waves.

    \begin{tcolorbox}[breakable, size=fbox, boxrule=1pt, pad at break*=1mm,colback=cellbackground, colframe=cellborder]
\prompt{In}{incolor}{ }{\boxspacing}
\begin{Verbatim}[commandchars=\\\{\}]

\end{Verbatim}
\end{tcolorbox}


    % Add a bibliography block to the postdoc
    
    
    
\end{document}
