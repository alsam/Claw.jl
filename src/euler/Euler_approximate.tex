\documentclass[11pt]{article}

    \usepackage[breakable]{tcolorbox}
    \usepackage{parskip} % Stop auto-indenting (to mimic markdown behaviour)
    
    \usepackage{iftex}
    \ifPDFTeX
    	\usepackage[T1]{fontenc}
    	\usepackage{mathpazo}
    \else
    	\usepackage{fontspec}
    \fi

    % Basic figure setup, for now with no caption control since it's done
    % automatically by Pandoc (which extracts ![](path) syntax from Markdown).
    \usepackage{graphicx}
    % Maintain compatibility with old templates. Remove in nbconvert 6.0
    \let\Oldincludegraphics\includegraphics
    % Ensure that by default, figures have no caption (until we provide a
    % proper Figure object with a Caption API and a way to capture that
    % in the conversion process - todo).
    \usepackage{caption}
    \DeclareCaptionFormat{nocaption}{}
    \captionsetup{format=nocaption,aboveskip=0pt,belowskip=0pt}

    \usepackage{float}
    \floatplacement{figure}{H} % forces figures to be placed at the correct location
    \usepackage{xcolor} % Allow colors to be defined
    \usepackage{enumerate} % Needed for markdown enumerations to work
    \usepackage{geometry} % Used to adjust the document margins
    \usepackage{amsmath} % Equations
    \usepackage{amssymb} % Equations
    \usepackage{textcomp} % defines textquotesingle
    % Hack from http://tex.stackexchange.com/a/47451/13684:
    \AtBeginDocument{%
        \def\PYZsq{\textquotesingle}% Upright quotes in Pygmentized code
    }
    \usepackage{upquote} % Upright quotes for verbatim code
    \usepackage{eurosym} % defines \euro
    \usepackage[mathletters]{ucs} % Extended unicode (utf-8) support
    \usepackage{fancyvrb} % verbatim replacement that allows latex
    \usepackage{grffile} % extends the file name processing of package graphics 
                         % to support a larger range
    \makeatletter % fix for old versions of grffile with XeLaTeX
    \@ifpackagelater{grffile}{2019/11/01}
    {
      % Do nothing on new versions
    }
    {
      \def\Gread@@xetex#1{%
        \IfFileExists{"\Gin@base".bb}%
        {\Gread@eps{\Gin@base.bb}}%
        {\Gread@@xetex@aux#1}%
      }
    }
    \makeatother
    \usepackage[Export]{adjustbox} % Used to constrain images to a maximum size
    \adjustboxset{max size={0.9\linewidth}{0.9\paperheight}}

    % The hyperref package gives us a pdf with properly built
    % internal navigation ('pdf bookmarks' for the table of contents,
    % internal cross-reference links, web links for URLs, etc.)
    \usepackage{hyperref}
    % The default LaTeX title has an obnoxious amount of whitespace. By default,
    % titling removes some of it. It also provides customization options.
    \usepackage{titling}
    \usepackage{longtable} % longtable support required by pandoc >1.10
    \usepackage{booktabs}  % table support for pandoc > 1.12.2
    \usepackage[inline]{enumitem} % IRkernel/repr support (it uses the enumerate* environment)
    \usepackage[normalem]{ulem} % ulem is needed to support strikethroughs (\sout)
                                % normalem makes italics be italics, not underlines
    \usepackage{mathrsfs}
    

    
    % Colors for the hyperref package
    \definecolor{urlcolor}{rgb}{0,.145,.698}
    \definecolor{linkcolor}{rgb}{.71,0.21,0.01}
    \definecolor{citecolor}{rgb}{.12,.54,.11}

    % ANSI colors
    \definecolor{ansi-black}{HTML}{3E424D}
    \definecolor{ansi-black-intense}{HTML}{282C36}
    \definecolor{ansi-red}{HTML}{E75C58}
    \definecolor{ansi-red-intense}{HTML}{B22B31}
    \definecolor{ansi-green}{HTML}{00A250}
    \definecolor{ansi-green-intense}{HTML}{007427}
    \definecolor{ansi-yellow}{HTML}{DDB62B}
    \definecolor{ansi-yellow-intense}{HTML}{B27D12}
    \definecolor{ansi-blue}{HTML}{208FFB}
    \definecolor{ansi-blue-intense}{HTML}{0065CA}
    \definecolor{ansi-magenta}{HTML}{D160C4}
    \definecolor{ansi-magenta-intense}{HTML}{A03196}
    \definecolor{ansi-cyan}{HTML}{60C6C8}
    \definecolor{ansi-cyan-intense}{HTML}{258F8F}
    \definecolor{ansi-white}{HTML}{C5C1B4}
    \definecolor{ansi-white-intense}{HTML}{A1A6B2}
    \definecolor{ansi-default-inverse-fg}{HTML}{FFFFFF}
    \definecolor{ansi-default-inverse-bg}{HTML}{000000}

    % common color for the border for error outputs.
    \definecolor{outerrorbackground}{HTML}{FFDFDF}

    % commands and environments needed by pandoc snippets
    % extracted from the output of `pandoc -s`
    \providecommand{\tightlist}{%
      \setlength{\itemsep}{0pt}\setlength{\parskip}{0pt}}
    \DefineVerbatimEnvironment{Highlighting}{Verbatim}{commandchars=\\\{\}}
    % Add ',fontsize=\small' for more characters per line
    \newenvironment{Shaded}{}{}
    \newcommand{\KeywordTok}[1]{\textcolor[rgb]{0.00,0.44,0.13}{\textbf{{#1}}}}
    \newcommand{\DataTypeTok}[1]{\textcolor[rgb]{0.56,0.13,0.00}{{#1}}}
    \newcommand{\DecValTok}[1]{\textcolor[rgb]{0.25,0.63,0.44}{{#1}}}
    \newcommand{\BaseNTok}[1]{\textcolor[rgb]{0.25,0.63,0.44}{{#1}}}
    \newcommand{\FloatTok}[1]{\textcolor[rgb]{0.25,0.63,0.44}{{#1}}}
    \newcommand{\CharTok}[1]{\textcolor[rgb]{0.25,0.44,0.63}{{#1}}}
    \newcommand{\StringTok}[1]{\textcolor[rgb]{0.25,0.44,0.63}{{#1}}}
    \newcommand{\CommentTok}[1]{\textcolor[rgb]{0.38,0.63,0.69}{\textit{{#1}}}}
    \newcommand{\OtherTok}[1]{\textcolor[rgb]{0.00,0.44,0.13}{{#1}}}
    \newcommand{\AlertTok}[1]{\textcolor[rgb]{1.00,0.00,0.00}{\textbf{{#1}}}}
    \newcommand{\FunctionTok}[1]{\textcolor[rgb]{0.02,0.16,0.49}{{#1}}}
    \newcommand{\RegionMarkerTok}[1]{{#1}}
    \newcommand{\ErrorTok}[1]{\textcolor[rgb]{1.00,0.00,0.00}{\textbf{{#1}}}}
    \newcommand{\NormalTok}[1]{{#1}}
    
    % Additional commands for more recent versions of Pandoc
    \newcommand{\ConstantTok}[1]{\textcolor[rgb]{0.53,0.00,0.00}{{#1}}}
    \newcommand{\SpecialCharTok}[1]{\textcolor[rgb]{0.25,0.44,0.63}{{#1}}}
    \newcommand{\VerbatimStringTok}[1]{\textcolor[rgb]{0.25,0.44,0.63}{{#1}}}
    \newcommand{\SpecialStringTok}[1]{\textcolor[rgb]{0.73,0.40,0.53}{{#1}}}
    \newcommand{\ImportTok}[1]{{#1}}
    \newcommand{\DocumentationTok}[1]{\textcolor[rgb]{0.73,0.13,0.13}{\textit{{#1}}}}
    \newcommand{\AnnotationTok}[1]{\textcolor[rgb]{0.38,0.63,0.69}{\textbf{\textit{{#1}}}}}
    \newcommand{\CommentVarTok}[1]{\textcolor[rgb]{0.38,0.63,0.69}{\textbf{\textit{{#1}}}}}
    \newcommand{\VariableTok}[1]{\textcolor[rgb]{0.10,0.09,0.49}{{#1}}}
    \newcommand{\ControlFlowTok}[1]{\textcolor[rgb]{0.00,0.44,0.13}{\textbf{{#1}}}}
    \newcommand{\OperatorTok}[1]{\textcolor[rgb]{0.40,0.40,0.40}{{#1}}}
    \newcommand{\BuiltInTok}[1]{{#1}}
    \newcommand{\ExtensionTok}[1]{{#1}}
    \newcommand{\PreprocessorTok}[1]{\textcolor[rgb]{0.74,0.48,0.00}{{#1}}}
    \newcommand{\AttributeTok}[1]{\textcolor[rgb]{0.49,0.56,0.16}{{#1}}}
    \newcommand{\InformationTok}[1]{\textcolor[rgb]{0.38,0.63,0.69}{\textbf{\textit{{#1}}}}}
    \newcommand{\WarningTok}[1]{\textcolor[rgb]{0.38,0.63,0.69}{\textbf{\textit{{#1}}}}}
    
    
    % Define a nice break command that doesn't care if a line doesn't already
    % exist.
    \def\br{\hspace*{\fill} \\* }
    % Math Jax compatibility definitions
    \def\gt{>}
    \def\lt{<}
    \let\Oldtex\TeX
    \let\Oldlatex\LaTeX
    \renewcommand{\TeX}{\textrm{\Oldtex}}
    \renewcommand{\LaTeX}{\textrm{\Oldlatex}}
    % Document parameters
    % Document title
    \title{Euler\_approximate}
    
    
    
    
    
% Pygments definitions
\makeatletter
\def\PY@reset{\let\PY@it=\relax \let\PY@bf=\relax%
    \let\PY@ul=\relax \let\PY@tc=\relax%
    \let\PY@bc=\relax \let\PY@ff=\relax}
\def\PY@tok#1{\csname PY@tok@#1\endcsname}
\def\PY@toks#1+{\ifx\relax#1\empty\else%
    \PY@tok{#1}\expandafter\PY@toks\fi}
\def\PY@do#1{\PY@bc{\PY@tc{\PY@ul{%
    \PY@it{\PY@bf{\PY@ff{#1}}}}}}}
\def\PY#1#2{\PY@reset\PY@toks#1+\relax+\PY@do{#2}}

\expandafter\def\csname PY@tok@w\endcsname{\def\PY@tc##1{\textcolor[rgb]{0.73,0.73,0.73}{##1}}}
\expandafter\def\csname PY@tok@c\endcsname{\let\PY@it=\textit\def\PY@tc##1{\textcolor[rgb]{0.25,0.50,0.50}{##1}}}
\expandafter\def\csname PY@tok@cp\endcsname{\def\PY@tc##1{\textcolor[rgb]{0.74,0.48,0.00}{##1}}}
\expandafter\def\csname PY@tok@k\endcsname{\let\PY@bf=\textbf\def\PY@tc##1{\textcolor[rgb]{0.00,0.50,0.00}{##1}}}
\expandafter\def\csname PY@tok@kp\endcsname{\def\PY@tc##1{\textcolor[rgb]{0.00,0.50,0.00}{##1}}}
\expandafter\def\csname PY@tok@kt\endcsname{\def\PY@tc##1{\textcolor[rgb]{0.69,0.00,0.25}{##1}}}
\expandafter\def\csname PY@tok@o\endcsname{\def\PY@tc##1{\textcolor[rgb]{0.40,0.40,0.40}{##1}}}
\expandafter\def\csname PY@tok@ow\endcsname{\let\PY@bf=\textbf\def\PY@tc##1{\textcolor[rgb]{0.67,0.13,1.00}{##1}}}
\expandafter\def\csname PY@tok@nb\endcsname{\def\PY@tc##1{\textcolor[rgb]{0.00,0.50,0.00}{##1}}}
\expandafter\def\csname PY@tok@nf\endcsname{\def\PY@tc##1{\textcolor[rgb]{0.00,0.00,1.00}{##1}}}
\expandafter\def\csname PY@tok@nc\endcsname{\let\PY@bf=\textbf\def\PY@tc##1{\textcolor[rgb]{0.00,0.00,1.00}{##1}}}
\expandafter\def\csname PY@tok@nn\endcsname{\let\PY@bf=\textbf\def\PY@tc##1{\textcolor[rgb]{0.00,0.00,1.00}{##1}}}
\expandafter\def\csname PY@tok@ne\endcsname{\let\PY@bf=\textbf\def\PY@tc##1{\textcolor[rgb]{0.82,0.25,0.23}{##1}}}
\expandafter\def\csname PY@tok@nv\endcsname{\def\PY@tc##1{\textcolor[rgb]{0.10,0.09,0.49}{##1}}}
\expandafter\def\csname PY@tok@no\endcsname{\def\PY@tc##1{\textcolor[rgb]{0.53,0.00,0.00}{##1}}}
\expandafter\def\csname PY@tok@nl\endcsname{\def\PY@tc##1{\textcolor[rgb]{0.63,0.63,0.00}{##1}}}
\expandafter\def\csname PY@tok@ni\endcsname{\let\PY@bf=\textbf\def\PY@tc##1{\textcolor[rgb]{0.60,0.60,0.60}{##1}}}
\expandafter\def\csname PY@tok@na\endcsname{\def\PY@tc##1{\textcolor[rgb]{0.49,0.56,0.16}{##1}}}
\expandafter\def\csname PY@tok@nt\endcsname{\let\PY@bf=\textbf\def\PY@tc##1{\textcolor[rgb]{0.00,0.50,0.00}{##1}}}
\expandafter\def\csname PY@tok@nd\endcsname{\def\PY@tc##1{\textcolor[rgb]{0.67,0.13,1.00}{##1}}}
\expandafter\def\csname PY@tok@s\endcsname{\def\PY@tc##1{\textcolor[rgb]{0.73,0.13,0.13}{##1}}}
\expandafter\def\csname PY@tok@sd\endcsname{\let\PY@it=\textit\def\PY@tc##1{\textcolor[rgb]{0.73,0.13,0.13}{##1}}}
\expandafter\def\csname PY@tok@si\endcsname{\let\PY@bf=\textbf\def\PY@tc##1{\textcolor[rgb]{0.73,0.40,0.53}{##1}}}
\expandafter\def\csname PY@tok@se\endcsname{\let\PY@bf=\textbf\def\PY@tc##1{\textcolor[rgb]{0.73,0.40,0.13}{##1}}}
\expandafter\def\csname PY@tok@sr\endcsname{\def\PY@tc##1{\textcolor[rgb]{0.73,0.40,0.53}{##1}}}
\expandafter\def\csname PY@tok@ss\endcsname{\def\PY@tc##1{\textcolor[rgb]{0.10,0.09,0.49}{##1}}}
\expandafter\def\csname PY@tok@sx\endcsname{\def\PY@tc##1{\textcolor[rgb]{0.00,0.50,0.00}{##1}}}
\expandafter\def\csname PY@tok@m\endcsname{\def\PY@tc##1{\textcolor[rgb]{0.40,0.40,0.40}{##1}}}
\expandafter\def\csname PY@tok@gh\endcsname{\let\PY@bf=\textbf\def\PY@tc##1{\textcolor[rgb]{0.00,0.00,0.50}{##1}}}
\expandafter\def\csname PY@tok@gu\endcsname{\let\PY@bf=\textbf\def\PY@tc##1{\textcolor[rgb]{0.50,0.00,0.50}{##1}}}
\expandafter\def\csname PY@tok@gd\endcsname{\def\PY@tc##1{\textcolor[rgb]{0.63,0.00,0.00}{##1}}}
\expandafter\def\csname PY@tok@gi\endcsname{\def\PY@tc##1{\textcolor[rgb]{0.00,0.63,0.00}{##1}}}
\expandafter\def\csname PY@tok@gr\endcsname{\def\PY@tc##1{\textcolor[rgb]{1.00,0.00,0.00}{##1}}}
\expandafter\def\csname PY@tok@ge\endcsname{\let\PY@it=\textit}
\expandafter\def\csname PY@tok@gs\endcsname{\let\PY@bf=\textbf}
\expandafter\def\csname PY@tok@gp\endcsname{\let\PY@bf=\textbf\def\PY@tc##1{\textcolor[rgb]{0.00,0.00,0.50}{##1}}}
\expandafter\def\csname PY@tok@go\endcsname{\def\PY@tc##1{\textcolor[rgb]{0.53,0.53,0.53}{##1}}}
\expandafter\def\csname PY@tok@gt\endcsname{\def\PY@tc##1{\textcolor[rgb]{0.00,0.27,0.87}{##1}}}
\expandafter\def\csname PY@tok@err\endcsname{\def\PY@bc##1{\setlength{\fboxsep}{0pt}\fcolorbox[rgb]{1.00,0.00,0.00}{1,1,1}{\strut ##1}}}
\expandafter\def\csname PY@tok@kc\endcsname{\let\PY@bf=\textbf\def\PY@tc##1{\textcolor[rgb]{0.00,0.50,0.00}{##1}}}
\expandafter\def\csname PY@tok@kd\endcsname{\let\PY@bf=\textbf\def\PY@tc##1{\textcolor[rgb]{0.00,0.50,0.00}{##1}}}
\expandafter\def\csname PY@tok@kn\endcsname{\let\PY@bf=\textbf\def\PY@tc##1{\textcolor[rgb]{0.00,0.50,0.00}{##1}}}
\expandafter\def\csname PY@tok@kr\endcsname{\let\PY@bf=\textbf\def\PY@tc##1{\textcolor[rgb]{0.00,0.50,0.00}{##1}}}
\expandafter\def\csname PY@tok@bp\endcsname{\def\PY@tc##1{\textcolor[rgb]{0.00,0.50,0.00}{##1}}}
\expandafter\def\csname PY@tok@fm\endcsname{\def\PY@tc##1{\textcolor[rgb]{0.00,0.00,1.00}{##1}}}
\expandafter\def\csname PY@tok@vc\endcsname{\def\PY@tc##1{\textcolor[rgb]{0.10,0.09,0.49}{##1}}}
\expandafter\def\csname PY@tok@vg\endcsname{\def\PY@tc##1{\textcolor[rgb]{0.10,0.09,0.49}{##1}}}
\expandafter\def\csname PY@tok@vi\endcsname{\def\PY@tc##1{\textcolor[rgb]{0.10,0.09,0.49}{##1}}}
\expandafter\def\csname PY@tok@vm\endcsname{\def\PY@tc##1{\textcolor[rgb]{0.10,0.09,0.49}{##1}}}
\expandafter\def\csname PY@tok@sa\endcsname{\def\PY@tc##1{\textcolor[rgb]{0.73,0.13,0.13}{##1}}}
\expandafter\def\csname PY@tok@sb\endcsname{\def\PY@tc##1{\textcolor[rgb]{0.73,0.13,0.13}{##1}}}
\expandafter\def\csname PY@tok@sc\endcsname{\def\PY@tc##1{\textcolor[rgb]{0.73,0.13,0.13}{##1}}}
\expandafter\def\csname PY@tok@dl\endcsname{\def\PY@tc##1{\textcolor[rgb]{0.73,0.13,0.13}{##1}}}
\expandafter\def\csname PY@tok@s2\endcsname{\def\PY@tc##1{\textcolor[rgb]{0.73,0.13,0.13}{##1}}}
\expandafter\def\csname PY@tok@sh\endcsname{\def\PY@tc##1{\textcolor[rgb]{0.73,0.13,0.13}{##1}}}
\expandafter\def\csname PY@tok@s1\endcsname{\def\PY@tc##1{\textcolor[rgb]{0.73,0.13,0.13}{##1}}}
\expandafter\def\csname PY@tok@mb\endcsname{\def\PY@tc##1{\textcolor[rgb]{0.40,0.40,0.40}{##1}}}
\expandafter\def\csname PY@tok@mf\endcsname{\def\PY@tc##1{\textcolor[rgb]{0.40,0.40,0.40}{##1}}}
\expandafter\def\csname PY@tok@mh\endcsname{\def\PY@tc##1{\textcolor[rgb]{0.40,0.40,0.40}{##1}}}
\expandafter\def\csname PY@tok@mi\endcsname{\def\PY@tc##1{\textcolor[rgb]{0.40,0.40,0.40}{##1}}}
\expandafter\def\csname PY@tok@il\endcsname{\def\PY@tc##1{\textcolor[rgb]{0.40,0.40,0.40}{##1}}}
\expandafter\def\csname PY@tok@mo\endcsname{\def\PY@tc##1{\textcolor[rgb]{0.40,0.40,0.40}{##1}}}
\expandafter\def\csname PY@tok@ch\endcsname{\let\PY@it=\textit\def\PY@tc##1{\textcolor[rgb]{0.25,0.50,0.50}{##1}}}
\expandafter\def\csname PY@tok@cm\endcsname{\let\PY@it=\textit\def\PY@tc##1{\textcolor[rgb]{0.25,0.50,0.50}{##1}}}
\expandafter\def\csname PY@tok@cpf\endcsname{\let\PY@it=\textit\def\PY@tc##1{\textcolor[rgb]{0.25,0.50,0.50}{##1}}}
\expandafter\def\csname PY@tok@c1\endcsname{\let\PY@it=\textit\def\PY@tc##1{\textcolor[rgb]{0.25,0.50,0.50}{##1}}}
\expandafter\def\csname PY@tok@cs\endcsname{\let\PY@it=\textit\def\PY@tc##1{\textcolor[rgb]{0.25,0.50,0.50}{##1}}}

\def\PYZbs{\char`\\}
\def\PYZus{\char`\_}
\def\PYZob{\char`\{}
\def\PYZcb{\char`\}}
\def\PYZca{\char`\^}
\def\PYZam{\char`\&}
\def\PYZlt{\char`\<}
\def\PYZgt{\char`\>}
\def\PYZsh{\char`\#}
\def\PYZpc{\char`\%}
\def\PYZdl{\char`\$}
\def\PYZhy{\char`\-}
\def\PYZsq{\char`\'}
\def\PYZdq{\char`\"}
\def\PYZti{\char`\~}
% for compatibility with earlier versions
\def\PYZat{@}
\def\PYZlb{[}
\def\PYZrb{]}
\makeatother


    % For linebreaks inside Verbatim environment from package fancyvrb. 
    \makeatletter
        \newbox\Wrappedcontinuationbox 
        \newbox\Wrappedvisiblespacebox 
        \newcommand*\Wrappedvisiblespace {\textcolor{red}{\textvisiblespace}} 
        \newcommand*\Wrappedcontinuationsymbol {\textcolor{red}{\llap{\tiny$\m@th\hookrightarrow$}}} 
        \newcommand*\Wrappedcontinuationindent {3ex } 
        \newcommand*\Wrappedafterbreak {\kern\Wrappedcontinuationindent\copy\Wrappedcontinuationbox} 
        % Take advantage of the already applied Pygments mark-up to insert 
        % potential linebreaks for TeX processing. 
        %        {, <, #, %, $, ' and ": go to next line. 
        %        _, }, ^, &, >, - and ~: stay at end of broken line. 
        % Use of \textquotesingle for straight quote. 
        \newcommand*\Wrappedbreaksatspecials {% 
            \def\PYGZus{\discretionary{\char`\_}{\Wrappedafterbreak}{\char`\_}}% 
            \def\PYGZob{\discretionary{}{\Wrappedafterbreak\char`\{}{\char`\{}}% 
            \def\PYGZcb{\discretionary{\char`\}}{\Wrappedafterbreak}{\char`\}}}% 
            \def\PYGZca{\discretionary{\char`\^}{\Wrappedafterbreak}{\char`\^}}% 
            \def\PYGZam{\discretionary{\char`\&}{\Wrappedafterbreak}{\char`\&}}% 
            \def\PYGZlt{\discretionary{}{\Wrappedafterbreak\char`\<}{\char`\<}}% 
            \def\PYGZgt{\discretionary{\char`\>}{\Wrappedafterbreak}{\char`\>}}% 
            \def\PYGZsh{\discretionary{}{\Wrappedafterbreak\char`\#}{\char`\#}}% 
            \def\PYGZpc{\discretionary{}{\Wrappedafterbreak\char`\%}{\char`\%}}% 
            \def\PYGZdl{\discretionary{}{\Wrappedafterbreak\char`\$}{\char`\$}}% 
            \def\PYGZhy{\discretionary{\char`\-}{\Wrappedafterbreak}{\char`\-}}% 
            \def\PYGZsq{\discretionary{}{\Wrappedafterbreak\textquotesingle}{\textquotesingle}}% 
            \def\PYGZdq{\discretionary{}{\Wrappedafterbreak\char`\"}{\char`\"}}% 
            \def\PYGZti{\discretionary{\char`\~}{\Wrappedafterbreak}{\char`\~}}% 
        } 
        % Some characters . , ; ? ! / are not pygmentized. 
        % This macro makes them "active" and they will insert potential linebreaks 
        \newcommand*\Wrappedbreaksatpunct {% 
            \lccode`\~`\.\lowercase{\def~}{\discretionary{\hbox{\char`\.}}{\Wrappedafterbreak}{\hbox{\char`\.}}}% 
            \lccode`\~`\,\lowercase{\def~}{\discretionary{\hbox{\char`\,}}{\Wrappedafterbreak}{\hbox{\char`\,}}}% 
            \lccode`\~`\;\lowercase{\def~}{\discretionary{\hbox{\char`\;}}{\Wrappedafterbreak}{\hbox{\char`\;}}}% 
            \lccode`\~`\:\lowercase{\def~}{\discretionary{\hbox{\char`\:}}{\Wrappedafterbreak}{\hbox{\char`\:}}}% 
            \lccode`\~`\?\lowercase{\def~}{\discretionary{\hbox{\char`\?}}{\Wrappedafterbreak}{\hbox{\char`\?}}}% 
            \lccode`\~`\!\lowercase{\def~}{\discretionary{\hbox{\char`\!}}{\Wrappedafterbreak}{\hbox{\char`\!}}}% 
            \lccode`\~`\/\lowercase{\def~}{\discretionary{\hbox{\char`\/}}{\Wrappedafterbreak}{\hbox{\char`\/}}}% 
            \catcode`\.\active
            \catcode`\,\active 
            \catcode`\;\active
            \catcode`\:\active
            \catcode`\?\active
            \catcode`\!\active
            \catcode`\/\active 
            \lccode`\~`\~ 	
        }
    \makeatother

    \let\OriginalVerbatim=\Verbatim
    \makeatletter
    \renewcommand{\Verbatim}[1][1]{%
        %\parskip\z@skip
        \sbox\Wrappedcontinuationbox {\Wrappedcontinuationsymbol}%
        \sbox\Wrappedvisiblespacebox {\FV@SetupFont\Wrappedvisiblespace}%
        \def\FancyVerbFormatLine ##1{\hsize\linewidth
            \vtop{\raggedright\hyphenpenalty\z@\exhyphenpenalty\z@
                \doublehyphendemerits\z@\finalhyphendemerits\z@
                \strut ##1\strut}%
        }%
        % If the linebreak is at a space, the latter will be displayed as visible
        % space at end of first line, and a continuation symbol starts next line.
        % Stretch/shrink are however usually zero for typewriter font.
        \def\FV@Space {%
            \nobreak\hskip\z@ plus\fontdimen3\font minus\fontdimen4\font
            \discretionary{\copy\Wrappedvisiblespacebox}{\Wrappedafterbreak}
            {\kern\fontdimen2\font}%
        }%
        
        % Allow breaks at special characters using \PYG... macros.
        \Wrappedbreaksatspecials
        % Breaks at punctuation characters . , ; ? ! and / need catcode=\active 	
        \OriginalVerbatim[#1,codes*=\Wrappedbreaksatpunct]%
    }
    \makeatother

    % Exact colors from NB
    \definecolor{incolor}{HTML}{303F9F}
    \definecolor{outcolor}{HTML}{D84315}
    \definecolor{cellborder}{HTML}{CFCFCF}
    \definecolor{cellbackground}{HTML}{F7F7F7}
    
    % prompt
    \makeatletter
    \newcommand{\boxspacing}{\kern\kvtcb@left@rule\kern\kvtcb@boxsep}
    \makeatother
    \newcommand{\prompt}[4]{
        {\ttfamily\llap{{\color{#2}[#3]:\hspace{3pt}#4}}\vspace{-\baselineskip}}
    }
    

    
    % Prevent overflowing lines due to hard-to-break entities
    \sloppy 
    % Setup hyperref package
    \hypersetup{
      breaklinks=true,  % so long urls are correctly broken across lines
      colorlinks=true,
      urlcolor=urlcolor,
      linkcolor=linkcolor,
      citecolor=citecolor,
      }
    % Slightly bigger margins than the latex defaults
    
    \geometry{verbose,tmargin=1in,bmargin=1in,lmargin=1in,rmargin=1in}
    
    

\begin{document}
    
    \maketitle
    
    

    
    \hypertarget{approximate-solvers-for-the-euler-equations-of-gas-dynamics}{%
\section{Approximate solvers for the Euler equations of gas
dynamics}\label{approximate-solvers-for-the-euler-equations-of-gas-dynamics}}

    In this chapter we discuss approximate solvers for the one-dimensional
Euler equations:

\begin{align}
    \rho_t + (\rho u)_x & = 0 \\
    (\rho u)_t + (\rho u^2 + p)_x & = 0 \\
    E_t + ((E+p)u)_x & = 0.
\end{align}

As in \href{Euler.ipynb}{Euler}, we focus on the case of an ideal gas,
for which the total energy is given by

\begin{align} \label{EA:EOS}
    E = \frac{p}{\gamma-1} + \frac{1}{2}\rho u^2.
\end{align}

    To examine the Python code for this chapter, and for the exact Riemann
solution, see:

\begin{itemize}
\tightlist
\item
  \url{exact_solvers/euler.py} \ldots{}
  \href{https://github.com/clawpack/riemann_book/blob/FA16/exact_solvers/euler.py}{on
  github.}
\end{itemize}

    \hypertarget{roe-solver}{%
\subsection{Roe solver}\label{roe-solver}}

    We first derive a Roe solver for the Euler equations, following the same
approach as in
\href{Shallow_water_approximate.ipynb}{Shallow\_water\_approximate}.
Namely, we assume that \(\hat{A} = f'(\hat{q})\) for some average state
\(\hat{q}\), and impose the condition of conservation:

\begin{align} \label{EA:cons}
    f'(\hat{q}) (q_r - q_\ell) & = f(q_r) - f(q_\ell).
\end{align}

We will need the following quantities:

\begin{align}
q & = \begin{pmatrix} \rho \\ \rho u \\ E \end{pmatrix}, \ \ \ \ \ \  f(q) = \begin{pmatrix} \rho u \\ \rho u^2 + p \\ H u \rho \end{pmatrix}, \\
f'(\hat{q}) & = \begin{pmatrix} 
                0 & 1 & 0 \\ 
                \frac{\gamma-3}{2}\hat{u}^2 & (3-\gamma)\hat{u} & \gamma-1 \\
                \frac{\gamma-1}{2}\hat{u}^3 - \hat{u}\hat{H} & \hat{H} - (\gamma-1)\hat{u}^2 & \gamma \hat{u} \end{pmatrix}.
\end{align}

Here \(H = \frac{E+p}{\rho}\) is the enthalpy. We have rewritten most
expressions involving \(E\) in terms of \(H\) because it simplifies the
derivation that follows. We now solve (\ref{EA:cons}) to find
\(\hat{u}\) and \(\hat{H}\). It turns out that, for the case of a
polytropic ideal gas, the average density \(\hat{\rho}\) plays no role
in the Roe solver.

The first equation of (\ref{EA:cons}) is an identity, satisfied
independently of our choice of \(\hat{q}\). The second equation is
(using (\ref{EA:EOS}))

\begin{align}
    \frac{\gamma-3}{2}\hat{u}^2 (\rho_r - \rho_\ell) + (3-\gamma)\hat{u}(\rho_r u_r - \rho_\ell u_\ell) \\ + (\gamma-1)\left( \frac{p_r-p_\ell}{\gamma-1} + \frac{1}{2}(\rho_r u_r^2 - \rho_\ell u_\ell^2) \right) & = \rho_r u_r^2 - \rho_\ell u_\ell^2 + p_r - p_\ell,
\end{align}

which simplifies to a quadratic equation for \(\hat{u}\):

\begin{align} \label{EA:u_quadratic}
    (\rho_r - \rho_\ell)\hat{u}^2 - 2(\rho_r u_r - \rho_\ell u_\ell) \hat{u} + (\rho_r u_r^2 - \rho_\ell u_\ell^2) & = 0,
\end{align}

with roots

\begin{align}
    \hat{u}_\pm & = \frac{\rho_r u_r - \rho_\ell u_\ell \mp \sqrt{\rho_r \rho_\ell} (u_\ell - u_r)}{\rho_r - \rho_\ell} = \frac{\sqrt{\rho_r} u_r \pm \sqrt{\rho_\ell} u_\ell}{\sqrt{\rho_r}\pm\sqrt{\rho_\ell}}
\end{align}

Notice that this is identical to the Roe average of the velocity for the
shallow water equations, if we replace the density \(\rho\) with depth
\(h\). As before, we choose the root \(u_+\) since it is well defined
for all values of \(\rho_r, \rho_\ell\).

    Next we find \(\hat{H}\) by solving the last equation of
(\ref{EA:cons}), which reads\\
\begin{align}
    \left( \frac{\gamma-1}{2}\hat{u}^3 - \hat{u}\hat{H} \right)(\rho_r - \rho_\ell) \\ + \left( \hat{H} - (\gamma-1)\hat{u}^2 \right)(\rho_r u_r - \rho_\ell u_\ell) + \gamma \hat{u}(E_r - E_\ell) & = H_r u_r \rho_r - H_\ell u_\ell \rho_\ell.
\end{align}\\
We can simplify this using the equality
\(\gamma E = \rho H + \frac{\gamma-1}{2}\rho u^2\) and solve for
\(\hat{H}\) to find\\
\begin{align}
    \hat{H}_{\pm} & = \frac{\rho_r H_r (u_r - \hat{u}_+) - \rho_\ell H_\ell (u_\ell - \hat{u}_+)}{\rho_r u_r - \rho_\ell u_\ell - \hat{u}_\pm(\rho_r -\rho_\ell)} \\
    & = \frac{\rho_r H_r (u_r - \hat{u}_+) - \rho_\ell H_\ell (u_\ell - \hat{u}_+)}{\pm\sqrt{\rho_r \rho_\ell}(u_r-u_\ell)} \\
    & = \frac{\rho_r H_r - \rho_\ell H_\ell \mp\sqrt{\rho_r \rho_\ell}(H_r - H_\ell)}{\rho_r - \rho_\ell} \\
    & = \frac{\sqrt{\rho_r}H_r \pm \sqrt{\rho_\ell} H_\ell}{\sqrt{\rho_r}\pm\sqrt{\rho_\ell}}.
\end{align}\\
Once more, we take the plus sign in the final expression for
\(\hat{H}\), giving the Roe averages \[
\hat{u} = \frac{\sqrt{\rho_r} u_r + \sqrt{\rho_\ell} u_\ell}{\sqrt{\rho_r} + \sqrt{\rho_\ell}},
\qquad \hat{H} = \frac{\sqrt{\rho_r}H_r + \sqrt{\rho_\ell} H_\ell}{\sqrt{\rho_r} + \sqrt{\rho_\ell}}.
\]

    To implement the Roe solver, we also need the eigenvalues and
eigenvectors of the averaged flux Jacobian \(f'(\hat{q})\). These are
just the eigenvalues of the true Jacobian, evaluated at the averaged
state: \begin{align}
    \lambda_1 & = \hat{u} - \hat{c}, & \lambda_2 & = \hat{u} & \lambda_3 & = \hat{u} + \hat{c},
\end{align}\\
\begin{align}
r_1 & = \begin{bmatrix} 1 \\ \hat{u}-\hat{c} \\ \hat{H}-\hat{u}\hat{c}\end{bmatrix} &
r_2 & = \begin{bmatrix} 1 \\ \hat{u} \\ \frac{1}{2}\hat{u}^2 \end{bmatrix} &
r_3 & = \begin{bmatrix} 1 \\ \hat{u}+\hat{c} \\ \hat{H}+\hat{u}\hat{c}\end{bmatrix}.
\end{align} Here \(\hat{c} = \sqrt{(\gamma-1)(\hat{H}-\hat{u}^2/2)}\).

Solving the system of equations \begin{align}
q_r - q_\ell & = \sum_{p=1}^3 {\mathcal W}_p = \sum_{p=1}^3 \alpha_p r_p
\end{align}\\
for the wave strengths gives \begin{align}
    \alpha_2 & = \delta_1 + (\gamma-1)\frac{\hat{u}\delta_2 - \delta_3}{\hat{c}^2} \\
    \alpha_3 & = \frac{\delta_2 + (\hat{c}-\hat{u})\delta_1 - \hat{c}\alpha_2}{2\hat{c}} \\
    \alpha_1 & = \delta_1 - \alpha_2 - \alpha_3,
\end{align}\\
where \(\delta = q_r - q_\ell\). We now have everything we need to
implement the Roe solver.

    \begin{tcolorbox}[breakable, size=fbox, boxrule=1pt, pad at break*=1mm,colback=cellbackground, colframe=cellborder]
\prompt{In}{incolor}{1}{\boxspacing}
\begin{Verbatim}[commandchars=\\\{\}]
\PY{o}{\PYZpc{}}\PY{k}{matplotlib} inline
\end{Verbatim}
\end{tcolorbox}

    \begin{tcolorbox}[breakable, size=fbox, boxrule=1pt, pad at break*=1mm,colback=cellbackground, colframe=cellborder]
\prompt{In}{incolor}{2}{\boxspacing}
\begin{Verbatim}[commandchars=\\\{\}]
\PY{o}{\PYZpc{}}\PY{k}{config} InlineBackend.figure\PYZus{}format = \PYZsq{}svg\PYZsq{}
\PY{k+kn}{import} \PY{n+nn}{numpy} \PY{k}{as} \PY{n+nn}{np}
\PY{k+kn}{from} \PY{n+nn}{exact\PYZus{}solvers} \PY{k+kn}{import} \PY{n}{euler}
\PY{k+kn}{from} \PY{n+nn}{utils} \PY{k+kn}{import} \PY{n}{riemann\PYZus{}tools} \PY{k}{as} \PY{n}{rt}
\PY{k+kn}{from} \PY{n+nn}{ipywidgets} \PY{k+kn}{import} \PY{n}{interact}
\PY{k+kn}{from} \PY{n+nn}{ipywidgets} \PY{k+kn}{import} \PY{n}{widgets}
\PY{n}{State} \PY{o}{=} \PY{n}{euler}\PY{o}{.}\PY{n}{Primitive\PYZus{}State}
\end{Verbatim}
\end{tcolorbox}

    \begin{tcolorbox}[breakable, size=fbox, boxrule=1pt, pad at break*=1mm,colback=cellbackground, colframe=cellborder]
\prompt{In}{incolor}{3}{\boxspacing}
\begin{Verbatim}[commandchars=\\\{\}]
\PY{k}{def} \PY{n+nf}{roe\PYZus{}averages}\PY{p}{(}\PY{n}{q\PYZus{}l}\PY{p}{,} \PY{n}{q\PYZus{}r}\PY{p}{,} \PY{n}{gamma}\PY{o}{=}\PY{l+m+mf}{1.4}\PY{p}{)}\PY{p}{:}
    \PY{n}{rho\PYZus{}sqrt\PYZus{}l} \PY{o}{=} \PY{n}{np}\PY{o}{.}\PY{n}{sqrt}\PY{p}{(}\PY{n}{q\PYZus{}l}\PY{p}{[}\PY{l+m+mi}{0}\PY{p}{]}\PY{p}{)}
    \PY{n}{rho\PYZus{}sqrt\PYZus{}r} \PY{o}{=} \PY{n}{np}\PY{o}{.}\PY{n}{sqrt}\PY{p}{(}\PY{n}{q\PYZus{}r}\PY{p}{[}\PY{l+m+mi}{0}\PY{p}{]}\PY{p}{)}
    \PY{n}{p\PYZus{}l} \PY{o}{=} \PY{p}{(}\PY{n}{gamma}\PY{o}{\PYZhy{}}\PY{l+m+mf}{1.}\PY{p}{)}\PY{o}{*}\PY{p}{(}\PY{n}{q\PYZus{}l}\PY{p}{[}\PY{l+m+mi}{2}\PY{p}{]}\PY{o}{\PYZhy{}}\PY{l+m+mf}{0.5}\PY{o}{*}\PY{p}{(}\PY{n}{q\PYZus{}l}\PY{p}{[}\PY{l+m+mi}{1}\PY{p}{]}\PY{o}{*}\PY{o}{*}\PY{l+m+mi}{2}\PY{p}{)}\PY{o}{/}\PY{n}{q\PYZus{}l}\PY{p}{[}\PY{l+m+mi}{0}\PY{p}{]}\PY{p}{)}
    \PY{n}{p\PYZus{}r} \PY{o}{=} \PY{p}{(}\PY{n}{gamma}\PY{o}{\PYZhy{}}\PY{l+m+mf}{1.}\PY{p}{)}\PY{o}{*}\PY{p}{(}\PY{n}{q\PYZus{}r}\PY{p}{[}\PY{l+m+mi}{2}\PY{p}{]}\PY{o}{\PYZhy{}}\PY{l+m+mf}{0.5}\PY{o}{*}\PY{p}{(}\PY{n}{q\PYZus{}r}\PY{p}{[}\PY{l+m+mi}{1}\PY{p}{]}\PY{o}{*}\PY{o}{*}\PY{l+m+mi}{2}\PY{p}{)}\PY{o}{/}\PY{n}{q\PYZus{}r}\PY{p}{[}\PY{l+m+mi}{0}\PY{p}{]}\PY{p}{)}
    \PY{n}{denom} \PY{o}{=} \PY{n}{rho\PYZus{}sqrt\PYZus{}l} \PY{o}{+} \PY{n}{rho\PYZus{}sqrt\PYZus{}r}
    \PY{n}{u\PYZus{}hat} \PY{o}{=} \PY{p}{(}\PY{n}{q\PYZus{}l}\PY{p}{[}\PY{l+m+mi}{1}\PY{p}{]}\PY{o}{/}\PY{n}{rho\PYZus{}sqrt\PYZus{}l} \PY{o}{+} \PY{n}{q\PYZus{}r}\PY{p}{[}\PY{l+m+mi}{1}\PY{p}{]}\PY{o}{/}\PY{n}{rho\PYZus{}sqrt\PYZus{}r}\PY{p}{)}\PY{o}{/}\PY{n}{denom}
    \PY{n}{H\PYZus{}hat} \PY{o}{=} \PY{p}{(}\PY{p}{(}\PY{n}{q\PYZus{}l}\PY{p}{[}\PY{l+m+mi}{2}\PY{p}{]}\PY{o}{+}\PY{n}{p\PYZus{}l}\PY{p}{)}\PY{o}{/}\PY{n}{rho\PYZus{}sqrt\PYZus{}l} \PY{o}{+} \PY{p}{(}\PY{n}{q\PYZus{}r}\PY{p}{[}\PY{l+m+mi}{2}\PY{p}{]}\PY{o}{+}\PY{n}{p\PYZus{}r}\PY{p}{)}\PY{o}{/}\PY{n}{rho\PYZus{}sqrt\PYZus{}r}\PY{p}{)}\PY{o}{/}\PY{n}{denom}
    \PY{n}{c\PYZus{}hat} \PY{o}{=} \PY{n}{np}\PY{o}{.}\PY{n}{sqrt}\PY{p}{(}\PY{p}{(}\PY{n}{gamma}\PY{o}{\PYZhy{}}\PY{l+m+mi}{1}\PY{p}{)}\PY{o}{*}\PY{p}{(}\PY{n}{H\PYZus{}hat}\PY{o}{\PYZhy{}}\PY{l+m+mf}{0.5}\PY{o}{*}\PY{n}{u\PYZus{}hat}\PY{o}{*}\PY{o}{*}\PY{l+m+mi}{2}\PY{p}{)}\PY{p}{)}
    
    \PY{k}{return} \PY{n}{u\PYZus{}hat}\PY{p}{,} \PY{n}{c\PYZus{}hat}\PY{p}{,} \PY{n}{H\PYZus{}hat}
    
    
\PY{k}{def} \PY{n+nf}{Euler\PYZus{}roe}\PY{p}{(}\PY{n}{q\PYZus{}l}\PY{p}{,} \PY{n}{q\PYZus{}r}\PY{p}{,} \PY{n}{gamma}\PY{o}{=}\PY{l+m+mf}{1.4}\PY{p}{)}\PY{p}{:}
    \PY{l+s+sd}{\PYZdq{}\PYZdq{}\PYZdq{}}
\PY{l+s+sd}{    Approximate Roe solver for the Euler equations.}
\PY{l+s+sd}{    \PYZdq{}\PYZdq{}\PYZdq{}}
    
    \PY{n}{rho\PYZus{}l} \PY{o}{=} \PY{n}{q\PYZus{}l}\PY{p}{[}\PY{l+m+mi}{0}\PY{p}{]}
    \PY{n}{rhou\PYZus{}l} \PY{o}{=} \PY{n}{q\PYZus{}l}\PY{p}{[}\PY{l+m+mi}{1}\PY{p}{]}
    \PY{n}{u\PYZus{}l} \PY{o}{=} \PY{n}{rhou\PYZus{}l}\PY{o}{/}\PY{n}{rho\PYZus{}l}
    \PY{n}{rho\PYZus{}r} \PY{o}{=} \PY{n}{q\PYZus{}r}\PY{p}{[}\PY{l+m+mi}{0}\PY{p}{]}
    \PY{n}{rhou\PYZus{}r} \PY{o}{=} \PY{n}{q\PYZus{}r}\PY{p}{[}\PY{l+m+mi}{1}\PY{p}{]}
    \PY{n}{u\PYZus{}r} \PY{o}{=} \PY{n}{rhou\PYZus{}r}\PY{o}{/}\PY{n}{rho\PYZus{}r}
    
    \PY{n}{u\PYZus{}hat}\PY{p}{,} \PY{n}{c\PYZus{}hat}\PY{p}{,} \PY{n}{H\PYZus{}hat} \PY{o}{=} \PY{n}{roe\PYZus{}averages}\PY{p}{(}\PY{n}{q\PYZus{}l}\PY{p}{,} \PY{n}{q\PYZus{}r}\PY{p}{,} \PY{n}{gamma}\PY{p}{)}
    
    \PY{n}{dq} \PY{o}{=} \PY{n}{q\PYZus{}r} \PY{o}{\PYZhy{}} \PY{n}{q\PYZus{}l}
    
    \PY{n}{s1} \PY{o}{=} \PY{n}{u\PYZus{}hat} \PY{o}{\PYZhy{}} \PY{n}{c\PYZus{}hat}
    \PY{n}{s2} \PY{o}{=} \PY{n}{u\PYZus{}hat}
    \PY{n}{s3} \PY{o}{=} \PY{n}{u\PYZus{}hat} \PY{o}{+} \PY{n}{c\PYZus{}hat}
    
    \PY{n}{alpha2} \PY{o}{=} \PY{p}{(}\PY{n}{gamma}\PY{o}{\PYZhy{}}\PY{l+m+mf}{1.}\PY{p}{)}\PY{o}{/}\PY{n}{c\PYZus{}hat}\PY{o}{*}\PY{o}{*}\PY{l+m+mi}{2} \PY{o}{*}\PY{p}{(}\PY{p}{(}\PY{n}{H\PYZus{}hat}\PY{o}{\PYZhy{}}\PY{n}{u\PYZus{}hat}\PY{o}{*}\PY{o}{*}\PY{l+m+mi}{2}\PY{p}{)}\PY{o}{*}\PY{n}{dq}\PY{p}{[}\PY{l+m+mi}{0}\PY{p}{]}\PY{o}{+}\PY{n}{u\PYZus{}hat}\PY{o}{*}\PY{n}{dq}\PY{p}{[}\PY{l+m+mi}{1}\PY{p}{]}\PY{o}{\PYZhy{}}\PY{n}{dq}\PY{p}{[}\PY{l+m+mi}{2}\PY{p}{]}\PY{p}{)}
    \PY{n}{alpha3} \PY{o}{=} \PY{p}{(}\PY{n}{dq}\PY{p}{[}\PY{l+m+mi}{1}\PY{p}{]} \PY{o}{+} \PY{p}{(}\PY{n}{c\PYZus{}hat} \PY{o}{\PYZhy{}} \PY{n}{u\PYZus{}hat}\PY{p}{)}\PY{o}{*}\PY{n}{dq}\PY{p}{[}\PY{l+m+mi}{0}\PY{p}{]} \PY{o}{\PYZhy{}} \PY{n}{c\PYZus{}hat}\PY{o}{*}\PY{n}{alpha2}\PY{p}{)} \PY{o}{/} \PY{p}{(}\PY{l+m+mf}{2.}\PY{o}{*}\PY{n}{c\PYZus{}hat}\PY{p}{)}
    \PY{n}{alpha1} \PY{o}{=} \PY{n}{dq}\PY{p}{[}\PY{l+m+mi}{0}\PY{p}{]} \PY{o}{\PYZhy{}} \PY{n}{alpha2} \PY{o}{\PYZhy{}} \PY{n}{alpha3}
    
    \PY{n}{r1} \PY{o}{=} \PY{n}{np}\PY{o}{.}\PY{n}{array}\PY{p}{(}\PY{p}{[}\PY{l+m+mf}{1.}\PY{p}{,} \PY{n}{u\PYZus{}hat}\PY{o}{\PYZhy{}}\PY{n}{c\PYZus{}hat}\PY{p}{,} \PY{n}{H\PYZus{}hat} \PY{o}{\PYZhy{}} \PY{n}{u\PYZus{}hat}\PY{o}{*}\PY{n}{c\PYZus{}hat}\PY{p}{]}\PY{p}{)}
    \PY{n}{r2} \PY{o}{=} \PY{n}{np}\PY{o}{.}\PY{n}{array}\PY{p}{(}\PY{p}{[}\PY{l+m+mf}{1.}\PY{p}{,} \PY{n}{u\PYZus{}hat}\PY{p}{,} \PY{l+m+mf}{0.5}\PY{o}{*}\PY{n}{u\PYZus{}hat}\PY{o}{*}\PY{o}{*}\PY{l+m+mi}{2}\PY{p}{]}\PY{p}{)}
    \PY{n}{q\PYZus{}l\PYZus{}star} \PY{o}{=} \PY{n}{q\PYZus{}l} \PY{o}{+} \PY{n}{alpha1}\PY{o}{*}\PY{n}{r1}
    \PY{n}{q\PYZus{}r\PYZus{}star} \PY{o}{=} \PY{n}{q\PYZus{}l\PYZus{}star} \PY{o}{+} \PY{n}{alpha2}\PY{o}{*}\PY{n}{r2}
    
    \PY{n}{states} \PY{o}{=} \PY{n}{np}\PY{o}{.}\PY{n}{column\PYZus{}stack}\PY{p}{(}\PY{p}{[}\PY{n}{q\PYZus{}l}\PY{p}{,}\PY{n}{q\PYZus{}l\PYZus{}star}\PY{p}{,}\PY{n}{q\PYZus{}r\PYZus{}star}\PY{p}{,}\PY{n}{q\PYZus{}r}\PY{p}{]}\PY{p}{)}
    \PY{n}{speeds} \PY{o}{=} \PY{p}{[}\PY{n}{s1}\PY{p}{,} \PY{n}{s2}\PY{p}{,} \PY{n}{s3}\PY{p}{]}
    \PY{n}{wave\PYZus{}types} \PY{o}{=} \PY{p}{[}\PY{l+s+s1}{\PYZsq{}}\PY{l+s+s1}{contact}\PY{l+s+s1}{\PYZsq{}}\PY{p}{,}\PY{l+s+s1}{\PYZsq{}}\PY{l+s+s1}{contact}\PY{l+s+s1}{\PYZsq{}}\PY{p}{,} \PY{l+s+s1}{\PYZsq{}}\PY{l+s+s1}{contact}\PY{l+s+s1}{\PYZsq{}}\PY{p}{]}
    
    \PY{k}{def} \PY{n+nf}{reval}\PY{p}{(}\PY{n}{xi}\PY{p}{)}\PY{p}{:}
        \PY{n}{rho} \PY{o}{=} \PY{p}{(}\PY{n}{xi}\PY{o}{\PYZlt{}}\PY{n}{s1}\PY{p}{)}\PY{o}{*}\PY{n}{states}\PY{p}{[}\PY{l+m+mi}{0}\PY{p}{,}\PY{l+m+mi}{0}\PY{p}{]} \PY{o}{+} \PY{p}{(}\PY{n}{s1}\PY{o}{\PYZlt{}}\PY{o}{=}\PY{n}{xi}\PY{p}{)}\PY{o}{*}\PY{p}{(}\PY{n}{xi}\PY{o}{\PYZlt{}}\PY{n}{s2}\PY{p}{)}\PY{o}{*}\PY{n}{states}\PY{p}{[}\PY{l+m+mi}{0}\PY{p}{,}\PY{l+m+mi}{1}\PY{p}{]} \PY{o}{+} \PYZbs{}
              \PY{p}{(}\PY{n}{s2}\PY{o}{\PYZlt{}}\PY{o}{=}\PY{n}{xi}\PY{p}{)}\PY{o}{*}\PY{p}{(}\PY{n}{xi}\PY{o}{\PYZlt{}}\PY{n}{s3}\PY{p}{)}\PY{o}{*}\PY{n}{states}\PY{p}{[}\PY{l+m+mi}{0}\PY{p}{,}\PY{l+m+mi}{2}\PY{p}{]} \PY{o}{+} \PY{p}{(}\PY{n}{s3}\PY{o}{\PYZlt{}}\PY{o}{=}\PY{n}{xi}\PY{p}{)}\PY{o}{*}\PY{n}{states}\PY{p}{[}\PY{l+m+mi}{0}\PY{p}{,}\PY{l+m+mi}{3}\PY{p}{]}
        \PY{n}{mom} \PY{o}{=} \PY{p}{(}\PY{n}{xi}\PY{o}{\PYZlt{}}\PY{n}{s1}\PY{p}{)}\PY{o}{*}\PY{n}{states}\PY{p}{[}\PY{l+m+mi}{1}\PY{p}{,}\PY{l+m+mi}{0}\PY{p}{]} \PY{o}{+} \PY{p}{(}\PY{n}{s1}\PY{o}{\PYZlt{}}\PY{o}{=}\PY{n}{xi}\PY{p}{)}\PY{o}{*}\PY{p}{(}\PY{n}{xi}\PY{o}{\PYZlt{}}\PY{n}{s2}\PY{p}{)}\PY{o}{*}\PY{n}{states}\PY{p}{[}\PY{l+m+mi}{1}\PY{p}{,}\PY{l+m+mi}{1}\PY{p}{]} \PY{o}{+} \PYZbs{}
              \PY{p}{(}\PY{n}{s2}\PY{o}{\PYZlt{}}\PY{o}{=}\PY{n}{xi}\PY{p}{)}\PY{o}{*}\PY{p}{(}\PY{n}{xi}\PY{o}{\PYZlt{}}\PY{n}{s3}\PY{p}{)}\PY{o}{*}\PY{n}{states}\PY{p}{[}\PY{l+m+mi}{1}\PY{p}{,}\PY{l+m+mi}{2}\PY{p}{]} \PY{o}{+} \PY{p}{(}\PY{n}{s3}\PY{o}{\PYZlt{}}\PY{o}{=}\PY{n}{xi}\PY{p}{)}\PY{o}{*}\PY{n}{states}\PY{p}{[}\PY{l+m+mi}{1}\PY{p}{,}\PY{l+m+mi}{3}\PY{p}{]}
        \PY{n}{E} \PY{o}{=} \PY{p}{(}\PY{n}{xi}\PY{o}{\PYZlt{}}\PY{n}{s1}\PY{p}{)}\PY{o}{*}\PY{n}{states}\PY{p}{[}\PY{l+m+mi}{2}\PY{p}{,}\PY{l+m+mi}{0}\PY{p}{]} \PY{o}{+} \PY{p}{(}\PY{n}{s1}\PY{o}{\PYZlt{}}\PY{o}{=}\PY{n}{xi}\PY{p}{)}\PY{o}{*}\PY{p}{(}\PY{n}{xi}\PY{o}{\PYZlt{}}\PY{n}{s2}\PY{p}{)}\PY{o}{*}\PY{n}{states}\PY{p}{[}\PY{l+m+mi}{2}\PY{p}{,}\PY{l+m+mi}{1}\PY{p}{]} \PY{o}{+} \PYZbs{}
              \PY{p}{(}\PY{n}{s2}\PY{o}{\PYZlt{}}\PY{o}{=}\PY{n}{xi}\PY{p}{)}\PY{o}{*}\PY{p}{(}\PY{n}{xi}\PY{o}{\PYZlt{}}\PY{n}{s3}\PY{p}{)}\PY{o}{*}\PY{n}{states}\PY{p}{[}\PY{l+m+mi}{2}\PY{p}{,}\PY{l+m+mi}{2}\PY{p}{]} \PY{o}{+} \PY{p}{(}\PY{n}{s3}\PY{o}{\PYZlt{}}\PY{o}{=}\PY{n}{xi}\PY{p}{)}\PY{o}{*}\PY{n}{states}\PY{p}{[}\PY{l+m+mi}{2}\PY{p}{,}\PY{l+m+mi}{3}\PY{p}{]}
        \PY{k}{return} \PY{n}{rho}\PY{p}{,} \PY{n}{mom}\PY{p}{,} \PY{n}{E}
    
    \PY{k}{return} \PY{n}{states}\PY{p}{,} \PY{n}{speeds}\PY{p}{,} \PY{n}{reval}\PY{p}{,} \PY{n}{wave\PYZus{}types}
\end{Verbatim}
\end{tcolorbox}

    An implementation of this solver for use in Clawpack can be found
\href{https://github.com/clawpack/riemann/blob/FA16/src/rp1_euler_with_efix.f90}{here}.
Recall that an exact Riemann solver for the Euler equations appears in
\url{exact_solvers/euler.py}.

    \hypertarget{examples}{%
\subsubsection{Examples}\label{examples}}

Let's compare the Roe approximation to the exact solution. As a first
example, we use the Sod shock tube.

    \begin{tcolorbox}[breakable, size=fbox, boxrule=1pt, pad at break*=1mm,colback=cellbackground, colframe=cellborder]
\prompt{In}{incolor}{4}{\boxspacing}
\begin{Verbatim}[commandchars=\\\{\}]
\PY{k}{def} \PY{n+nf}{compare\PYZus{}solutions}\PY{p}{(}\PY{n}{left\PYZus{}state}\PY{p}{,} \PY{n}{right\PYZus{}state}\PY{p}{,} \PY{n}{solvers}\PY{o}{=}\PY{p}{[}\PY{l+s+s1}{\PYZsq{}}\PY{l+s+s1}{Exact}\PY{l+s+s1}{\PYZsq{}}\PY{p}{,}\PY{l+s+s1}{\PYZsq{}}\PY{l+s+s1}{HLLE}\PY{l+s+s1}{\PYZsq{}}\PY{p}{]}\PY{p}{)}\PY{p}{:}
    \PY{n}{q\PYZus{}l} \PY{o}{=} \PY{n}{np}\PY{o}{.}\PY{n}{array}\PY{p}{(}\PY{n}{euler}\PY{o}{.}\PY{n}{primitive\PYZus{}to\PYZus{}conservative}\PY{p}{(}\PY{o}{*}\PY{n}{left\PYZus{}state}\PY{p}{)}\PY{p}{)}
    \PY{n}{q\PYZus{}r} \PY{o}{=} \PY{n}{np}\PY{o}{.}\PY{n}{array}\PY{p}{(}\PY{n}{euler}\PY{o}{.}\PY{n}{primitive\PYZus{}to\PYZus{}conservative}\PY{p}{(}\PY{o}{*}\PY{n}{right}\PY{p}{)}\PY{p}{)}

    \PY{n}{outputs} \PY{o}{=} \PY{p}{[}\PY{p}{]}
    \PY{n}{states} \PY{o}{=} \PY{p}{\PYZob{}}\PY{p}{\PYZcb{}}

    \PY{k}{for} \PY{n}{solver} \PY{o+ow}{in} \PY{n}{solvers}\PY{p}{:}
        \PY{k}{if} \PY{n}{solver}\PY{o}{.}\PY{n}{lower}\PY{p}{(}\PY{p}{)} \PY{o}{==} \PY{l+s+s1}{\PYZsq{}}\PY{l+s+s1}{exact}\PY{l+s+s1}{\PYZsq{}}\PY{p}{:}
            \PY{n}{outputs}\PY{o}{.}\PY{n}{append}\PY{p}{(}\PY{n}{euler}\PY{o}{.}\PY{n}{exact\PYZus{}riemann\PYZus{}solution}\PY{p}{(}\PY{n}{q\PYZus{}l}\PY{p}{,}\PY{n}{q\PYZus{}r}\PY{p}{)}\PY{p}{)}
        \PY{k}{if} \PY{n}{solver}\PY{o}{.}\PY{n}{lower}\PY{p}{(}\PY{p}{)} \PY{o}{==} \PY{l+s+s1}{\PYZsq{}}\PY{l+s+s1}{hlle}\PY{l+s+s1}{\PYZsq{}}\PY{p}{:}
            \PY{n}{outputs}\PY{o}{.}\PY{n}{append}\PY{p}{(}\PY{n}{Euler\PYZus{}hlle}\PY{p}{(}\PY{n}{q\PYZus{}l}\PY{p}{,} \PY{n}{q\PYZus{}r}\PY{p}{)}\PY{p}{)}
            \PY{n}{states}\PY{p}{[}\PY{l+s+s1}{\PYZsq{}}\PY{l+s+s1}{hlle}\PY{l+s+s1}{\PYZsq{}}\PY{p}{]} \PY{o}{=} \PY{n}{outputs}\PY{p}{[}\PY{o}{\PYZhy{}}\PY{l+m+mi}{1}\PY{p}{]}\PY{p}{[}\PY{l+m+mi}{0}\PY{p}{]}
        \PY{k}{if} \PY{n}{solver}\PY{o}{.}\PY{n}{lower}\PY{p}{(}\PY{p}{)} \PY{o}{==} \PY{l+s+s1}{\PYZsq{}}\PY{l+s+s1}{roe}\PY{l+s+s1}{\PYZsq{}}\PY{p}{:}
            \PY{n}{outputs}\PY{o}{.}\PY{n}{append}\PY{p}{(}\PY{n}{Euler\PYZus{}roe}\PY{p}{(}\PY{n}{q\PYZus{}l}\PY{p}{,} \PY{n}{q\PYZus{}r}\PY{p}{)}\PY{p}{)}
            \PY{n}{states}\PY{p}{[}\PY{l+s+s1}{\PYZsq{}}\PY{l+s+s1}{roe}\PY{l+s+s1}{\PYZsq{}}\PY{p}{]} \PY{o}{=} \PY{n}{outputs}\PY{p}{[}\PY{o}{\PYZhy{}}\PY{l+m+mi}{1}\PY{p}{]}\PY{p}{[}\PY{l+m+mi}{0}\PY{p}{]}

    \PY{n}{plot\PYZus{}function} \PY{o}{=} \PYZbs{}
        \PY{n}{rt}\PY{o}{.}\PY{n}{make\PYZus{}plot\PYZus{}function}\PY{p}{(}\PY{p}{[}\PY{n}{val}\PY{p}{[}\PY{l+m+mi}{0}\PY{p}{]} \PY{k}{for} \PY{n}{val} \PY{o+ow}{in} \PY{n}{outputs}\PY{p}{]}\PY{p}{,}
                              \PY{p}{[}\PY{n}{val}\PY{p}{[}\PY{l+m+mi}{1}\PY{p}{]} \PY{k}{for} \PY{n}{val} \PY{o+ow}{in} \PY{n}{outputs}\PY{p}{]}\PY{p}{,}
                              \PY{p}{[}\PY{n}{val}\PY{p}{[}\PY{l+m+mi}{2}\PY{p}{]} \PY{k}{for} \PY{n}{val} \PY{o+ow}{in} \PY{n}{outputs}\PY{p}{]}\PY{p}{,}
                              \PY{p}{[}\PY{n}{val}\PY{p}{[}\PY{l+m+mi}{3}\PY{p}{]} \PY{k}{for} \PY{n}{val} \PY{o+ow}{in} \PY{n}{outputs}\PY{p}{]}\PY{p}{,}
                              \PY{n}{solvers}\PY{p}{,} \PY{n}{layout}\PY{o}{=}\PY{l+s+s1}{\PYZsq{}}\PY{l+s+s1}{vertical}\PY{l+s+s1}{\PYZsq{}}\PY{p}{,}
                              \PY{n}{variable\PYZus{}names}\PY{o}{=}\PY{n}{euler}\PY{o}{.}\PY{n}{primitive\PYZus{}variables}\PY{p}{,}
                              \PY{n}{derived\PYZus{}variables}\PY{o}{=}\PY{n}{euler}\PY{o}{.}\PY{n}{cons\PYZus{}to\PYZus{}prim}\PY{p}{,}
                              \PY{n}{vertical\PYZus{}spacing}\PY{o}{=}\PY{l+m+mf}{0.15}\PY{p}{,}
                              \PY{n}{show\PYZus{}time\PYZus{}legend}\PY{o}{=}\PY{k+kc}{True}\PY{p}{)}
    
    \PY{n}{interact}\PY{p}{(}\PY{n}{plot\PYZus{}function}\PY{p}{,}
             \PY{n}{t}\PY{o}{=}\PY{n}{widgets}\PY{o}{.}\PY{n}{FloatSlider}\PY{p}{(}\PY{n+nb}{min}\PY{o}{=}\PY{l+m+mi}{0}\PY{p}{,}\PY{n+nb}{max}\PY{o}{=}\PY{l+m+mf}{0.9}\PY{p}{,}\PY{n}{step}\PY{o}{=}\PY{l+m+mf}{0.1}\PY{p}{,}\PY{n}{value}\PY{o}{=}\PY{l+m+mf}{0.4}\PY{p}{)}\PY{p}{)}\PY{p}{;}
    
    \PY{k}{return} \PY{n}{states}
\end{Verbatim}
\end{tcolorbox}

    \begin{tcolorbox}[breakable, size=fbox, boxrule=1pt, pad at break*=1mm,colback=cellbackground, colframe=cellborder]
\prompt{In}{incolor}{5}{\boxspacing}
\begin{Verbatim}[commandchars=\\\{\}]
\PY{n}{left}  \PY{o}{=} \PY{n}{State}\PY{p}{(}\PY{n}{Density} \PY{o}{=} \PY{l+m+mf}{3.}\PY{p}{,}
              \PY{n}{Velocity} \PY{o}{=} \PY{l+m+mf}{0.}\PY{p}{,}
              \PY{n}{Pressure} \PY{o}{=} \PY{l+m+mf}{3.}\PY{p}{)}
\PY{n}{right} \PY{o}{=} \PY{n}{State}\PY{p}{(}\PY{n}{Density} \PY{o}{=} \PY{l+m+mf}{1.}\PY{p}{,}
              \PY{n}{Velocity} \PY{o}{=} \PY{l+m+mf}{0.}\PY{p}{,}
              \PY{n}{Pressure} \PY{o}{=} \PY{l+m+mf}{1.}\PY{p}{)}

\PY{n}{states} \PY{o}{=} \PY{n}{compare\PYZus{}solutions}\PY{p}{(}\PY{n}{left}\PY{p}{,} \PY{n}{right}\PY{p}{,} \PY{n}{solvers}\PY{o}{=}\PY{p}{[}\PY{l+s+s1}{\PYZsq{}}\PY{l+s+s1}{Exact}\PY{l+s+s1}{\PYZsq{}}\PY{p}{,}\PY{l+s+s1}{\PYZsq{}}\PY{l+s+s1}{Roe}\PY{l+s+s1}{\PYZsq{}}\PY{p}{]}\PY{p}{)}
\end{Verbatim}
\end{tcolorbox}

    
    \begin{Verbatim}[commandchars=\\\{\}]
interactive(children=(FloatSlider(value=0.4, description='t', max=0.9), Output()), \_dom\_classes=('widget-inter…
    \end{Verbatim}

    
    \begin{tcolorbox}[breakable, size=fbox, boxrule=1pt, pad at break*=1mm,colback=cellbackground, colframe=cellborder]
\prompt{In}{incolor}{6}{\boxspacing}
\begin{Verbatim}[commandchars=\\\{\}]
\PY{n}{euler}\PY{o}{.}\PY{n}{phase\PYZus{}plane\PYZus{}plot}\PY{p}{(}\PY{n}{left}\PY{p}{,} \PY{n}{right}\PY{p}{,} \PY{n}{approx\PYZus{}states}\PY{o}{=}\PY{n}{states}\PY{p}{[}\PY{l+s+s1}{\PYZsq{}}\PY{l+s+s1}{roe}\PY{l+s+s1}{\PYZsq{}}\PY{p}{]}\PY{p}{)}
\end{Verbatim}
\end{tcolorbox}

    \begin{center}
    \adjustimage{max size={0.9\linewidth}{0.9\paperheight}}{Euler_approximate_files/Euler_approximate_14_0.pdf}
    \end{center}
    { \hspace*{\fill} \\}
    
    Recall that in the true solution the middle wave is a contact
discontinuity and carries only a jump in the density. For that reason
the three-dimensional phase space plot is generally shown projected onto
the pressure-velocity plane as shown above: The two intermediate states
in the true solution have the same pressure and velocity, and so are
denoted by a single Middle state in the phase plane plot.

The Roe solver, on the other hand, generates a middle wave that carries
a jump in all 3 variables and there are two green dots appearing in the
plot above for the two middle states (though the pressure jump is quite
small in this example). For a Riemann problem like this one with zero
initial velocity on both sides, the Roe average velocity must also be
zero, so the middle wave is stationary; this is of course not typically
true in the exact solution, even when \(u_\ell=u_r=0\).

    Here is a second example. Experiment with the initial states to explore
how the Roe solution compares to the exact solution.

    \begin{tcolorbox}[breakable, size=fbox, boxrule=1pt, pad at break*=1mm,colback=cellbackground, colframe=cellborder]
\prompt{In}{incolor}{7}{\boxspacing}
\begin{Verbatim}[commandchars=\\\{\}]
\PY{n}{left}  \PY{o}{=} \PY{n}{State}\PY{p}{(}\PY{n}{Density} \PY{o}{=} \PY{l+m+mf}{0.1}\PY{p}{,}
              \PY{n}{Velocity} \PY{o}{=} \PY{l+m+mf}{0.}\PY{p}{,}
              \PY{n}{Pressure} \PY{o}{=} \PY{l+m+mf}{0.1}\PY{p}{)}
\PY{n}{right} \PY{o}{=} \PY{n}{State}\PY{p}{(}\PY{n}{Density} \PY{o}{=} \PY{l+m+mf}{1.}\PY{p}{,}
              \PY{n}{Velocity} \PY{o}{=} \PY{l+m+mf}{1.}\PY{p}{,}
              \PY{n}{Pressure} \PY{o}{=} \PY{l+m+mf}{1.}\PY{p}{)}

\PY{n}{states} \PY{o}{=} \PY{n}{compare\PYZus{}solutions}\PY{p}{(}\PY{n}{left}\PY{p}{,} \PY{n}{right}\PY{p}{,} \PY{n}{solvers}\PY{o}{=}\PY{p}{[}\PY{l+s+s1}{\PYZsq{}}\PY{l+s+s1}{Exact}\PY{l+s+s1}{\PYZsq{}}\PY{p}{,}\PY{l+s+s1}{\PYZsq{}}\PY{l+s+s1}{Roe}\PY{l+s+s1}{\PYZsq{}}\PY{p}{]}\PY{p}{)}
\end{Verbatim}
\end{tcolorbox}

    
    \begin{Verbatim}[commandchars=\\\{\}]
interactive(children=(FloatSlider(value=0.4, description='t', max=0.9), Output()), \_dom\_classes=('widget-inter…
    \end{Verbatim}

    
    \begin{tcolorbox}[breakable, size=fbox, boxrule=1pt, pad at break*=1mm,colback=cellbackground, colframe=cellborder]
\prompt{In}{incolor}{8}{\boxspacing}
\begin{Verbatim}[commandchars=\\\{\}]
\PY{n}{euler}\PY{o}{.}\PY{n}{phase\PYZus{}plane\PYZus{}plot}\PY{p}{(}\PY{n}{left}\PY{p}{,} \PY{n}{right}\PY{p}{,} \PY{n}{approx\PYZus{}states}\PY{o}{=}\PY{n}{states}\PY{p}{[}\PY{l+s+s1}{\PYZsq{}}\PY{l+s+s1}{roe}\PY{l+s+s1}{\PYZsq{}}\PY{p}{]}\PY{p}{)}
\end{Verbatim}
\end{tcolorbox}

    \begin{center}
    \adjustimage{max size={0.9\linewidth}{0.9\paperheight}}{Euler_approximate_files/Euler_approximate_18_0.pdf}
    \end{center}
    { \hspace*{\fill} \\}
    
    \hypertarget{single-shock-solution}{%
\subsubsection{Single-shock solution}\label{single-shock-solution}}

Next we demonstrate the exactness property of the Roe solver by applying
it to a case where the left and right states are connected by a single
shock wave.

    \begin{tcolorbox}[breakable, size=fbox, boxrule=1pt, pad at break*=1mm,colback=cellbackground, colframe=cellborder]
\prompt{In}{incolor}{9}{\boxspacing}
\begin{Verbatim}[commandchars=\\\{\}]
\PY{n}{M} \PY{o}{=} \PY{l+m+mf}{2.}  \PY{c+c1}{\PYZsh{} Mach number of the shock wave}
\PY{n}{gamma} \PY{o}{=} \PY{l+m+mf}{1.4}
\PY{n}{mu} \PY{o}{=} \PY{l+m+mi}{2}\PY{o}{*}\PY{p}{(}\PY{n}{M}\PY{o}{*}\PY{o}{*}\PY{l+m+mi}{2}\PY{o}{\PYZhy{}}\PY{l+m+mi}{1}\PY{p}{)}\PY{o}{/}\PY{p}{(}\PY{n}{M}\PY{o}{*}\PY{p}{(}\PY{n}{gamma}\PY{o}{+}\PY{l+m+mf}{1.}\PY{p}{)}\PY{p}{)}
\PY{n}{right} \PY{o}{=} \PY{n}{State}\PY{p}{(}\PY{n}{Density} \PY{o}{=} \PY{l+m+mf}{1.}\PY{p}{,}
              \PY{n}{Velocity} \PY{o}{=} \PY{l+m+mf}{0.}\PY{p}{,}
              \PY{n}{Pressure} \PY{o}{=} \PY{l+m+mf}{1.}\PY{p}{)}
\PY{n}{c\PYZus{}r} \PY{o}{=} \PY{n}{np}\PY{o}{.}\PY{n}{sqrt}\PY{p}{(}\PY{n}{gamma}\PY{o}{*}\PY{n}{right}\PY{o}{.}\PY{n}{Pressure}\PY{o}{/}\PY{n}{right}\PY{o}{.}\PY{n}{Density}\PY{p}{)}

\PY{n}{rho\PYZus{}l} \PY{o}{=} \PY{n}{right}\PY{o}{.}\PY{n}{Density} \PY{o}{*} \PY{n}{M}\PY{o}{/}\PY{p}{(}\PY{n}{M}\PY{o}{\PYZhy{}}\PY{n}{mu}\PY{p}{)}
\PY{n}{p\PYZus{}l} \PY{o}{=} \PY{n}{right}\PY{o}{.}\PY{n}{Pressure} \PY{o}{*} \PY{p}{(}\PY{p}{(}\PY{l+m+mi}{2}\PY{o}{*}\PY{n}{M}\PY{o}{*}\PY{o}{*}\PY{l+m+mi}{2}\PY{o}{\PYZhy{}}\PY{l+m+mi}{1}\PY{p}{)}\PY{o}{*}\PY{n}{gamma}\PY{o}{+}\PY{l+m+mi}{1}\PY{p}{)}\PY{o}{/}\PY{p}{(}\PY{n}{gamma}\PY{o}{+}\PY{l+m+mi}{1}\PY{p}{)}
\PY{n}{u\PYZus{}l} \PY{o}{=} \PY{n}{mu}\PY{o}{*}\PY{n}{c\PYZus{}r}

\PY{n}{left} \PY{o}{=} \PY{n}{State}\PY{p}{(}\PY{n}{Density} \PY{o}{=} \PY{n}{rho\PYZus{}l}\PY{p}{,}
             \PY{n}{Velocity} \PY{o}{=} \PY{n}{u\PYZus{}l}\PY{p}{,}
            \PY{n}{Pressure} \PY{o}{=} \PY{n}{p\PYZus{}l}\PY{p}{)}

\PY{n}{states} \PY{o}{=} \PY{n}{compare\PYZus{}solutions}\PY{p}{(}\PY{n}{left}\PY{p}{,} \PY{n}{right}\PY{p}{,} \PY{n}{solvers}\PY{o}{=}\PY{p}{[}\PY{l+s+s1}{\PYZsq{}}\PY{l+s+s1}{Exact}\PY{l+s+s1}{\PYZsq{}}\PY{p}{,}\PY{l+s+s1}{\PYZsq{}}\PY{l+s+s1}{Roe}\PY{l+s+s1}{\PYZsq{}}\PY{p}{]}\PY{p}{)}
\end{Verbatim}
\end{tcolorbox}

    
    \begin{Verbatim}[commandchars=\\\{\}]
interactive(children=(FloatSlider(value=0.4, description='t', max=0.9), Output()), \_dom\_classes=('widget-inter…
    \end{Verbatim}

    
    \begin{tcolorbox}[breakable, size=fbox, boxrule=1pt, pad at break*=1mm,colback=cellbackground, colframe=cellborder]
\prompt{In}{incolor}{10}{\boxspacing}
\begin{Verbatim}[commandchars=\\\{\}]
\PY{n}{euler}\PY{o}{.}\PY{n}{phase\PYZus{}plane\PYZus{}plot}\PY{p}{(}\PY{n}{left}\PY{p}{,} \PY{n}{right}\PY{p}{,} \PY{n}{approx\PYZus{}states}\PY{o}{=}\PY{n}{states}\PY{p}{[}\PY{l+s+s1}{\PYZsq{}}\PY{l+s+s1}{roe}\PY{l+s+s1}{\PYZsq{}}\PY{p}{]}\PY{p}{)}
\end{Verbatim}
\end{tcolorbox}

    \begin{center}
    \adjustimage{max size={0.9\linewidth}{0.9\paperheight}}{Euler_approximate_files/Euler_approximate_21_0.pdf}
    \end{center}
    { \hspace*{\fill} \\}
    
    It is evident that the solution consists of a single right-going shock.
The exact solution cannot be seen because it coincides exactly with the
Roe solution. The path of the shock in the first plot also cannot be
seen since it is plotted under the path of the rightmost Roe solution
wave. The two solutions differ only in the wave speeds predicted for the
other two waves, but since these waves have zero strength this makes no
difference.

    \hypertarget{transonic-rarefactions-and-an-entropy-fix}{%
\subsubsection{Transonic rarefactions and an entropy
fix}\label{transonic-rarefactions-and-an-entropy-fix}}

Here is an example of a Riemann problem whose solution includes a
transonic 2-rarefaction:

    \begin{tcolorbox}[breakable, size=fbox, boxrule=1pt, pad at break*=1mm,colback=cellbackground, colframe=cellborder]
\prompt{In}{incolor}{11}{\boxspacing}
\begin{Verbatim}[commandchars=\\\{\}]
\PY{n}{left}  \PY{o}{=} \PY{n}{State}\PY{p}{(}\PY{n}{Density} \PY{o}{=} \PY{l+m+mf}{0.1}\PY{p}{,}
              \PY{n}{Velocity} \PY{o}{=} \PY{o}{\PYZhy{}}\PY{l+m+mf}{2.}\PY{p}{,}
              \PY{n}{Pressure} \PY{o}{=} \PY{l+m+mf}{0.1}\PY{p}{)}
\PY{n}{right} \PY{o}{=} \PY{n}{State}\PY{p}{(}\PY{n}{Density} \PY{o}{=} \PY{l+m+mf}{1.}\PY{p}{,}
              \PY{n}{Velocity} \PY{o}{=} \PY{o}{\PYZhy{}}\PY{l+m+mf}{1.}\PY{p}{,}
              \PY{n}{Pressure} \PY{o}{=} \PY{l+m+mf}{1.}\PY{p}{)}

\PY{n}{states} \PY{o}{=} \PY{n}{compare\PYZus{}solutions}\PY{p}{(}\PY{n}{left}\PY{p}{,} \PY{n}{right}\PY{p}{,} \PY{n}{solvers}\PY{o}{=}\PY{p}{[}\PY{l+s+s1}{\PYZsq{}}\PY{l+s+s1}{Exact}\PY{l+s+s1}{\PYZsq{}}\PY{p}{,}\PY{l+s+s1}{\PYZsq{}}\PY{l+s+s1}{Roe}\PY{l+s+s1}{\PYZsq{}}\PY{p}{]}\PY{p}{)}
\end{Verbatim}
\end{tcolorbox}

    
    \begin{Verbatim}[commandchars=\\\{\}]
interactive(children=(FloatSlider(value=0.4, description='t', max=0.9), Output()), \_dom\_classes=('widget-inter…
    \end{Verbatim}

    
    Notice that in the exact solution, the right edge of the rarefaction
travels to the right. In the Roe solution, all waves travel to the left.
As in the case of the shallow water equations, here too this behavior
can lead to unphysical solutions when this approximate solver is used in
a numerical discretization. In order to correct this, we can split the
single wave into two when a transonic rarefaction is present, in a way
similar to what is done in the shallow water equations. We do not go
into details here.

    \hypertarget{hlle-solver}{%
\subsection{HLLE Solver}\label{hlle-solver}}

Recall that an HLL solver uses only two waves with a constant state
between them. The Euler equations are our first example for which the
number of waves in the true solution is larger than the number of waves
in the approximate solution. As one might expect, this leads to
noticeable inaccuracy in solutions produced by the solver.

Again following Einfeldt, the left-going wave speed is chosen to be the
minimum of the Roe speed for the 1-wave and the characterstic speed
\(\lambda^1\) in the left state \(q_\ell\). The right-going wave speed
is chosen to be the maximum of the Roe speed for the 3-wave and the
characteristic speed \(\lambda^3\) in the right state \(q_r\).
Effectively, this means that \begin{align}
    s_1 & = \min(u_\ell - c_\ell, \hat{u}-\hat{c}) \\
    s_2 & = \max(u_r + c_r, \hat{u}+\hat{c})
\end{align}

Recall that once we have chosen these two wave speeds, conservation
dictates the value of the intermediate state:\\
\begin{align}  \label{SWA:hll_middle_state}
q_m = \frac{f(q_r) - f(q_\ell) - s_2 q_r + s_1 q_\ell}{s_1 - s_2}.
\end{align}

    \begin{tcolorbox}[breakable, size=fbox, boxrule=1pt, pad at break*=1mm,colback=cellbackground, colframe=cellborder]
\prompt{In}{incolor}{12}{\boxspacing}
\begin{Verbatim}[commandchars=\\\{\}]
\PY{k}{def} \PY{n+nf}{Euler\PYZus{}hlle}\PY{p}{(}\PY{n}{q\PYZus{}l}\PY{p}{,} \PY{n}{q\PYZus{}r}\PY{p}{,} \PY{n}{gamma}\PY{o}{=}\PY{l+m+mf}{1.4}\PY{p}{)}\PY{p}{:}
    \PY{l+s+sd}{\PYZdq{}\PYZdq{}\PYZdq{}HLLE approximate solver for the Euler equations.\PYZdq{}\PYZdq{}\PYZdq{}}
    
    \PY{n}{rho\PYZus{}l} \PY{o}{=} \PY{n}{q\PYZus{}l}\PY{p}{[}\PY{l+m+mi}{0}\PY{p}{]}
    \PY{n}{rhou\PYZus{}l} \PY{o}{=} \PY{n}{q\PYZus{}l}\PY{p}{[}\PY{l+m+mi}{1}\PY{p}{]}
    \PY{n}{u\PYZus{}l} \PY{o}{=} \PY{n}{rhou\PYZus{}l}\PY{o}{/}\PY{n}{rho\PYZus{}l}
    \PY{n}{rho\PYZus{}r} \PY{o}{=} \PY{n}{q\PYZus{}r}\PY{p}{[}\PY{l+m+mi}{0}\PY{p}{]}
    \PY{n}{rhou\PYZus{}r} \PY{o}{=} \PY{n}{q\PYZus{}r}\PY{p}{[}\PY{l+m+mi}{1}\PY{p}{]}
    \PY{n}{u\PYZus{}r} \PY{o}{=} \PY{n}{rhou\PYZus{}r}\PY{o}{/}\PY{n}{rho\PYZus{}r}
    \PY{n}{E\PYZus{}r} \PY{o}{=} \PY{n}{q\PYZus{}r}\PY{p}{[}\PY{l+m+mi}{2}\PY{p}{]}
    \PY{n}{E\PYZus{}l} \PY{o}{=} \PY{n}{q\PYZus{}l}\PY{p}{[}\PY{l+m+mi}{2}\PY{p}{]}
    
    \PY{n}{u\PYZus{}hat}\PY{p}{,} \PY{n}{c\PYZus{}hat}\PY{p}{,} \PY{n}{H\PYZus{}hat} \PY{o}{=} \PY{n}{roe\PYZus{}averages}\PY{p}{(}\PY{n}{q\PYZus{}l}\PY{p}{,} \PY{n}{q\PYZus{}r}\PY{p}{,} \PY{n}{gamma}\PY{p}{)}
    \PY{n}{p\PYZus{}r} \PY{o}{=} \PY{p}{(}\PY{n}{gamma}\PY{o}{\PYZhy{}}\PY{l+m+mf}{1.}\PY{p}{)} \PY{o}{*} \PY{p}{(}\PY{n}{E\PYZus{}r} \PY{o}{\PYZhy{}} \PY{n}{rho\PYZus{}r}\PY{o}{*}\PY{n}{u\PYZus{}r}\PY{o}{*}\PY{o}{*}\PY{l+m+mi}{2}\PY{o}{/}\PY{l+m+mf}{2.}\PY{p}{)}
    \PY{n}{p\PYZus{}l} \PY{o}{=} \PY{p}{(}\PY{n}{gamma}\PY{o}{\PYZhy{}}\PY{l+m+mf}{1.}\PY{p}{)} \PY{o}{*} \PY{p}{(}\PY{n}{E\PYZus{}l} \PY{o}{\PYZhy{}} \PY{n}{rho\PYZus{}l}\PY{o}{*}\PY{n}{u\PYZus{}l}\PY{o}{*}\PY{o}{*}\PY{l+m+mi}{2}\PY{o}{/}\PY{l+m+mf}{2.}\PY{p}{)}
    \PY{n}{H\PYZus{}r} \PY{o}{=} \PY{p}{(}\PY{n}{E\PYZus{}r}\PY{o}{+}\PY{n}{p\PYZus{}r}\PY{p}{)} \PY{o}{/} \PY{n}{rho\PYZus{}r}
    \PY{n}{H\PYZus{}l} \PY{o}{=} \PY{p}{(}\PY{n}{E\PYZus{}l}\PY{o}{+}\PY{n}{p\PYZus{}l}\PY{p}{)} \PY{o}{/} \PY{n}{rho\PYZus{}l}
    \PY{n}{c\PYZus{}r} \PY{o}{=} \PY{n}{np}\PY{o}{.}\PY{n}{sqrt}\PY{p}{(}\PY{p}{(}\PY{n}{gamma}\PY{o}{\PYZhy{}}\PY{l+m+mf}{1.}\PY{p}{)}\PY{o}{*}\PY{p}{(}\PY{n}{H\PYZus{}r}\PY{o}{\PYZhy{}}\PY{n}{u\PYZus{}r}\PY{o}{*}\PY{o}{*}\PY{l+m+mi}{2}\PY{o}{/}\PY{l+m+mf}{2.}\PY{p}{)}\PY{p}{)}
    \PY{n}{c\PYZus{}l} \PY{o}{=} \PY{n}{np}\PY{o}{.}\PY{n}{sqrt}\PY{p}{(}\PY{p}{(}\PY{n}{gamma}\PY{o}{\PYZhy{}}\PY{l+m+mf}{1.}\PY{p}{)}\PY{o}{*}\PY{p}{(}\PY{n}{H\PYZus{}l}\PY{o}{\PYZhy{}}\PY{n}{u\PYZus{}l}\PY{o}{*}\PY{o}{*}\PY{l+m+mi}{2}\PY{o}{/}\PY{l+m+mf}{2.}\PY{p}{)}\PY{p}{)}
    
    \PY{n}{s1} \PY{o}{=} \PY{n+nb}{min}\PY{p}{(}\PY{n}{u\PYZus{}l}\PY{o}{\PYZhy{}}\PY{n}{c\PYZus{}l}\PY{p}{,}\PY{n}{u\PYZus{}hat}\PY{o}{\PYZhy{}}\PY{n}{c\PYZus{}hat}\PY{p}{)}
    \PY{n}{s2} \PY{o}{=} \PY{n+nb}{max}\PY{p}{(}\PY{n}{u\PYZus{}r}\PY{o}{+}\PY{n}{c\PYZus{}r}\PY{p}{,}\PY{n}{u\PYZus{}hat}\PY{o}{+}\PY{n}{c\PYZus{}hat}\PY{p}{)}
    
    \PY{n}{rho\PYZus{}m} \PY{o}{=} \PY{p}{(}\PY{n}{rhou\PYZus{}r} \PY{o}{\PYZhy{}} \PY{n}{rhou\PYZus{}l} \PY{o}{\PYZhy{}} \PY{n}{s2}\PY{o}{*}\PY{n}{rho\PYZus{}r} \PY{o}{+} \PY{n}{s1}\PY{o}{*}\PY{n}{rho\PYZus{}l}\PY{p}{)}\PY{o}{/}\PY{p}{(}\PY{n}{s1}\PY{o}{\PYZhy{}}\PY{n}{s2}\PY{p}{)}
    \PY{n}{rhou\PYZus{}m} \PY{o}{=} \PY{p}{(}\PY{n}{rho\PYZus{}r}\PY{o}{*}\PY{n}{u\PYZus{}r}\PY{o}{*}\PY{o}{*}\PY{l+m+mi}{2} \PY{o}{\PYZhy{}} \PY{n}{rho\PYZus{}l}\PY{o}{*}\PY{n}{u\PYZus{}l}\PY{o}{*}\PY{o}{*}\PY{l+m+mi}{2} \PYZbs{}
              \PY{o}{+} \PY{n}{p\PYZus{}r} \PY{o}{\PYZhy{}} \PY{n}{p\PYZus{}l} \PY{o}{\PYZhy{}} \PY{n}{s2}\PY{o}{*}\PY{n}{rhou\PYZus{}r} \PY{o}{+} \PY{n}{s1}\PY{o}{*}\PY{n}{rhou\PYZus{}l}\PY{p}{)}\PY{o}{/}\PY{p}{(}\PY{n}{s1}\PY{o}{\PYZhy{}}\PY{n}{s2}\PY{p}{)}
    \PY{n}{E\PYZus{}m} \PY{o}{=} \PY{p}{(} \PY{n}{u\PYZus{}r}\PY{o}{*}\PY{p}{(}\PY{n}{E\PYZus{}r}\PY{o}{+}\PY{n}{p\PYZus{}r}\PY{p}{)} \PY{o}{\PYZhy{}} \PY{n}{u\PYZus{}l}\PY{o}{*}\PY{p}{(}\PY{n}{E\PYZus{}l}\PY{o}{+}\PY{n}{p\PYZus{}l}\PY{p}{)} \PY{o}{\PYZhy{}} \PY{n}{s2}\PY{o}{*}\PY{n}{E\PYZus{}r} \PY{o}{+} \PY{n}{s1}\PY{o}{*}\PY{n}{E\PYZus{}l}\PY{p}{)}\PY{o}{/}\PY{p}{(}\PY{n}{s1}\PY{o}{\PYZhy{}}\PY{n}{s2}\PY{p}{)}
    \PY{n}{q\PYZus{}m} \PY{o}{=} \PY{n}{np}\PY{o}{.}\PY{n}{array}\PY{p}{(}\PY{p}{[}\PY{n}{rho\PYZus{}m}\PY{p}{,} \PY{n}{rhou\PYZus{}m}\PY{p}{,} \PY{n}{E\PYZus{}m}\PY{p}{]}\PY{p}{)}
    
    \PY{n}{states} \PY{o}{=} \PY{n}{np}\PY{o}{.}\PY{n}{column\PYZus{}stack}\PY{p}{(}\PY{p}{[}\PY{n}{q\PYZus{}l}\PY{p}{,}\PY{n}{q\PYZus{}m}\PY{p}{,}\PY{n}{q\PYZus{}r}\PY{p}{]}\PY{p}{)}
    \PY{n}{speeds} \PY{o}{=} \PY{p}{[}\PY{n}{s1}\PY{p}{,} \PY{n}{s2}\PY{p}{]}
    \PY{n}{wave\PYZus{}types} \PY{o}{=} \PY{p}{[}\PY{l+s+s1}{\PYZsq{}}\PY{l+s+s1}{contact}\PY{l+s+s1}{\PYZsq{}}\PY{p}{,}\PY{l+s+s1}{\PYZsq{}}\PY{l+s+s1}{contact}\PY{l+s+s1}{\PYZsq{}}\PY{p}{]}
    
    \PY{k}{def} \PY{n+nf}{reval}\PY{p}{(}\PY{n}{xi}\PY{p}{)}\PY{p}{:}
        \PY{n}{rho}  \PY{o}{=} \PY{p}{(}\PY{n}{xi}\PY{o}{\PYZlt{}}\PY{n}{s1}\PY{p}{)}\PY{o}{*}\PY{n}{rho\PYZus{}l} \PY{o}{+} \PY{p}{(}\PY{n}{s1}\PY{o}{\PYZlt{}}\PY{o}{=}\PY{n}{xi}\PY{p}{)}\PY{o}{*}\PY{p}{(}\PY{n}{xi}\PY{o}{\PYZlt{}}\PY{o}{=}\PY{n}{s2}\PY{p}{)}\PY{o}{*}\PY{n}{rho\PYZus{}m} \PY{o}{+} \PY{p}{(}\PY{n}{s2}\PY{o}{\PYZlt{}}\PY{n}{xi}\PY{p}{)}\PY{o}{*}\PY{n}{rho\PYZus{}r}
        \PY{n}{mom}  \PY{o}{=} \PY{p}{(}\PY{n}{xi}\PY{o}{\PYZlt{}}\PY{n}{s1}\PY{p}{)}\PY{o}{*}\PY{n}{rhou\PYZus{}l} \PY{o}{+} \PY{p}{(}\PY{n}{s1}\PY{o}{\PYZlt{}}\PY{o}{=}\PY{n}{xi}\PY{p}{)}\PY{o}{*}\PY{p}{(}\PY{n}{xi}\PY{o}{\PYZlt{}}\PY{o}{=}\PY{n}{s2}\PY{p}{)}\PY{o}{*}\PY{n}{rhou\PYZus{}m} \PY{o}{+} \PY{p}{(}\PY{n}{s2}\PY{o}{\PYZlt{}}\PY{n}{xi}\PY{p}{)}\PY{o}{*}\PY{n}{rhou\PYZus{}r}
        \PY{n}{E}  \PY{o}{=} \PY{p}{(}\PY{n}{xi}\PY{o}{\PYZlt{}}\PY{n}{s1}\PY{p}{)}\PY{o}{*}\PY{n}{E\PYZus{}l} \PY{o}{+} \PY{p}{(}\PY{n}{s1}\PY{o}{\PYZlt{}}\PY{o}{=}\PY{n}{xi}\PY{p}{)}\PY{o}{*}\PY{p}{(}\PY{n}{xi}\PY{o}{\PYZlt{}}\PY{o}{=}\PY{n}{s2}\PY{p}{)}\PY{o}{*}\PY{n}{E\PYZus{}m} \PY{o}{+} \PY{p}{(}\PY{n}{s2}\PY{o}{\PYZlt{}}\PY{n}{xi}\PY{p}{)}\PY{o}{*}\PY{n}{E\PYZus{}r}
        \PY{k}{return} \PY{n}{rho}\PY{p}{,} \PY{n}{mom}\PY{p}{,} \PY{n}{E}

    \PY{k}{return} \PY{n}{states}\PY{p}{,} \PY{n}{speeds}\PY{p}{,} \PY{n}{reval}\PY{p}{,} \PY{n}{wave\PYZus{}types}
\end{Verbatim}
\end{tcolorbox}

    \hypertarget{examples}{%
\subsubsection{Examples}\label{examples}}

    \begin{tcolorbox}[breakable, size=fbox, boxrule=1pt, pad at break*=1mm,colback=cellbackground, colframe=cellborder]
\prompt{In}{incolor}{13}{\boxspacing}
\begin{Verbatim}[commandchars=\\\{\}]
\PY{n}{left}  \PY{o}{=} \PY{n}{State}\PY{p}{(}\PY{n}{Density} \PY{o}{=} \PY{l+m+mf}{3.}\PY{p}{,}
              \PY{n}{Velocity} \PY{o}{=} \PY{l+m+mf}{0.}\PY{p}{,}
              \PY{n}{Pressure} \PY{o}{=} \PY{l+m+mf}{3.}\PY{p}{)}
\PY{n}{right} \PY{o}{=} \PY{n}{State}\PY{p}{(}\PY{n}{Density} \PY{o}{=} \PY{l+m+mf}{1.}\PY{p}{,}
              \PY{n}{Velocity} \PY{o}{=} \PY{l+m+mf}{0.}\PY{p}{,}
              \PY{n}{Pressure} \PY{o}{=} \PY{l+m+mf}{1.}\PY{p}{)}
    
\PY{n}{states} \PY{o}{=} \PY{n}{compare\PYZus{}solutions}\PY{p}{(}\PY{n}{left}\PY{p}{,} \PY{n}{right}\PY{p}{,} \PY{n}{solvers}\PY{o}{=}\PY{p}{[}\PY{l+s+s1}{\PYZsq{}}\PY{l+s+s1}{Exact}\PY{l+s+s1}{\PYZsq{}}\PY{p}{,}\PY{l+s+s1}{\PYZsq{}}\PY{l+s+s1}{HLLE}\PY{l+s+s1}{\PYZsq{}}\PY{p}{]}\PY{p}{)}
\end{Verbatim}
\end{tcolorbox}

    
    \begin{Verbatim}[commandchars=\\\{\}]
interactive(children=(FloatSlider(value=0.4, description='t', max=0.9), Output()), \_dom\_classes=('widget-inter…
    \end{Verbatim}

    
    \begin{tcolorbox}[breakable, size=fbox, boxrule=1pt, pad at break*=1mm,colback=cellbackground, colframe=cellborder]
\prompt{In}{incolor}{14}{\boxspacing}
\begin{Verbatim}[commandchars=\\\{\}]
\PY{n}{euler}\PY{o}{.}\PY{n}{phase\PYZus{}plane\PYZus{}plot}\PY{p}{(}\PY{n}{left}\PY{p}{,} \PY{n}{right}\PY{p}{,} \PY{n}{approx\PYZus{}states}\PY{o}{=}\PY{n}{states}\PY{p}{[}\PY{l+s+s1}{\PYZsq{}}\PY{l+s+s1}{hlle}\PY{l+s+s1}{\PYZsq{}}\PY{p}{]}\PY{p}{)}
\end{Verbatim}
\end{tcolorbox}

    \begin{center}
    \adjustimage{max size={0.9\linewidth}{0.9\paperheight}}{Euler_approximate_files/Euler_approximate_30_0.pdf}
    \end{center}
    { \hspace*{\fill} \\}
    
    \hypertarget{preservation-of-positivity}{%
\subsubsection{Preservation of
positivity}\label{preservation-of-positivity}}

Just as we saw in the case of the shallow water equations, the Roe
solver (or any linearized solver) for the Euler equations fails to
preserve positivity of the pressure and/or density in some situations.
Here is one example.

    \begin{tcolorbox}[breakable, size=fbox, boxrule=1pt, pad at break*=1mm,colback=cellbackground, colframe=cellborder]
\prompt{In}{incolor}{15}{\boxspacing}
\begin{Verbatim}[commandchars=\\\{\}]
\PY{n}{left}  \PY{o}{=} \PY{n}{State}\PY{p}{(}\PY{n}{Density} \PY{o}{=} \PY{l+m+mf}{1.}\PY{p}{,}
              \PY{n}{Velocity} \PY{o}{=} \PY{o}{\PYZhy{}}\PY{l+m+mf}{5.}\PY{p}{,}
              \PY{n}{Pressure} \PY{o}{=} \PY{l+m+mf}{1.}\PY{p}{)}
\PY{n}{right} \PY{o}{=} \PY{n}{State}\PY{p}{(}\PY{n}{Density} \PY{o}{=} \PY{l+m+mf}{1.}\PY{p}{,}
              \PY{n}{Velocity} \PY{o}{=} \PY{l+m+mf}{1.}\PY{p}{,}
              \PY{n}{Pressure} \PY{o}{=} \PY{l+m+mf}{1.}\PY{p}{)}

\PY{n}{states} \PY{o}{=} \PY{n}{compare\PYZus{}solutions}\PY{p}{(}\PY{n}{left}\PY{p}{,} \PY{n}{right}\PY{p}{,} \PY{n}{solvers}\PY{o}{=}\PY{p}{[}\PY{l+s+s1}{\PYZsq{}}\PY{l+s+s1}{Exact}\PY{l+s+s1}{\PYZsq{}}\PY{p}{,} \PY{l+s+s1}{\PYZsq{}}\PY{l+s+s1}{Roe}\PY{l+s+s1}{\PYZsq{}}\PY{p}{]}\PY{p}{)}
\end{Verbatim}
\end{tcolorbox}

    
    \begin{Verbatim}[commandchars=\\\{\}]
interactive(children=(FloatSlider(value=0.4, description='t', max=0.9), Output()), \_dom\_classes=('widget-inter…
    \end{Verbatim}

    
    As we can see, in this example each Roe solver wave moves much more
slowly than the leading edge of the corresponding true rarefaction. In
order to maintain conservation, this implies that the middle Roe state
must have lower density than the true middle state. This leads to a
negative density. Note that the velocity and pressure take huge values
in the intermediate state.

The HLLE solver, on the other hand, guarantees positivity of the density
and pressure. Since the HLLE wave speed in the case of a rarefaction is
always the speed of the leading edge of the true rarefaction, and since
the HLLE solution is conservative, the density in a rarefaction will
always be at least as great as that of the true solution. This can be
seen clearly in the example below.

    \begin{tcolorbox}[breakable, size=fbox, boxrule=1pt, pad at break*=1mm,colback=cellbackground, colframe=cellborder]
\prompt{In}{incolor}{16}{\boxspacing}
\begin{Verbatim}[commandchars=\\\{\}]
\PY{n}{left}  \PY{o}{=} \PY{n}{State}\PY{p}{(}\PY{n}{Density} \PY{o}{=} \PY{l+m+mf}{1.}\PY{p}{,}
                    \PY{n}{Velocity} \PY{o}{=} \PY{o}{\PYZhy{}}\PY{l+m+mf}{10.}\PY{p}{,}
                    \PY{n}{Pressure} \PY{o}{=} \PY{l+m+mf}{1.}\PY{p}{)}
\PY{n}{right} \PY{o}{=} \PY{n}{State}\PY{p}{(}\PY{n}{Density} \PY{o}{=} \PY{l+m+mf}{1.}\PY{p}{,}
                    \PY{n}{Velocity} \PY{o}{=} \PY{l+m+mf}{1.}\PY{p}{,}
                    \PY{n}{Pressure} \PY{o}{=} \PY{l+m+mf}{1.}\PY{p}{)}

\PY{n}{states} \PY{o}{=} \PY{n}{compare\PYZus{}solutions}\PY{p}{(}\PY{n}{left}\PY{p}{,} \PY{n}{right}\PY{p}{,} \PY{n}{solvers}\PY{o}{=}\PY{p}{[}\PY{l+s+s1}{\PYZsq{}}\PY{l+s+s1}{Exact}\PY{l+s+s1}{\PYZsq{}}\PY{p}{,} \PY{l+s+s1}{\PYZsq{}}\PY{l+s+s1}{HLLE}\PY{l+s+s1}{\PYZsq{}}\PY{p}{]}\PY{p}{)}\PY{p}{;}
\end{Verbatim}
\end{tcolorbox}

    
    \begin{Verbatim}[commandchars=\\\{\}]
interactive(children=(FloatSlider(value=0.4, description='t', max=0.9), Output()), \_dom\_classes=('widget-inter…
    \end{Verbatim}

    
    \begin{tcolorbox}[breakable, size=fbox, boxrule=1pt, pad at break*=1mm,colback=cellbackground, colframe=cellborder]
\prompt{In}{incolor}{17}{\boxspacing}
\begin{Verbatim}[commandchars=\\\{\}]
\PY{n}{euler}\PY{o}{.}\PY{n}{phase\PYZus{}plane\PYZus{}plot}\PY{p}{(}\PY{n}{left}\PY{p}{,}\PY{n}{right}\PY{p}{,}\PY{n}{approx\PYZus{}states}\PY{o}{=}\PY{n}{states}\PY{p}{[}\PY{l+s+s1}{\PYZsq{}}\PY{l+s+s1}{hlle}\PY{l+s+s1}{\PYZsq{}}\PY{p}{]}\PY{p}{)}
\end{Verbatim}
\end{tcolorbox}

    \begin{center}
    \adjustimage{max size={0.9\linewidth}{0.9\paperheight}}{Euler_approximate_files/Euler_approximate_35_0.pdf}
    \end{center}
    { \hspace*{\fill} \\}
    
    Again recall that we are only considering a single Riemann solution in
this chapter. In \href{FV_compare.ipynb}{FV\_compare} we observe the
effect of using these approximate solvers in a full discretization.

    \begin{tcolorbox}[breakable, size=fbox, boxrule=1pt, pad at break*=1mm,colback=cellbackground, colframe=cellborder]
\prompt{In}{incolor}{ }{\boxspacing}
\begin{Verbatim}[commandchars=\\\{\}]

\end{Verbatim}
\end{tcolorbox}


    % Add a bibliography block to the postdoc
    
    
    
\end{document}
