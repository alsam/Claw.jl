\documentclass[11pt]{article}

    \usepackage[breakable]{tcolorbox}
    \usepackage{parskip} % Stop auto-indenting (to mimic markdown behaviour)
    
    \usepackage{iftex}
    \ifPDFTeX
    	\usepackage[T1]{fontenc}
    	\usepackage{mathpazo}
    \else
    	\usepackage{fontspec}
    \fi

    % Basic figure setup, for now with no caption control since it's done
    % automatically by Pandoc (which extracts ![](path) syntax from Markdown).
    \usepackage{graphicx}
    % Maintain compatibility with old templates. Remove in nbconvert 6.0
    \let\Oldincludegraphics\includegraphics
    % Ensure that by default, figures have no caption (until we provide a
    % proper Figure object with a Caption API and a way to capture that
    % in the conversion process - todo).
    \usepackage{caption}
    \DeclareCaptionFormat{nocaption}{}
    \captionsetup{format=nocaption,aboveskip=0pt,belowskip=0pt}

    \usepackage[Export]{adjustbox} % Used to constrain images to a maximum size
    \adjustboxset{max size={0.9\linewidth}{0.9\paperheight}}
    \usepackage{float}
    \floatplacement{figure}{H} % forces figures to be placed at the correct location
    \usepackage{xcolor} % Allow colors to be defined
    \usepackage{enumerate} % Needed for markdown enumerations to work
    \usepackage{geometry} % Used to adjust the document margins
    \usepackage{amsmath} % Equations
    \usepackage{amssymb} % Equations
    \usepackage{textcomp} % defines textquotesingle
    % Hack from http://tex.stackexchange.com/a/47451/13684:
    \AtBeginDocument{%
        \def\PYZsq{\textquotesingle}% Upright quotes in Pygmentized code
    }
    \usepackage{upquote} % Upright quotes for verbatim code
    \usepackage{eurosym} % defines \euro
    \usepackage[mathletters]{ucs} % Extended unicode (utf-8) support
    \usepackage{fancyvrb} % verbatim replacement that allows latex
    \usepackage{grffile} % extends the file name processing of package graphics 
                         % to support a larger range
    \makeatletter % fix for grffile with XeLaTeX
    \def\Gread@@xetex#1{%
      \IfFileExists{"\Gin@base".bb}%
      {\Gread@eps{\Gin@base.bb}}%
      {\Gread@@xetex@aux#1}%
    }
    \makeatother

    % The hyperref package gives us a pdf with properly built
    % internal navigation ('pdf bookmarks' for the table of contents,
    % internal cross-reference links, web links for URLs, etc.)
    \usepackage{hyperref}
    % The default LaTeX title has an obnoxious amount of whitespace. By default,
    % titling removes some of it. It also provides customization options.
    \usepackage{titling}
    \usepackage{longtable} % longtable support required by pandoc >1.10
    \usepackage{booktabs}  % table support for pandoc > 1.12.2
    \usepackage[inline]{enumitem} % IRkernel/repr support (it uses the enumerate* environment)
    \usepackage[normalem]{ulem} % ulem is needed to support strikethroughs (\sout)
                                % normalem makes italics be italics, not underlines
    \usepackage{mathrsfs}
    

    
    % Colors for the hyperref package
    \definecolor{urlcolor}{rgb}{0,.145,.698}
    \definecolor{linkcolor}{rgb}{.71,0.21,0.01}
    \definecolor{citecolor}{rgb}{.12,.54,.11}

    % ANSI colors
    \definecolor{ansi-black}{HTML}{3E424D}
    \definecolor{ansi-black-intense}{HTML}{282C36}
    \definecolor{ansi-red}{HTML}{E75C58}
    \definecolor{ansi-red-intense}{HTML}{B22B31}
    \definecolor{ansi-green}{HTML}{00A250}
    \definecolor{ansi-green-intense}{HTML}{007427}
    \definecolor{ansi-yellow}{HTML}{DDB62B}
    \definecolor{ansi-yellow-intense}{HTML}{B27D12}
    \definecolor{ansi-blue}{HTML}{208FFB}
    \definecolor{ansi-blue-intense}{HTML}{0065CA}
    \definecolor{ansi-magenta}{HTML}{D160C4}
    \definecolor{ansi-magenta-intense}{HTML}{A03196}
    \definecolor{ansi-cyan}{HTML}{60C6C8}
    \definecolor{ansi-cyan-intense}{HTML}{258F8F}
    \definecolor{ansi-white}{HTML}{C5C1B4}
    \definecolor{ansi-white-intense}{HTML}{A1A6B2}
    \definecolor{ansi-default-inverse-fg}{HTML}{FFFFFF}
    \definecolor{ansi-default-inverse-bg}{HTML}{000000}

    % commands and environments needed by pandoc snippets
    % extracted from the output of `pandoc -s`
    \providecommand{\tightlist}{%
      \setlength{\itemsep}{0pt}\setlength{\parskip}{0pt}}
    \DefineVerbatimEnvironment{Highlighting}{Verbatim}{commandchars=\\\{\}}
    % Add ',fontsize=\small' for more characters per line
    \newenvironment{Shaded}{}{}
    \newcommand{\KeywordTok}[1]{\textcolor[rgb]{0.00,0.44,0.13}{\textbf{{#1}}}}
    \newcommand{\DataTypeTok}[1]{\textcolor[rgb]{0.56,0.13,0.00}{{#1}}}
    \newcommand{\DecValTok}[1]{\textcolor[rgb]{0.25,0.63,0.44}{{#1}}}
    \newcommand{\BaseNTok}[1]{\textcolor[rgb]{0.25,0.63,0.44}{{#1}}}
    \newcommand{\FloatTok}[1]{\textcolor[rgb]{0.25,0.63,0.44}{{#1}}}
    \newcommand{\CharTok}[1]{\textcolor[rgb]{0.25,0.44,0.63}{{#1}}}
    \newcommand{\StringTok}[1]{\textcolor[rgb]{0.25,0.44,0.63}{{#1}}}
    \newcommand{\CommentTok}[1]{\textcolor[rgb]{0.38,0.63,0.69}{\textit{{#1}}}}
    \newcommand{\OtherTok}[1]{\textcolor[rgb]{0.00,0.44,0.13}{{#1}}}
    \newcommand{\AlertTok}[1]{\textcolor[rgb]{1.00,0.00,0.00}{\textbf{{#1}}}}
    \newcommand{\FunctionTok}[1]{\textcolor[rgb]{0.02,0.16,0.49}{{#1}}}
    \newcommand{\RegionMarkerTok}[1]{{#1}}
    \newcommand{\ErrorTok}[1]{\textcolor[rgb]{1.00,0.00,0.00}{\textbf{{#1}}}}
    \newcommand{\NormalTok}[1]{{#1}}
    
    % Additional commands for more recent versions of Pandoc
    \newcommand{\ConstantTok}[1]{\textcolor[rgb]{0.53,0.00,0.00}{{#1}}}
    \newcommand{\SpecialCharTok}[1]{\textcolor[rgb]{0.25,0.44,0.63}{{#1}}}
    \newcommand{\VerbatimStringTok}[1]{\textcolor[rgb]{0.25,0.44,0.63}{{#1}}}
    \newcommand{\SpecialStringTok}[1]{\textcolor[rgb]{0.73,0.40,0.53}{{#1}}}
    \newcommand{\ImportTok}[1]{{#1}}
    \newcommand{\DocumentationTok}[1]{\textcolor[rgb]{0.73,0.13,0.13}{\textit{{#1}}}}
    \newcommand{\AnnotationTok}[1]{\textcolor[rgb]{0.38,0.63,0.69}{\textbf{\textit{{#1}}}}}
    \newcommand{\CommentVarTok}[1]{\textcolor[rgb]{0.38,0.63,0.69}{\textbf{\textit{{#1}}}}}
    \newcommand{\VariableTok}[1]{\textcolor[rgb]{0.10,0.09,0.49}{{#1}}}
    \newcommand{\ControlFlowTok}[1]{\textcolor[rgb]{0.00,0.44,0.13}{\textbf{{#1}}}}
    \newcommand{\OperatorTok}[1]{\textcolor[rgb]{0.40,0.40,0.40}{{#1}}}
    \newcommand{\BuiltInTok}[1]{{#1}}
    \newcommand{\ExtensionTok}[1]{{#1}}
    \newcommand{\PreprocessorTok}[1]{\textcolor[rgb]{0.74,0.48,0.00}{{#1}}}
    \newcommand{\AttributeTok}[1]{\textcolor[rgb]{0.49,0.56,0.16}{{#1}}}
    \newcommand{\InformationTok}[1]{\textcolor[rgb]{0.38,0.63,0.69}{\textbf{\textit{{#1}}}}}
    \newcommand{\WarningTok}[1]{\textcolor[rgb]{0.38,0.63,0.69}{\textbf{\textit{{#1}}}}}
    
    
    % Define a nice break command that doesn't care if a line doesn't already
    % exist.
    \def\br{\hspace*{\fill} \\* }
    % Math Jax compatibility definitions
    \def\gt{>}
    \def\lt{<}
    \let\Oldtex\TeX
    \let\Oldlatex\LaTeX
    \renewcommand{\TeX}{\textrm{\Oldtex}}
    \renewcommand{\LaTeX}{\textrm{\Oldlatex}}
    % Document parameters
    % Document title
    \title{Acoustics}
    
    
    
    
    
% Pygments definitions
\makeatletter
\def\PY@reset{\let\PY@it=\relax \let\PY@bf=\relax%
    \let\PY@ul=\relax \let\PY@tc=\relax%
    \let\PY@bc=\relax \let\PY@ff=\relax}
\def\PY@tok#1{\csname PY@tok@#1\endcsname}
\def\PY@toks#1+{\ifx\relax#1\empty\else%
    \PY@tok{#1}\expandafter\PY@toks\fi}
\def\PY@do#1{\PY@bc{\PY@tc{\PY@ul{%
    \PY@it{\PY@bf{\PY@ff{#1}}}}}}}
\def\PY#1#2{\PY@reset\PY@toks#1+\relax+\PY@do{#2}}

\expandafter\def\csname PY@tok@w\endcsname{\def\PY@tc##1{\textcolor[rgb]{0.73,0.73,0.73}{##1}}}
\expandafter\def\csname PY@tok@c\endcsname{\let\PY@it=\textit\def\PY@tc##1{\textcolor[rgb]{0.25,0.50,0.50}{##1}}}
\expandafter\def\csname PY@tok@cp\endcsname{\def\PY@tc##1{\textcolor[rgb]{0.74,0.48,0.00}{##1}}}
\expandafter\def\csname PY@tok@k\endcsname{\let\PY@bf=\textbf\def\PY@tc##1{\textcolor[rgb]{0.00,0.50,0.00}{##1}}}
\expandafter\def\csname PY@tok@kp\endcsname{\def\PY@tc##1{\textcolor[rgb]{0.00,0.50,0.00}{##1}}}
\expandafter\def\csname PY@tok@kt\endcsname{\def\PY@tc##1{\textcolor[rgb]{0.69,0.00,0.25}{##1}}}
\expandafter\def\csname PY@tok@o\endcsname{\def\PY@tc##1{\textcolor[rgb]{0.40,0.40,0.40}{##1}}}
\expandafter\def\csname PY@tok@ow\endcsname{\let\PY@bf=\textbf\def\PY@tc##1{\textcolor[rgb]{0.67,0.13,1.00}{##1}}}
\expandafter\def\csname PY@tok@nb\endcsname{\def\PY@tc##1{\textcolor[rgb]{0.00,0.50,0.00}{##1}}}
\expandafter\def\csname PY@tok@nf\endcsname{\def\PY@tc##1{\textcolor[rgb]{0.00,0.00,1.00}{##1}}}
\expandafter\def\csname PY@tok@nc\endcsname{\let\PY@bf=\textbf\def\PY@tc##1{\textcolor[rgb]{0.00,0.00,1.00}{##1}}}
\expandafter\def\csname PY@tok@nn\endcsname{\let\PY@bf=\textbf\def\PY@tc##1{\textcolor[rgb]{0.00,0.00,1.00}{##1}}}
\expandafter\def\csname PY@tok@ne\endcsname{\let\PY@bf=\textbf\def\PY@tc##1{\textcolor[rgb]{0.82,0.25,0.23}{##1}}}
\expandafter\def\csname PY@tok@nv\endcsname{\def\PY@tc##1{\textcolor[rgb]{0.10,0.09,0.49}{##1}}}
\expandafter\def\csname PY@tok@no\endcsname{\def\PY@tc##1{\textcolor[rgb]{0.53,0.00,0.00}{##1}}}
\expandafter\def\csname PY@tok@nl\endcsname{\def\PY@tc##1{\textcolor[rgb]{0.63,0.63,0.00}{##1}}}
\expandafter\def\csname PY@tok@ni\endcsname{\let\PY@bf=\textbf\def\PY@tc##1{\textcolor[rgb]{0.60,0.60,0.60}{##1}}}
\expandafter\def\csname PY@tok@na\endcsname{\def\PY@tc##1{\textcolor[rgb]{0.49,0.56,0.16}{##1}}}
\expandafter\def\csname PY@tok@nt\endcsname{\let\PY@bf=\textbf\def\PY@tc##1{\textcolor[rgb]{0.00,0.50,0.00}{##1}}}
\expandafter\def\csname PY@tok@nd\endcsname{\def\PY@tc##1{\textcolor[rgb]{0.67,0.13,1.00}{##1}}}
\expandafter\def\csname PY@tok@s\endcsname{\def\PY@tc##1{\textcolor[rgb]{0.73,0.13,0.13}{##1}}}
\expandafter\def\csname PY@tok@sd\endcsname{\let\PY@it=\textit\def\PY@tc##1{\textcolor[rgb]{0.73,0.13,0.13}{##1}}}
\expandafter\def\csname PY@tok@si\endcsname{\let\PY@bf=\textbf\def\PY@tc##1{\textcolor[rgb]{0.73,0.40,0.53}{##1}}}
\expandafter\def\csname PY@tok@se\endcsname{\let\PY@bf=\textbf\def\PY@tc##1{\textcolor[rgb]{0.73,0.40,0.13}{##1}}}
\expandafter\def\csname PY@tok@sr\endcsname{\def\PY@tc##1{\textcolor[rgb]{0.73,0.40,0.53}{##1}}}
\expandafter\def\csname PY@tok@ss\endcsname{\def\PY@tc##1{\textcolor[rgb]{0.10,0.09,0.49}{##1}}}
\expandafter\def\csname PY@tok@sx\endcsname{\def\PY@tc##1{\textcolor[rgb]{0.00,0.50,0.00}{##1}}}
\expandafter\def\csname PY@tok@m\endcsname{\def\PY@tc##1{\textcolor[rgb]{0.40,0.40,0.40}{##1}}}
\expandafter\def\csname PY@tok@gh\endcsname{\let\PY@bf=\textbf\def\PY@tc##1{\textcolor[rgb]{0.00,0.00,0.50}{##1}}}
\expandafter\def\csname PY@tok@gu\endcsname{\let\PY@bf=\textbf\def\PY@tc##1{\textcolor[rgb]{0.50,0.00,0.50}{##1}}}
\expandafter\def\csname PY@tok@gd\endcsname{\def\PY@tc##1{\textcolor[rgb]{0.63,0.00,0.00}{##1}}}
\expandafter\def\csname PY@tok@gi\endcsname{\def\PY@tc##1{\textcolor[rgb]{0.00,0.63,0.00}{##1}}}
\expandafter\def\csname PY@tok@gr\endcsname{\def\PY@tc##1{\textcolor[rgb]{1.00,0.00,0.00}{##1}}}
\expandafter\def\csname PY@tok@ge\endcsname{\let\PY@it=\textit}
\expandafter\def\csname PY@tok@gs\endcsname{\let\PY@bf=\textbf}
\expandafter\def\csname PY@tok@gp\endcsname{\let\PY@bf=\textbf\def\PY@tc##1{\textcolor[rgb]{0.00,0.00,0.50}{##1}}}
\expandafter\def\csname PY@tok@go\endcsname{\def\PY@tc##1{\textcolor[rgb]{0.53,0.53,0.53}{##1}}}
\expandafter\def\csname PY@tok@gt\endcsname{\def\PY@tc##1{\textcolor[rgb]{0.00,0.27,0.87}{##1}}}
\expandafter\def\csname PY@tok@err\endcsname{\def\PY@bc##1{\setlength{\fboxsep}{0pt}\fcolorbox[rgb]{1.00,0.00,0.00}{1,1,1}{\strut ##1}}}
\expandafter\def\csname PY@tok@kc\endcsname{\let\PY@bf=\textbf\def\PY@tc##1{\textcolor[rgb]{0.00,0.50,0.00}{##1}}}
\expandafter\def\csname PY@tok@kd\endcsname{\let\PY@bf=\textbf\def\PY@tc##1{\textcolor[rgb]{0.00,0.50,0.00}{##1}}}
\expandafter\def\csname PY@tok@kn\endcsname{\let\PY@bf=\textbf\def\PY@tc##1{\textcolor[rgb]{0.00,0.50,0.00}{##1}}}
\expandafter\def\csname PY@tok@kr\endcsname{\let\PY@bf=\textbf\def\PY@tc##1{\textcolor[rgb]{0.00,0.50,0.00}{##1}}}
\expandafter\def\csname PY@tok@bp\endcsname{\def\PY@tc##1{\textcolor[rgb]{0.00,0.50,0.00}{##1}}}
\expandafter\def\csname PY@tok@fm\endcsname{\def\PY@tc##1{\textcolor[rgb]{0.00,0.00,1.00}{##1}}}
\expandafter\def\csname PY@tok@vc\endcsname{\def\PY@tc##1{\textcolor[rgb]{0.10,0.09,0.49}{##1}}}
\expandafter\def\csname PY@tok@vg\endcsname{\def\PY@tc##1{\textcolor[rgb]{0.10,0.09,0.49}{##1}}}
\expandafter\def\csname PY@tok@vi\endcsname{\def\PY@tc##1{\textcolor[rgb]{0.10,0.09,0.49}{##1}}}
\expandafter\def\csname PY@tok@vm\endcsname{\def\PY@tc##1{\textcolor[rgb]{0.10,0.09,0.49}{##1}}}
\expandafter\def\csname PY@tok@sa\endcsname{\def\PY@tc##1{\textcolor[rgb]{0.73,0.13,0.13}{##1}}}
\expandafter\def\csname PY@tok@sb\endcsname{\def\PY@tc##1{\textcolor[rgb]{0.73,0.13,0.13}{##1}}}
\expandafter\def\csname PY@tok@sc\endcsname{\def\PY@tc##1{\textcolor[rgb]{0.73,0.13,0.13}{##1}}}
\expandafter\def\csname PY@tok@dl\endcsname{\def\PY@tc##1{\textcolor[rgb]{0.73,0.13,0.13}{##1}}}
\expandafter\def\csname PY@tok@s2\endcsname{\def\PY@tc##1{\textcolor[rgb]{0.73,0.13,0.13}{##1}}}
\expandafter\def\csname PY@tok@sh\endcsname{\def\PY@tc##1{\textcolor[rgb]{0.73,0.13,0.13}{##1}}}
\expandafter\def\csname PY@tok@s1\endcsname{\def\PY@tc##1{\textcolor[rgb]{0.73,0.13,0.13}{##1}}}
\expandafter\def\csname PY@tok@mb\endcsname{\def\PY@tc##1{\textcolor[rgb]{0.40,0.40,0.40}{##1}}}
\expandafter\def\csname PY@tok@mf\endcsname{\def\PY@tc##1{\textcolor[rgb]{0.40,0.40,0.40}{##1}}}
\expandafter\def\csname PY@tok@mh\endcsname{\def\PY@tc##1{\textcolor[rgb]{0.40,0.40,0.40}{##1}}}
\expandafter\def\csname PY@tok@mi\endcsname{\def\PY@tc##1{\textcolor[rgb]{0.40,0.40,0.40}{##1}}}
\expandafter\def\csname PY@tok@il\endcsname{\def\PY@tc##1{\textcolor[rgb]{0.40,0.40,0.40}{##1}}}
\expandafter\def\csname PY@tok@mo\endcsname{\def\PY@tc##1{\textcolor[rgb]{0.40,0.40,0.40}{##1}}}
\expandafter\def\csname PY@tok@ch\endcsname{\let\PY@it=\textit\def\PY@tc##1{\textcolor[rgb]{0.25,0.50,0.50}{##1}}}
\expandafter\def\csname PY@tok@cm\endcsname{\let\PY@it=\textit\def\PY@tc##1{\textcolor[rgb]{0.25,0.50,0.50}{##1}}}
\expandafter\def\csname PY@tok@cpf\endcsname{\let\PY@it=\textit\def\PY@tc##1{\textcolor[rgb]{0.25,0.50,0.50}{##1}}}
\expandafter\def\csname PY@tok@c1\endcsname{\let\PY@it=\textit\def\PY@tc##1{\textcolor[rgb]{0.25,0.50,0.50}{##1}}}
\expandafter\def\csname PY@tok@cs\endcsname{\let\PY@it=\textit\def\PY@tc##1{\textcolor[rgb]{0.25,0.50,0.50}{##1}}}

\def\PYZbs{\char`\\}
\def\PYZus{\char`\_}
\def\PYZob{\char`\{}
\def\PYZcb{\char`\}}
\def\PYZca{\char`\^}
\def\PYZam{\char`\&}
\def\PYZlt{\char`\<}
\def\PYZgt{\char`\>}
\def\PYZsh{\char`\#}
\def\PYZpc{\char`\%}
\def\PYZdl{\char`\$}
\def\PYZhy{\char`\-}
\def\PYZsq{\char`\'}
\def\PYZdq{\char`\"}
\def\PYZti{\char`\~}
% for compatibility with earlier versions
\def\PYZat{@}
\def\PYZlb{[}
\def\PYZrb{]}
\makeatother


    % For linebreaks inside Verbatim environment from package fancyvrb. 
    \makeatletter
        \newbox\Wrappedcontinuationbox 
        \newbox\Wrappedvisiblespacebox 
        \newcommand*\Wrappedvisiblespace {\textcolor{red}{\textvisiblespace}} 
        \newcommand*\Wrappedcontinuationsymbol {\textcolor{red}{\llap{\tiny$\m@th\hookrightarrow$}}} 
        \newcommand*\Wrappedcontinuationindent {3ex } 
        \newcommand*\Wrappedafterbreak {\kern\Wrappedcontinuationindent\copy\Wrappedcontinuationbox} 
        % Take advantage of the already applied Pygments mark-up to insert 
        % potential linebreaks for TeX processing. 
        %        {, <, #, %, $, ' and ": go to next line. 
        %        _, }, ^, &, >, - and ~: stay at end of broken line. 
        % Use of \textquotesingle for straight quote. 
        \newcommand*\Wrappedbreaksatspecials {% 
            \def\PYGZus{\discretionary{\char`\_}{\Wrappedafterbreak}{\char`\_}}% 
            \def\PYGZob{\discretionary{}{\Wrappedafterbreak\char`\{}{\char`\{}}% 
            \def\PYGZcb{\discretionary{\char`\}}{\Wrappedafterbreak}{\char`\}}}% 
            \def\PYGZca{\discretionary{\char`\^}{\Wrappedafterbreak}{\char`\^}}% 
            \def\PYGZam{\discretionary{\char`\&}{\Wrappedafterbreak}{\char`\&}}% 
            \def\PYGZlt{\discretionary{}{\Wrappedafterbreak\char`\<}{\char`\<}}% 
            \def\PYGZgt{\discretionary{\char`\>}{\Wrappedafterbreak}{\char`\>}}% 
            \def\PYGZsh{\discretionary{}{\Wrappedafterbreak\char`\#}{\char`\#}}% 
            \def\PYGZpc{\discretionary{}{\Wrappedafterbreak\char`\%}{\char`\%}}% 
            \def\PYGZdl{\discretionary{}{\Wrappedafterbreak\char`\$}{\char`\$}}% 
            \def\PYGZhy{\discretionary{\char`\-}{\Wrappedafterbreak}{\char`\-}}% 
            \def\PYGZsq{\discretionary{}{\Wrappedafterbreak\textquotesingle}{\textquotesingle}}% 
            \def\PYGZdq{\discretionary{}{\Wrappedafterbreak\char`\"}{\char`\"}}% 
            \def\PYGZti{\discretionary{\char`\~}{\Wrappedafterbreak}{\char`\~}}% 
        } 
        % Some characters . , ; ? ! / are not pygmentized. 
        % This macro makes them "active" and they will insert potential linebreaks 
        \newcommand*\Wrappedbreaksatpunct {% 
            \lccode`\~`\.\lowercase{\def~}{\discretionary{\hbox{\char`\.}}{\Wrappedafterbreak}{\hbox{\char`\.}}}% 
            \lccode`\~`\,\lowercase{\def~}{\discretionary{\hbox{\char`\,}}{\Wrappedafterbreak}{\hbox{\char`\,}}}% 
            \lccode`\~`\;\lowercase{\def~}{\discretionary{\hbox{\char`\;}}{\Wrappedafterbreak}{\hbox{\char`\;}}}% 
            \lccode`\~`\:\lowercase{\def~}{\discretionary{\hbox{\char`\:}}{\Wrappedafterbreak}{\hbox{\char`\:}}}% 
            \lccode`\~`\?\lowercase{\def~}{\discretionary{\hbox{\char`\?}}{\Wrappedafterbreak}{\hbox{\char`\?}}}% 
            \lccode`\~`\!\lowercase{\def~}{\discretionary{\hbox{\char`\!}}{\Wrappedafterbreak}{\hbox{\char`\!}}}% 
            \lccode`\~`\/\lowercase{\def~}{\discretionary{\hbox{\char`\/}}{\Wrappedafterbreak}{\hbox{\char`\/}}}% 
            \catcode`\.\active
            \catcode`\,\active 
            \catcode`\;\active
            \catcode`\:\active
            \catcode`\?\active
            \catcode`\!\active
            \catcode`\/\active 
            \lccode`\~`\~ 	
        }
    \makeatother

    \let\OriginalVerbatim=\Verbatim
    \makeatletter
    \renewcommand{\Verbatim}[1][1]{%
        %\parskip\z@skip
        \sbox\Wrappedcontinuationbox {\Wrappedcontinuationsymbol}%
        \sbox\Wrappedvisiblespacebox {\FV@SetupFont\Wrappedvisiblespace}%
        \def\FancyVerbFormatLine ##1{\hsize\linewidth
            \vtop{\raggedright\hyphenpenalty\z@\exhyphenpenalty\z@
                \doublehyphendemerits\z@\finalhyphendemerits\z@
                \strut ##1\strut}%
        }%
        % If the linebreak is at a space, the latter will be displayed as visible
        % space at end of first line, and a continuation symbol starts next line.
        % Stretch/shrink are however usually zero for typewriter font.
        \def\FV@Space {%
            \nobreak\hskip\z@ plus\fontdimen3\font minus\fontdimen4\font
            \discretionary{\copy\Wrappedvisiblespacebox}{\Wrappedafterbreak}
            {\kern\fontdimen2\font}%
        }%
        
        % Allow breaks at special characters using \PYG... macros.
        \Wrappedbreaksatspecials
        % Breaks at punctuation characters . , ; ? ! and / need catcode=\active 	
        \OriginalVerbatim[#1,codes*=\Wrappedbreaksatpunct]%
    }
    \makeatother

    % Exact colors from NB
    \definecolor{incolor}{HTML}{303F9F}
    \definecolor{outcolor}{HTML}{D84315}
    \definecolor{cellborder}{HTML}{CFCFCF}
    \definecolor{cellbackground}{HTML}{F7F7F7}
    
    % prompt
    \makeatletter
    \newcommand{\boxspacing}{\kern\kvtcb@left@rule\kern\kvtcb@boxsep}
    \makeatother
    \newcommand{\prompt}[4]{
        \ttfamily\llap{{\color{#2}[#3]:\hspace{3pt}#4}}\vspace{-\baselineskip}
    }
    

    
    % Prevent overflowing lines due to hard-to-break entities
    \sloppy 
    % Setup hyperref package
    \hypersetup{
      breaklinks=true,  % so long urls are correctly broken across lines
      colorlinks=true,
      urlcolor=urlcolor,
      linkcolor=linkcolor,
      citecolor=citecolor,
      }
    % Slightly bigger margins than the latex defaults
    
    \geometry{verbose,tmargin=1in,bmargin=1in,lmargin=1in,rmargin=1in}
    
    

\begin{document}
    
    \maketitle
    
    

    
    \hypertarget{acoustics}{%
\section{Acoustics}\label{acoustics}}

    \begin{tcolorbox}[breakable, size=fbox, boxrule=1pt, pad at break*=1mm,colback=cellbackground, colframe=cellborder]
\prompt{In}{incolor}{1}{\boxspacing}
\begin{Verbatim}[commandchars=\\\{\}]
\PY{o}{\PYZpc{}}\PY{k}{matplotlib} inline
\end{Verbatim}
\end{tcolorbox}

    \begin{tcolorbox}[breakable, size=fbox, boxrule=1pt, pad at break*=1mm,colback=cellbackground, colframe=cellborder]
\prompt{In}{incolor}{2}{\boxspacing}
\begin{Verbatim}[commandchars=\\\{\}]
\PY{o}{\PYZpc{}}\PY{k}{config} InlineBackend.figure\PYZus{}format = \PYZsq{}svg\PYZsq{}
\PY{k+kn}{import} \PY{n+nn}{numpy} \PY{k}{as} \PY{n+nn}{np}
\PY{k+kn}{from} \PY{n+nn}{exact\PYZus{}solvers} \PY{k}{import} \PY{n}{acoustics}\PY{p}{,} \PY{n}{acoustics\PYZus{}demos}
\PY{k+kn}{from} \PY{n+nn}{IPython}\PY{n+nn}{.}\PY{n+nn}{display} \PY{k}{import} \PY{n}{IFrame}\PY{p}{,} \PY{n}{HTML}\PY{p}{,} \PY{n}{Image}
\end{Verbatim}
\end{tcolorbox}

    In this chapter we consider our first \emph{system} of hyperbolic
conservation laws. We study the acoustics equations that were introduced
briefly in \href{Introduction.ipynb}{Introduction}. We first describe
the physical context of this system and then investigate its
characteristic structure and the solution to the Riemann problem. This
system is described in more detail in Chapter 3 of \cite{fvmhp}.

    If you wish to examine the Python code for this chapter, please see:

\begin{itemize}
\tightlist
\item
  \url{exact_solvers/acoustics.py} \ldots{}
  \href{https://github.com/clawpack/riemann_book/blob/FA16/exact_solvers/acoustics.py}{on
  github,}
\item
  \url{exact_solvers/acoustics_demos.py} \ldots{}
  \href{https://github.com/clawpack/riemann_book/blob/FA16/exact_solvers/acoustics_demos.py}{on
  github.}
\end{itemize}

    \hypertarget{physical-setting}{%
\subsection{Physical setting}\label{physical-setting}}

The linear acoustic equations describe the propagation of small
perturbations in a fluid. In \href{Advection.ipynb}{Advection} we
derived the one-dimensional continuity equation, which describes mass
conservation:\\
\begin{align} \label{Ac:continuity}
    \rho_t + (\rho u)_x & = 0.
\end{align}\\
For more realistic fluid models, we need another equation that
determines the velocity \(u\). This typically takes the form of a
conservation law for the momentum \(\rho u\). Momentum, like density, is
transported by fluid motion with corresponding flux \(\rho u^2\).
Additionally, any difference in pressure will also lead to a flux of
momentum that is proportional to the pressure difference. Thus the
momentum equation takes the form\\
\begin{align} \label{Ac:mom_cons}
(\rho u)_t + (\rho u^2 + P(\rho))_x & = 0,
\end{align}\\
where the pressure \(P\) is is given by the equation of state
\(P(\rho)\); here we have assumed the pressure depends only on the
density. A more general equation of state will be considered, along with
fully nonlinear fluid motions, in \href{Euler.ipynb}{Euler}. The linear
acoustics equations focus on the behavior of small perturbations in the
system above.

    In order to derive the equations of linear acoustics, observe that
equations (\ref{Ac:continuity})-(\ref{Ac:mom_cons}) form a hyperbolic
system \(q_t+f(q)_x=0\) with\\
\begin{align*}
q & = \begin{bmatrix} \rho \\ \rho u \end{bmatrix} & 
f(q) & = \begin{bmatrix} \rho u \\ \rho u^2 + P(\rho) \end{bmatrix}
\end{align*}\\
We will make use of the quasilinear form of a hyperbolic system:
\[q_t + f'(q) q_x = 0.\]\\
Here \(f'(q)\) denotes the Jacobian of the flux \(f\) with respect to
the conserved variables \(q\). In the present system, as is often the
case, \(f\) is most naturally written in terms of so-called primitive
variables (in this case \(\rho\) and \(u\)) rather than in terms of the
conserved variables \(q\). In order to find the flux Jacobian (and thus
the quasilinear form), we first write \(f\) in terms of the conserved
variables \((q_1,q_2) = (\rho, \rho u)\):\\
\begin{align}
f(q) & = \begin{bmatrix} q_2 \\ q_2^2/q_1 + P(q_1) \end{bmatrix}.
\end{align}

    Now we can differentiate to find the flux Jacobian:\\
\begin{align*}
f'(q) & = \begin{bmatrix} \partial f_1/\partial q_1 & \partial f_1/\partial q_2 \\
                          \partial f_2/\partial q_1 & \partial f_2/\partial q_2 \end{bmatrix} \\
      & = \begin{bmatrix} 0 & 1 \\ -q_2^2/q_1^2 + P'(q_1) & 2q_2/q_1 \end{bmatrix} \\
      & = \begin{bmatrix} 0 & 1 \\ P'(\rho)-u^2 & 2u \end{bmatrix}.
\end{align*}

Thus small perturbations to an ambient fluid state \(\rho_0, u_0\)
evolve according to the linearized equations \(q_t + f'(q_0) q_x = 0\),
or more explicitly \begin{align*}
\rho_t + (\rho u)_x & = 0 \\
(\rho u)_t + (P'(\rho_0)-u_0^2)\rho_x + 2u_0(\rho u)_x & = 0.
\end{align*}\\
As we are only interested in small perturbations of equation
(\ref{Ac:mom_cons}), we expand the perturbations \(\rho-\rho_0\) and
\(\rho u - \rho_0 u_0\) as functions of a small parameter \(\epsilon\),
and then we discard terms of order \(\epsilon^2\) and higher. This
results in the linear hyperbolic system\\
\begin{align*}
p_t + u_0 p_x + P'(\rho_0) u_x & = 0 \\
u_t + \frac{1}{\rho_0} p_x + u_0 u_x & = 0,
\end{align*} where \(p(x,t)\) is the pressure as a function of \(x\) and
\(t\). If the ambient fluid is at rest (i.e.~\(u_0=0\)) and the pressure
is directly proportional to the density, then this simplifies to
\begin{align} \label{Ac:main}
 \left[ \begin{array}{c}
p \\
u 
\end{array} \right]_t +  \underbrace{\left[ \begin{array}{cc}
0 & K_0 \\
1/\rho_0 & 0  \\
\end{array} \right]}_{\mathbf{A}}
\left[ \begin{array}{c}
p \\
u \end{array} \right]_x = 0,
\end{align} where \(K_0=P'(\rho_0)\) is referred to as the bulk modulus
of compressibility. The system of equations (\ref{Ac:main}) is called
the linear acoustics equations.

    For the rest of this chapter we work with (\ref{Ac:main}) and let
\(q=[p,u]^T\). Then we can write (\ref{Ac:main}) as \(q_t + A q_x = 0\).
For simplicity, we also drop the subscripts on \(K, \rho\). Direct
calculation reveals that the eigenvectors of \(A\) are \begin{align}
\lambda_1 = -c, \qquad \lambda_2 = c
\end{align} where \(c=\sqrt{{K}/{\rho}}\) is the speed of sound in a
medium with a given density and bulk modulus. The right eigenvectors of
\(A\) are given by \begin{align*}
r_1 = \begin{bmatrix}\begin{array}{c}-Z\\1\end{array}\end{bmatrix}, \qquad r_2 = \begin{bmatrix}\begin{array}{c}Z\\1\end{array}\end{bmatrix},
\end{align*} where \(Z=\rho c\) is called the acoustic impedance.
Defining \(R = [r_1 r_2]\) and \(\Lambda = diag(\lambda_1, \lambda_2)\),
we have \(AR = R\Lambda\), or \(A = R \Lambda R^{-1}\). Substituting
this into (\ref{Ac:main}) yields \begin{align*}
q_t + A q_x & = 0 \\
q_t + R \Lambda R^{-1} q_x & = 0 \\
R^{-1}q_t + \Lambda R^{-1} q_x & = 0 \\
w_t + \Lambda w_x & = 0,
\end{align*} where we have introduced the \emph{characteristic
variables} \(w=R^{-1}q\). The last system above is simply a pair of
decoupled advection equations for \(w_1\) and \(w_2\), with velocities
\(\lambda_1\) and \(\lambda_2\); a system we already know how to solve.
Thus we see that the eigenvalues of \(A\) are the velocities at which
information propagates in the solution.

    \hypertarget{solution-by-characteristics}{%
\subsection{Solution by
characteristics}\label{solution-by-characteristics}}

    The discussion above suggests a strategy for solving the Cauchy problem:

\begin{enumerate}
\def\labelenumi{\arabic{enumi}.}
\tightlist
\item
  Decompose the initial data \((p(x,0), u(x,0))\) into characteristic
  variables \(w(x,0)=(w_1^0(x),w_2^0(x,0))\) using the relation
  \(w = R^{-1}q\).
\item
  Evolve the characteristic variables:
  \(w_p(x,t) = w_p^0(x-\lambda_p t)\).
\item
  Transform back to the physical variables: \(q = Rw\).
\end{enumerate}

The first step in this process amounts to expressing the vector \(q\) in
the basis given by \(r_1, r_2\). Solving the system \(Rw=q\) yields
\begin{align*}
q = w_1 r_1 + w_2 r_2,
\end{align*} where \begin{align*}
w_1 = \frac{- p + Z u}{2Z}, \ \ \ \ \ \
w_2 = \frac{ p + Z u}{2Z}.
\end{align*}

We visualize this below, where the first plot shows the two
eigenvectors, and the second plot shows how \(q\) can be expressed as a
linear combination of the two eigenvectors, \(r_1\) and \(r_2\).
\emph{In the live notebook you can adjust the left and right states or
the material parameters to see how this affects the construction of the
Riemann solution.}

    \begin{tcolorbox}[breakable, size=fbox, boxrule=1pt, pad at break*=1mm,colback=cellbackground, colframe=cellborder]
\prompt{In}{incolor}{3}{\boxspacing}
\begin{Verbatim}[commandchars=\\\{\}]
\PY{o}{\PYZpc{}}\PY{k}{matplotlib} inline
\end{Verbatim}
\end{tcolorbox}

    \begin{tcolorbox}[breakable, size=fbox, boxrule=1pt, pad at break*=1mm,colback=cellbackground, colframe=cellborder]
\prompt{In}{incolor}{4}{\boxspacing}
\begin{Verbatim}[commandchars=\\\{\}]
\PY{o}{\PYZpc{}}\PY{k}{config} InlineBackend.figure\PYZus{}format = \PYZsq{}svg\PYZsq{}
\PY{k+kn}{import} \PY{n+nn}{numpy} \PY{k}{as} \PY{n+nn}{np}
\PY{k+kn}{from} \PY{n+nn}{exact\PYZus{}solvers} \PY{k}{import} \PY{n}{acoustics}\PY{p}{,} \PY{n}{acoustics\PYZus{}demos}
\PY{k+kn}{from} \PY{n+nn}{IPython}\PY{n+nn}{.}\PY{n+nn}{display} \PY{k}{import} \PY{n}{IFrame}
\end{Verbatim}
\end{tcolorbox}

    \begin{tcolorbox}[breakable, size=fbox, boxrule=1pt, pad at break*=1mm,colback=cellbackground, colframe=cellborder]
\prompt{In}{incolor}{5}{\boxspacing}
\begin{Verbatim}[commandchars=\\\{\}]
\PY{n}{acoustics\PYZus{}demos}\PY{o}{.}\PY{n}{decompose\PYZus{}q\PYZus{}interactive}\PY{p}{(}\PY{p}{)}
\end{Verbatim}
\end{tcolorbox}

    
    \begin{verbatim}
interactive(children=(FloatSlider(value=1.0, description='p', max=1.0, min=-1.0), FloatSlider(value=0.3, descr…
    \end{verbatim}

    
    
    \begin{verbatim}
VBox(children=(HBox(children=(FloatSlider(value=1.0, description='p', max=1.0, min=-1.0), FloatSlider(value=1.…
    \end{verbatim}

    
    
    \begin{verbatim}
Output()
    \end{verbatim}

    
    In the second and third steps, we evolve the characteristic variables
\(w\) and then transform back to the original variables. We take as
initial pressure a Gaussian, with zero initial velocity. We visualize
this below, where the time evolution in the characteristic variables is
shown in the first plot, and the time evolution of the velocity is shown
in the second plot.

    \begin{tcolorbox}[breakable, size=fbox, boxrule=1pt, pad at break*=1mm,colback=cellbackground, colframe=cellborder]
\prompt{In}{incolor}{6}{\boxspacing}
\begin{Verbatim}[commandchars=\\\{\}]
\PY{n}{acoustics\PYZus{}demos}\PY{o}{.}\PY{n}{char\PYZus{}solution\PYZus{}interactive}\PY{p}{(}\PY{p}{)}
\end{Verbatim}
\end{tcolorbox}

    
    \begin{verbatim}
interactive(children=(FloatSlider(value=0.0, description='t', max=1.2), FloatSlider(value=1.0, description='K'…
    \end{verbatim}

    
    
    \begin{verbatim}
HBox(children=(VBox(children=(FloatSlider(value=0.0, description='t', max=1.2),)), VBox(children=(FloatSlider(…
    \end{verbatim}

    
    
    \begin{verbatim}
Output()
    \end{verbatim}

    
    \emph{In the live notebook, you can advance the above solutions in time
and select which of the two characteristic variables to display.} Notice
how in the characteristic variables \(w\) (plotted on the left), each
part of the solution simply advects (translates) since each of the
characteristics variables simply obeys an uncoupled advection equation.

    \hypertarget{the-riemann-problem}{%
\subsection{The Riemann problem}\label{the-riemann-problem}}

Now that we know how to solve the Cauchy problem, solution of the
Riemann problem is merely a special case. We have the special initial
data\\
\begin{align*}
q(x,0) = \begin{cases}
q_\ell & \text{if   } x \le 0, \\
q_r & \text{if   } x > 0.
\end{cases}
\end{align*}\\
We can proceed as before, by decomposing into characteristic components,
advecting, and then transforming back. But since we know the solution
will be constant almost everywhere, it's even simpler to just decompose
the jump \(\Delta q = q_r - q_\ell\) in terms of the characteristic
variables, and advect the two resulting jumps \(\Delta w_1\) and
\(\Delta w_2\):\\
\begin{align*}
\Delta q = \Delta w_1 r_1 + \Delta w_2 r_2,
\end{align*}\\
Since \(R\Delta w = \Delta q\), we have\\
\begin{align*}
\Delta w_1 = \frac{-\Delta p + Z\Delta u}{2Z}, \ \ \ \ \ \
\Delta w_2 = \frac{\Delta p + Z\Delta u}{2Z}.
\end{align*}\\
Thus the solution has the structure depicted below.

    \begin{tcolorbox}[breakable, size=fbox, boxrule=1pt, pad at break*=1mm,colback=cellbackground, colframe=cellborder]
\prompt{In}{incolor}{7}{\boxspacing}
\begin{Verbatim}[commandchars=\\\{\}]
\PY{n}{Image}\PY{p}{(}\PY{l+s+s1}{\PYZsq{}}\PY{l+s+s1}{figures/acoustics\PYZus{}xt\PYZus{}plane.png}\PY{l+s+s1}{\PYZsq{}}\PY{p}{,} \PY{n}{width}\PY{o}{=}\PY{l+m+mi}{350}\PY{p}{)}
\end{Verbatim}
\end{tcolorbox}
 
            
\prompt{Out}{outcolor}{7}{}
    
    \begin{center}
    \adjustimage{max size={0.9\linewidth}{0.9\paperheight}}{output_18_0.png}
    \end{center}
    { \hspace*{\fill} \\}
    

    The three constant states are related by the jumps:\\
\begin{align}
q_m = q_\ell + \Delta w_1 r_1 = q_r - \Delta w_2 r_2.
\label{eq:acussol}
\end{align}\\
The jumps in pressure and velocity for each propagating discontinuity
are related in a particular way, since each jump is a multiple of one of
the eigenvectors of \(A\). More generally, the eigenvectors of the
coefficient matrix of a linear hyperbolic system reveal the relation
between jumps in the conserved variables across a wave propagating with
speed given by the corresponding eigenvalue. For acoustics, the
impedance is the physical parameter that determines this relation.

    \hypertarget{a-simple-solution}{%
\subsubsection{A simple solution}\label{a-simple-solution}}

Here we provide some very simple initial data, and determine the Riemann
solution, which consists of three states \(q_\ell\), \(q_m\) and
\(q_r\), and the speeds of the two waves.

    \begin{tcolorbox}[breakable, size=fbox, boxrule=1pt, pad at break*=1mm,colback=cellbackground, colframe=cellborder]
\prompt{In}{incolor}{8}{\boxspacing}
\begin{Verbatim}[commandchars=\\\{\}]
\PY{c+c1}{\PYZsh{} Initial data for Riemann problem}
\PY{n}{rho} \PY{o}{=} \PY{l+m+mf}{0.5}               \PY{c+c1}{\PYZsh{} density}
\PY{n}{bulk} \PY{o}{=} \PY{l+m+mf}{2.}            \PY{c+c1}{\PYZsh{} bulk modulus}
\PY{n}{ql} \PY{o}{=} \PY{n}{np}\PY{o}{.}\PY{n}{array}\PY{p}{(}\PY{p}{[}\PY{l+m+mi}{3}\PY{p}{,}\PY{l+m+mi}{2}\PY{p}{]}\PY{p}{)}   \PY{c+c1}{\PYZsh{} Left state}
\PY{n}{qr} \PY{o}{=} \PY{n}{np}\PY{o}{.}\PY{n}{array}\PY{p}{(}\PY{p}{[}\PY{l+m+mi}{3}\PY{p}{,}\PY{o}{\PYZhy{}}\PY{l+m+mi}{2}\PY{p}{]}\PY{p}{)}  \PY{c+c1}{\PYZsh{} Right state}
\PY{c+c1}{\PYZsh{} Calculated parameters}
\PY{n}{c} \PY{o}{=} \PY{n}{np}\PY{o}{.}\PY{n}{sqrt}\PY{p}{(}\PY{n}{bulk}\PY{o}{/}\PY{n}{rho}\PY{p}{)}  \PY{c+c1}{\PYZsh{} calculate sound speed}
\PY{n}{Z} \PY{o}{=} \PY{n}{np}\PY{o}{.}\PY{n}{sqrt}\PY{p}{(}\PY{n}{bulk}\PY{o}{*}\PY{n}{rho}\PY{p}{)}  \PY{c+c1}{\PYZsh{} calculate impedance}
\PY{n+nb}{print}\PY{p}{(}\PY{l+s+s2}{\PYZdq{}}\PY{l+s+s2}{With density rho = }\PY{l+s+si}{\PYZpc{}g}\PY{l+s+s2}{,  bulk modulus K = }\PY{l+s+si}{\PYZpc{}g}\PY{l+s+s2}{\PYZdq{}} \PYZbs{}
      \PY{o}{\PYZpc{}} \PY{p}{(}\PY{n}{rho}\PY{p}{,}\PY{n}{bulk}\PY{p}{)}\PY{p}{)}
\PY{n+nb}{print}\PY{p}{(}\PY{l+s+s2}{\PYZdq{}}\PY{l+s+s2}{We compute: sound speed c = }\PY{l+s+si}{\PYZpc{}g}\PY{l+s+s2}{, impedance Z = }\PY{l+s+si}{\PYZpc{}g}\PY{l+s+s2}{ }\PY{l+s+se}{\PYZbs{}n}\PY{l+s+s2}{\PYZdq{}} \PYZbs{}
      \PY{o}{\PYZpc{}} \PY{p}{(}\PY{n}{c}\PY{p}{,}\PY{n}{Z}\PY{p}{)}\PY{p}{)}
\end{Verbatim}
\end{tcolorbox}

    \begin{Verbatim}[commandchars=\\\{\}]
With density rho = 0.5,  bulk modulus K = 2
We compute: sound speed c = 2, impedance Z = 1

    \end{Verbatim}

    \begin{tcolorbox}[breakable, size=fbox, boxrule=1pt, pad at break*=1mm,colback=cellbackground, colframe=cellborder]
\prompt{In}{incolor}{9}{\boxspacing}
\begin{Verbatim}[commandchars=\\\{\}]
\PY{c+c1}{\PYZsh{} Call and print Riemann solution}
\PY{n}{states}\PY{p}{,} \PY{n}{speeds}\PY{p}{,} \PY{n}{reval} \PY{o}{=} \PYZbs{}
    \PY{n}{acoustics}\PY{o}{.}\PY{n}{exact\PYZus{}riemann\PYZus{}solution}\PY{p}{(}\PY{n}{ql} \PY{p}{,}\PY{n}{qr}\PY{p}{,} \PY{p}{[}\PY{n}{rho}\PY{p}{,} \PY{n}{bulk}\PY{p}{]}\PY{p}{)}
    
\PY{n+nb}{print}\PY{p}{(}\PY{l+s+s2}{\PYZdq{}}\PY{l+s+s2}{The states ql, qm and qr are: }\PY{l+s+s2}{\PYZdq{}}\PY{p}{)}
\PY{n+nb}{print}\PY{p}{(}\PY{n}{states}\PY{p}{,} \PY{l+s+s2}{\PYZdq{}}\PY{l+s+se}{\PYZbs{}n}\PY{l+s+s2}{\PYZdq{}}\PY{p}{)}
\PY{n+nb}{print}\PY{p}{(}\PY{l+s+s2}{\PYZdq{}}\PY{l+s+s2}{The left and right wave speeds are:}\PY{l+s+s2}{\PYZdq{}}\PY{p}{)}
\PY{n+nb}{print}\PY{p}{(}\PY{n}{speeds}\PY{p}{)}
\end{Verbatim}
\end{tcolorbox}

    \begin{Verbatim}[commandchars=\\\{\}]
The states ql, qm and qr are:
[[ 3.  5.  3.]
 [ 2.  0. -2.]]

The left and right wave speeds are:
[-2.  2.]
    \end{Verbatim}

    One way to visualize the Riemann solution for a system of two equations
is by looking at the \(p-u\) phase plane. In the figure below, we show
the two initial conditions of the Riemann problem \(q_\ell\) and \(q_r\)
as points in the phase space; the lines passing through these points
correspond to the eigenvectors, \(r_1\) and \(r_2\).

The middle state \(q_m\) is simply the intersection of the line in the
direction \(r_1\) passing through \(q_\ell\) and the line in the
direction \(r_2\) passing through \(q_r\). The structure of this
solution becomes evident from equation (\ref{eq:acussol}). The dashed
lines correspond to a line in the direction \(r_2\) passing through
\(q_\ell\) and a line in the direction \(r_1\) passing through \(q_r\);
these also intersect, but cannot represent a Riemann solution since they
would involve a wave going to the right but connected to \(q_\ell\) and
a wave going to the left but connected to \(q_r\).

In the live notebook, the cell below allows you to interactively adjust
the initial conditions the material parameters as well as the plot
range, so that you can explore how the structure of the solution in the
phase plane is affected by these quantities.

    \begin{tcolorbox}[breakable, size=fbox, boxrule=1pt, pad at break*=1mm,colback=cellbackground, colframe=cellborder]
\prompt{In}{incolor}{10}{\boxspacing}
\begin{Verbatim}[commandchars=\\\{\}]
\PY{n}{acoustics\PYZus{}demos}\PY{o}{.}\PY{n}{interactive\PYZus{}phase\PYZus{}plane}\PY{p}{(}\PY{n}{ql}\PY{p}{,}\PY{n}{qr}\PY{p}{,}\PY{n}{rho}\PY{p}{,}\PY{n}{bulk}\PY{p}{)}
\end{Verbatim}
\end{tcolorbox}

    
    \begin{verbatim}
interactive(children=(FloatSlider(value=3.0, description='$p_l$', max=10.0, min=0.01), FloatSlider(value=2.0, …
    \end{verbatim}

    
    
    \begin{verbatim}
Tab(children=(VBox(children=(HBox(children=(FloatSlider(value=3.0, description='$p_l$', max=10.0, min=0.01), F…
    \end{verbatim}

    
    
    \begin{verbatim}
Output()
    \end{verbatim}

    
    Note that the eigenvectors are given in terms of the impedance \(Z\),
which depends on the density \(\rho\) and the bulk modulus \(K\).
Therefore, when \(\rho\) and \(K\) are modified the eigenvectors change
and consequently the slope of the lines changes as well.

    \hypertarget{examples}{%
\subsection{Examples}\label{examples}}

We will use the exact solver in \url{exact_solvers/acoustics.py} and the
functions in \url{exact_solvers/acoustics_demos.py} to plot interactive
solutions for a few examples.

\hypertarget{shock-tube}{%
\subsubsection{Shock tube}\label{shock-tube}}

If there is a jump in pressure and the velocity is zero in both initial
states (the shock tube problem) then the resulting Riemann solution
consists of pressure jumps of equal magnitude propagating in both
directions, with equal and opposite jumps in velocity. This is the
linearized version of what is known in fluid dynamics as a shock tube
problem, since it emulates what would happen inside a shock tube, where
the air is initially stationary and a separate chamber at the end of the
tube is pressurized and then released.

    \begin{tcolorbox}[breakable, size=fbox, boxrule=1pt, pad at break*=1mm,colback=cellbackground, colframe=cellborder]
\prompt{In}{incolor}{11}{\boxspacing}
\begin{Verbatim}[commandchars=\\\{\}]
\PY{n}{ql} \PY{o}{=} \PY{n}{np}\PY{o}{.}\PY{n}{array}\PY{p}{(}\PY{p}{[}\PY{l+m+mi}{5}\PY{p}{,}\PY{l+m+mi}{0}\PY{p}{]}\PY{p}{)}
\PY{n}{qr} \PY{o}{=} \PY{n}{np}\PY{o}{.}\PY{n}{array}\PY{p}{(}\PY{p}{[}\PY{l+m+mi}{1}\PY{p}{,}\PY{l+m+mi}{0}\PY{p}{]}\PY{p}{)}
\PY{n}{rho} \PY{o}{=} \PY{l+m+mf}{1.0}
\PY{n}{bulk} \PY{o}{=} \PY{l+m+mf}{4.0}
\PY{n}{acoustics\PYZus{}demos}\PY{o}{.}\PY{n}{riemann\PYZus{}plot\PYZus{}pplane}\PY{p}{(}\PY{n}{ql}\PY{p}{,}\PY{n}{qr}\PY{p}{,}\PY{n}{rho}\PY{p}{,}\PY{n}{bulk}\PY{p}{)}
\end{Verbatim}
\end{tcolorbox}

    
    \begin{verbatim}
interactive(children=(FloatSlider(value=0.0, description='$t$', max=1.0), Dropdown(description='Characs.', opt…
    \end{verbatim}

    
    
    \begin{verbatim}
HBox(children=(FloatSlider(value=0.0, description='$t$', max=1.0), Dropdown(description='Characs.', options=(N…
    \end{verbatim}

    
    
    \begin{verbatim}
Output()
    \end{verbatim}

    
    We can also observe the structure of the solution in the phase plane. In
the second plot, we show the structure of the solution in the phase
plane.

    \hypertarget{reflection-from-a-wall}{%
\subsubsection{Reflection from a wall}\label{reflection-from-a-wall}}

As another example, suppose the pressure is initially the same in the
left and right states, while the velocities are non-zero with
\(u_r = -u_\ell > 0\). The flow is converging from both sides and
because of the symmetry of the initial states, the result is a middle
state \(q_m\) in which the velocity is 0 (and the pressure is higher
than on either side).

    \begin{tcolorbox}[breakable, size=fbox, boxrule=1pt, pad at break*=1mm,colback=cellbackground, colframe=cellborder]
\prompt{In}{incolor}{12}{\boxspacing}
\begin{Verbatim}[commandchars=\\\{\}]
\PY{n}{ql} \PY{o}{=} \PY{n}{np}\PY{o}{.}\PY{n}{array}\PY{p}{(}\PY{p}{[}\PY{l+m+mi}{2}\PY{p}{,}\PY{l+m+mi}{1}\PY{p}{]}\PY{p}{)}  
\PY{n}{qr} \PY{o}{=} \PY{n}{np}\PY{o}{.}\PY{n}{array}\PY{p}{(}\PY{p}{[}\PY{l+m+mi}{2}\PY{p}{,}\PY{o}{\PYZhy{}}\PY{l+m+mi}{1}\PY{p}{]}\PY{p}{)}  
\PY{n}{rho} \PY{o}{=} \PY{l+m+mf}{1.0}
\PY{n}{bulk} \PY{o}{=} \PY{l+m+mf}{1.5}
\PY{n}{acoustics\PYZus{}demos}\PY{o}{.}\PY{n}{riemann\PYZus{}plot\PYZus{}pplane}\PY{p}{(}\PY{n}{ql}\PY{p}{,}\PY{n}{qr}\PY{p}{,}\PY{n}{rho}\PY{p}{,}\PY{n}{bulk}\PY{p}{)}
\end{Verbatim}
\end{tcolorbox}

    
    \begin{verbatim}
interactive(children=(FloatSlider(value=0.0, description='$t$', max=1.0), Dropdown(description='Characs.', opt…
    \end{verbatim}

    
    
    \begin{verbatim}
HBox(children=(FloatSlider(value=0.0, description='$t$', max=1.0), Dropdown(description='Characs.', options=(N…
    \end{verbatim}

    
    
    \begin{verbatim}
Output()
    \end{verbatim}

    
    We again show the Riemann solution in space and in the phase plane,
where the symmetry is also evident.

    Disregarding the left half of the domain (\(x<0\)), one can view this as
a solution to the problem of an acoustic wave impacting a solid wall.
The result is a reflected wave that moves away from the wall; notice
that the velocity vanishes at the wall, as it must. This type of Riemann
solution is important when simulating waves in a domain with reflecting
boundaries. The reflecting condition can be imposed by the use of
fictitious \emph{ghost cells} that lie just outside the domain and whose
state is set by reflecting the interior solution with the symmetry just
described (equal pressure, negated velocity).

In reality, at a material boundary only part of a wave is reflected
while the rest is transmitted. This can be accounted for by including
the spatial variation in \(\rho, K\) and solving a variable-coefficient
Riemann problem.

    \hypertarget{interactive-phase-plane-with-solution-at-fixed-time}{%
\subsubsection{Interactive phase plane with solution at fixed
time}\label{interactive-phase-plane-with-solution-at-fixed-time}}

For a more general exploration of the solution to the acoustics
equation, we now show an interactive solution of the acoustics
equations. The initial states \(q_\ell\) and \(q_r\) can be modified by
dragging and dropping the points in the phase plane plot (in the
notebook version, or on
\href{http://www.clawpack.org/riemann_book/phase_plane/acoustics_small.html}{this
webpage}).

    \begin{tcolorbox}[breakable, size=fbox, boxrule=1pt, pad at break*=1mm,colback=cellbackground, colframe=cellborder]
\prompt{In}{incolor}{13}{\boxspacing}
\begin{Verbatim}[commandchars=\\\{\}]
\PY{n}{IFrame}\PY{p}{(}\PY{n}{src}\PY{o}{=}\PY{l+s+s1}{\PYZsq{}}\PY{l+s+s1}{phase\PYZus{}plane/acoustics\PYZus{}small\PYZus{}notitle.html}\PY{l+s+s1}{\PYZsq{}}\PY{p}{,} 
       \PY{n}{width}\PY{o}{=}\PY{l+m+mi}{980}\PY{p}{,} \PY{n}{height}\PY{o}{=}\PY{l+m+mi}{340}\PY{p}{)}
\end{Verbatim}
\end{tcolorbox}

            \begin{tcolorbox}[breakable, size=fbox, boxrule=.5pt, pad at break*=1mm, opacityfill=0]
\prompt{Out}{outcolor}{13}{\boxspacing}
\begin{Verbatim}[commandchars=\\\{\}]
<IPython.lib.display.IFrame at 0x7fa3748f3780>
\end{Verbatim}
\end{tcolorbox}
        
    \hypertarget{gaussian-initial-condition}{%
\subsubsection{Gaussian initial
condition}\label{gaussian-initial-condition}}

In this example, we use the first example described near the beginning
of this chapter. The initial condition is a Gaussian pressure
perturbation, while the initial velocity is zero. Reflecting boundary
conditions are imposed at \(x=-2\) and \(x=2\), so the wave is fully
reflected back, and we can see how it interacts with itself. This
animation is produced using a numerical method from
\href{http://www.clawpack.org/pyclaw/}{PyClaw}, and can be viewed in the
interactive notebook or on
\href{http://www.clawpack.org/riemann_book/html/acoustics_bump_animation.html}{this
webpage}.

    \begin{tcolorbox}[breakable, size=fbox, boxrule=1pt, pad at break*=1mm,colback=cellbackground, colframe=cellborder]
\prompt{In}{incolor}{14}{\boxspacing}
\begin{Verbatim}[commandchars=\\\{\}]
\PY{n}{anim} \PY{o}{=} \PY{n}{acoustics\PYZus{}demos}\PY{o}{.}\PY{n}{bump\PYZus{}animation}\PY{p}{(}\PY{n}{numframes} \PY{o}{=} \PY{l+m+mi}{50}\PY{p}{)}
\PY{n}{HTML}\PY{p}{(}\PY{n}{anim}\PY{p}{)}
\end{Verbatim}
\end{tcolorbox}

            \begin{tcolorbox}[breakable, size=fbox, boxrule=.5pt, pad at break*=1mm, opacityfill=0]
\prompt{Out}{outcolor}{14}{\boxspacing}
\begin{Verbatim}[commandchars=\\\{\}]
<IPython.core.display.HTML object>
\end{Verbatim}
\end{tcolorbox}
        

    % Add a bibliography block to the postdoc
    
    
    
\end{document}
